% VECTOR_GRAPHICS.TEX
%
% The documentation in this file is part of PyXPlot
% <http://www.pyxplot.org.uk>
%
% Copyright (C) 2006-2010 Dominic Ford <coders@pyxplot.org.uk>
%               2009-2010 Ross Church
%
% $Id$
%
% PyXPlot is free software; you can redistribute it and/or modify it under the
% terms of the GNU General Public License as published by the Free Software
% Foundation; either version 2 of the License, or (at your option) any later
% version.
%
% You should have received a copy of the GNU General Public License along with
% PyXPlot; if not, write to the Free Software Foundation, Inc., 51 Franklin
% Street, Fifth Floor, Boston, MA  02110-1301, USA

% ----------------------------------------------------------------------------

% LaTeX source for the PyXPlot Users' Guide

\chapter{Producing Vector Graphics}
\label{ch:vector_graphics}

In the previous two chapters, we have seen how the \indcmdt{plot} may be used to
produce single graphs of functions and \datafile s, how the \indcmdt{set
terminal} can be used to produce graphical output in a wide range of different
image formats (see Section~\ref{sec:set_terminal}), and the \indcmdt{set
papersize} can be used to produce postscript output to fit on different sizes
of paper (see Section~\ref{sec:set_papersize}). Often, however, there is a need
to produce more sophisticated vector graphics.  For example, a figure with
several plots side-by-side may be required, some line-art may be wanted to
superimpose on top of or next to a graph, or what may be wanted may not be a
graph at all, but rather a high-precision technical diagram. In this chapter,
we turn our attention to such cases.

\section{Multiplot Mode}
\label{sec:multiplot}
\index{multiplot}

PyXPlot has two modes in which it can produce graphical output. In {\it
singleplot} mode, the default, each time the {\tt plot} command is issued, the
canvas is wiped clean and the new plot is placed alone on a blank page. In {\it
multiplot} mode, vector graphics objects accumulate on the canvas. Each time
the {\tt plot} command is issued, the new graph is placed on top of any other
objects which were already on the canvas, and many plots can be placed
side-by-side.

The user can switch between these two modes of operation by issuing the
commands \indcmdts{set multiplot} and \indcmdts{set nomultiplot}. The
\indcmdt{set origin} is required for multiplot mode to be particularly useful
when placing plots side-by-side: it sets the position on the page of the
lower-left corner of the next plot to be produced. It takes a comma-separated
$(x,y)$ co-ordinate pair, which may have units of length, or, if dimensionless,
are assumed to be measured in centimetres. The following example plots a graph
of $\sin(x)$ to the left of a plot of $\cos(x)$:
\begin{verbatim}
set multiplot
set width 8
plot sin(x)
set origin 10,0
plot cos(x)
\end{verbatim}

All objects on a multiplot canvas have a unique identification number.  By
default, these count up from one, such that the first item placed on the canvas
is number one, the next is number two, and so forth. Alternatively, the user
may specify a particular number for a particular object by supplying the
modifier {\tt item} to the {\tt plot} command, followed by an integer
identification number, as in the following example:
\begin{verbatim}
plot item 6 'data.dat'
\end{verbatim}
If there were already an object on the canvas with identification number~6,
this object would be deleted and replaced with the new object.

A list of all of the objects on the current multiplot canvas can be obtained
using the \indcmdt{list}, which produces output in the following format:
\begin{verbatim}
# ID   Command
    1  plot item 1 'data1.dat'
    2  plot item 2 'data2.dat'
    3  [deleted] plot item 3 'data3.dat'
\end{verbatim}

A multiplot canvas can be wiped clean by issuing the \indcmdt{clear}, which
removes all items currently on the canvas. Alternatively, individual items may
be removed using the \indcmdt{delete}, which should be followed by a
comma-separated list of the identification numbers of the objects to be
deleted.  Deleted items may be restored using the \indcmdt{undelete}, which
likewise takes a comma-separated list of the identification numbers of the
objects to be restored, e.g.:
\begin{verbatim}
delete 1,2
undelete 2
\end{verbatim}
Once a canvas has been cleared using the \indcmdt{clear}, however, there is no
way to restore it.  Objects may be moved around on the canvas using the
\indcmdt{move}. For example, the following would move item 23 to position
$(8,8)$ measured in inches:
\begin{verbatim}
move 23 to 8*unit(in), 8*unit(in)
\end{verbatim}

\subsection{Settings Associated with Multiplot Items}

Of the settings which can be set with the \indcmdt{set}, some refer to
PyXPlot's global environment and whole multiplot canvases. Others, such as {\tt
set width} and {\tt set origin} refer specifically to individual graphs and
vector graphics items. For this reason, whenever a new multiplot graphics item
is produced, it takes a copy of the settings which are specific to it, allowing
these settings to be changed by the user before producing other multiplot
items, without affecting previous items. The settings associated with a
particular multiplot item can be queried by passing the modifier {\tt item} to
the \indcmdt{show}, followed by the integer identification number of the item,
as in the examples:
\begin{verbatim}
show item 3 width    # Shows the width of item 3
show item 3 settings # Shows all settings associated with item 3
\end{verbatim}

The settings associated with a particular multiplot item can be changed by
passing the same {\tt item} modifier to the \indcmdt{set}, as in the example,
which sets the width of item~3 to be $10\,\mathrm{cm}$:
\begin{verbatim}
set item 3 width 10*unit(cm)
\end{verbatim}
After making such changes, the \indcmdt{refresh} is useful: it produces a new
graphical image of the current multiplot to reflect any settings which have
been changed. The following example would produce a pair of plots, and then
change the colour of the text on the first plot:
\begin{verbatim}
set multiplot
plot f(x)
set origin 10,0
plot g(x)
set item 1 textcolour red
refresh
\end{verbatim}

Another common use of the \indcmdt{refresh} is to produce multiple
copies of an image in different graphical formats. For example, having just
developed a multiplot canvas interactively in the {\tt X11\_singlewindow},
copies can be produced as {\tt eps} and {\tt jpeg} images using the following
commands:
\begin{verbatim}
set terminal eps
set output 'figure.eps'
refresh
set terminal jpeg
set output 'figure.jpg'
refresh
\end{verbatim}

\subsection{Reordering Multiplot Items}

Items on multiplot are drawn in order of increasing identification number, and
thus items with low identification numbers are drawn first, at the back of the
multiplot, and items with higher identification numbers are later, towards the
front of the multiplot. When new items are added, they are given higher
identification numbers than previous items, and appear at the front of the
multiplot.

If this ordering is not the desired ordering, then the \indcmdt{swap} may be
used to rearrange items. It simply takes two multiplot items, and swaps their
identification numbers and hence positions in the ordered sequence in which the
items on the multiplot are drawn. Thus, if, for example, the corner of item~3
disappears behind the corner of item~5, when the converse effect is actually
desired, the following command should be issued:
\begin{verbatim}
swap 3 5
\end{verbatim}

\subsection{The Construction of Large Multiplots}
\label{sec:set_display}

By default, whenever an item is added to a multiplot, or an existing item moved
or replotted, the whole multiplot is replotted to show the change. This can be
a time consuming process on large and complex multiplots. For this reason, the
\indcmdt{set nodisplay} is provided, which stops PyXPlot from producing any
output. The \indcmdt{set display} can subsequently be issued to return to
normal behaviour.

This can be especially useful in scripts which produce large multiplots. There
is no point in producing output at each step in the construction of a large
multiplot, and a great speed increase can be achieved by wrapping the script
with:

\begin{verbatim}
set nodisplay
[...prepare large multiplot...]
set display
refresh
\end{verbatim}

\section{Linked Axes and Galleries of Plots}

\section{The {\tt replot} Command Revisited}

In multiplot mode, the \indcmdt{replot} can be used to modify the last plot
added to the page. For example, the following would change the title of the
latest plot to `foo', and add a plot of the function $g(x)$ to it:

\begin{verbatim}
set title 'foo'
replot cos(x)
\end{verbatim}

Additionally, it is possible to modify any plot on the page by adding an {\tt
item} modifier to the {\tt replot} statement to specify which plot should be
replotted.  The following example would produce two plots, and then add an
additional function to the first plot:

\begin{verbatim}
set multiplot
plot f(x)
set origin 10,0
plot g(x)
replot item 1 h(x)
\end{verbatim}

If no {\tt item} number is specified, then the \indcmdt{replot} acts by default
upon the most recent plot to have been added to the multiplot canvas.

\section{Adding Other Vector Graphics Objects}

In addition to graphs, a range of other objects can be placed on multiplot
canvases:
\begin{itemize}
\item Arcs of circles (the \indcmdt{arc}).
\item Arrows (the \indcmdt{arrow}).
\item Rectangular boxes (the \indcmdt{box}).
\item Circles (the \indcmdt{circle}).
\item Ellipses (the \indcmdt{ellipse}).
\item Encapsulated postscript images (the \indcmdt{eps}).
\item Graphical images in {\tt bmp}, {\tt gif}, {\tt jpeg} or {\tt png} formats (the \indcmdt{image}).
\item Lines (the \indcmdt{line}).
\item Piecharts (the \indcmdt{piechart}).
\item Points labelled by crosses and other symbols (the \indcmdt{point}).
\item Text labels (the \indcmdt{text}).
\end{itemize}
Put together, these commands can be used to produce a wide range of vector
graphics. In the remainder of this chapter, we describe these commands in turn,
providing a variety of examples of their use.

\subsection{The {\tt text} Command}

Text labels may be added to multiplot canvases using the \indcmdt{text}. This
has the following syntax:

\begin{verbatim}
text 'This is some text' at x,y
\end{verbatim}

In this case, the string `This is some text' would be rendered at position
$(x,y)$ on the multiplot. As with the \indcmdt{set label}, a colour may
optionally be specified with the {\tt with colour} modifier, as well as a
rotation angle to rotate text labels through any given angle, measured in
degrees counter-clockwise. For example:\indkey{rotate}

\begin{verbatim}
text 'This is some text' at x,y rotate r with colour red
\end{verbatim}

The commands \indcmdts{set textcolour}, \indcmdts{set texthalign} and
\indcmdts{set textvalign} can be used to set the colour and alignment of the
text produced with the \indcmdt{text}. Alternatively, the \indcmdt{text} takes
three modifiers to control the alignment of the text which override these {\tt
set} commands. The {\tt halign} and {\tt valign} modifiers may be followed by
any of the settings which may follow the {\tt set texthalign} and {\tt set
textvalign} commands respectively, as in the following examples:

\begin{verbatim}
text 'This is some text' at 0,0 halign left valign top
text 'This is some text' at 0,0 halign right valign centre
\end{verbatim}

\noindent The {\tt gap} modifier allows a gap to be inserted in the alignment
of the text. For example, the string {\tt halign left gap 3*unit(mm)} would
cause text to be rendered with its left side $3\,\mathrm{mm}$ to the right of
the position specified for the text. This is useful for labelling points on
diagrams, where the labels should be slightly offset from the points that they
are associated with. If the {\tt gap} modifier is followed by a dimensionless
number, rather than one with dimensions of lengths, then it is assumed to be
measured in centimetres.

It should be noted that the \indcmdt{text} can also be used outside of the
multiplot environment, to render a single piece of short text instead of a
graph. One obvious application is to produce equations rendered as graphical
files which can subsequently be imported into documents, slideshows or
webpages.\index{presentations}

\subsection{The {\tt arrow} and {\tt line} Commands}

Arrows may also be added to multiplot canvases using the \indcmdt{arrow}, which
has syntax:

\begin{verbatim}
arrow from x,y to x,y
\end{verbatim}

The \indcmdt{arrow} may be followed by the \indmodt{with} keyword to specify to
style of the arrow. The line type, line width and colour of the arrow, may be
specified using the same syntax as used in the plot command, using the {\tt
linetype}, {\tt linewidth} and {\tt colour} modifiers after the word {\tt
with}, as in the example:

\begin{verbatim}
arrow from 0,0 to 10,10 \
with linetype 2 linewidth 5 colour red
\end{verbatim}

\noindent The style of the arrow may also be specified after the word {\tt
with}, and three options are available: {\tt head} (the default), {\tt nohead},
which produces line segments with no arrowheads on them, and {\tt twoway},
which produced arrows with heads on both ends.

The \indcmdt{arrow} has a twin, the \indcmdt{line}, which has the same syntax
but with a different style setting of {\tt nohead}.

\example{ex:notice}{A simple notice with text and arrow}{
In this example script, we use PyXPlot's {\tt arrow} and {\tt text} commands to
produce a simple notice advertising that a lecture has moved to a different
seminar room:
\nlscf
\noindent {\tt \# Turn on multiplot mode }\newline
\noindent {\tt set multiplot ; set nodisplay }\newline
\\
\noindent {\tt \# Set the dimensions of the notice }\newline
\noindent {\tt w = 20~~~~~~~~\# Width of notice / cm }\newline
\noindent {\tt h = w/sqrt(2)~\# Height of notice / cm }\newline
\\
\noindent {\tt \# Put a rectangular box around the notice }\newline
\noindent {\tt line from 0,0 to w,0 with linewidth 5 }\newline
\noindent {\tt line from w,0 to w,h with linewidth 5 }\newline
\noindent {\tt line from w,h to 0,h with linewidth 5 }\newline
\noindent {\tt line from 0,h to 0,0 with linewidth 5 }\newline
\\
\noindent {\tt \# Write the text of the notice in big letters }\newline
\noindent {\tt set texthalign centre ; set fontsize 3 }\newline
\noindent {\tt text "$\backslash$bf Astrophysical Fluids Lecture" at w/2,3/4*h }\newline
\noindent {\tt text "$\backslash$bf MOVED to Seminar Room 3" at w/2, h/2 }\newline
\noindent {\tt arrow from w/4, h/4 to 3/4*w, h/4 with linewidth 8 }\newline
\\
\noindent {\tt \# Display the notice }\newline
\noindent {\tt set display ; refresh }\newline
\nlfcf
The resulting notice is shown below:
\nlscf
\centerline{\includegraphics[width=\textwidth]{examples/eps/ex_notice.eps}}
}

\example{ex:euclid}{Reproducing a diagram from Euclid's {\it Elements}}{
In this more extended example script, we use PyXPlot's {\tt arrow} and {\tt
text} commands to produce a diagram illustrating the 47th Proposition from
Euclid's First Book of {\it Elements}, better known as Pythagoras' Theorem. A
full text of the proof which accompanies this diagram can be found at
\url{http://www.gutenberg.org/etext/21076}.
\nlscf
{\footnotesize
\noindent {\tt set multiplot ; set nodisplay}\newline
}\\{\footnotesize
\noindent {\tt \# Lengths of three sides of triangle}\newline
\noindent {\tt AB = 2*unit(cm)}\newline
\noindent {\tt AC = 4*unit(cm)}\newline
\noindent {\tt BC = hypot(AC, AB) \# Hypotenuse}\newline
\noindent {\tt CBA = atan2(AC, AB) \# Angle CBA}\newline
}\\{\footnotesize
\noindent {\tt \# Positions of three corners of triangle}\newline
\noindent {\tt Bx = 0*unit(cm)~~~~~~~; By = 0*unit(cm) \# The origin}\newline
\noindent {\tt Cx = Bx + BC~~~~~~~~~~; Cy = By}\newline
\noindent {\tt Ax = Bx + AB*cos(CBA) ; Ay = By + AB*sin(CBA)}\newline
}\\{\footnotesize
\noindent {\tt \# Positions of constructed points}\newline
\noindent {\tt Dx = Bx~~~~~~~~~~~~~~~; Dy = -BC}\newline
\noindent {\tt Lx = Ax~~~~~~~~~~~~~~~; Ly = Dy}\newline
\noindent {\tt Ex = Cx~~~~~~~~~~~~~~~; Ey = Dy}\newline
}\\{\footnotesize
\noindent {\tt Hx = Bx + (AB + AC) * cos(CBA)}\newline
\noindent {\tt Hy = By + (AB + AC) * sin(CBA)}\newline
\noindent {\tt Kx = Cx + (~~~~~AC) * cos(CBA)}\newline
\noindent {\tt Ky = Cy + (~~~~~AC) * sin(CBA)}\newline
}\\{\footnotesize
\noindent {\tt Fx = Bx + AB*cos(CBA+90*unit(deg))}\newline
\noindent {\tt Fy = By + AB*sin(CBA+90*unit(deg))}\newline
\noindent {\tt Gx = Ax + AB*cos(CBA+90*unit(deg))}\newline
\noindent {\tt Gy = Ay + AB*sin(CBA+90*unit(deg))}\newline
}\\{\footnotesize
\noindent {\tt \# Construct diagram}\newline
\noindent {\tt box from Dx,Dy to Cx,Cy with fillcol grey80}\newline
\noindent {\tt box at Ax,Ay width AC height AC rot CBA-90*unit(deg) with fillcol grey80}\newline
\noindent {\tt box at Bx,By width AB height AB rot CBA with fillcol grey80}\newline
\noindent {\tt line from Bx,By to Kx,Ky}\newline
\noindent {\tt line from Fx,Fy to Cx,Cy}\newline
\noindent {\tt line from Ax,Ay to Dx,Dy}\newline
\noindent {\tt line from Ax,Ay to Lx,Ly}\newline
\noindent {\tt line from Ax,Ay to Ex,Ey}\newline
}\\{\footnotesize
\noindent {\tt \# Label diagram}\newline
\noindent {\tt set fontsize 1.3}\newline
\noindent {\tt TG = 0.5*unit(mm) \# Gap left between labels and figure}\newline
\noindent {\tt text "A" at Ax,Ay gap TG*5 hal c val b}\newline
\noindent {\tt text "B" at Bx,By gap TG~~~hal r val t}\newline
\noindent {\tt text "C" at Cx,Cy gap TG~~~hal l val t}\newline
\noindent {\tt text "D" at Dx,Dy gap TG~~~hal c val t}\newline
\noindent {\tt text "E" at Ex,Ey gap TG~~~hal c val t}\newline
\noindent {\tt text "F" at Fx,Fy gap TG~~~hal r val c}\newline
\noindent {\tt text "G" at Gx,Gy gap TG~~~hal c val b}\newline
\noindent {\tt text "H" at Hx,Hy gap TG~~~hal c val b}\newline
\noindent {\tt text "K" at Kx,Ky gap TG~~~hal l val c}\newline
\noindent {\tt text "L" at Lx,Ly gap TG~~~hal c val t}\newline
}\\{\footnotesize
\noindent {\tt \# Display diagram}\newline
\noindent {\tt set display ; refresh}\newline
}
\nlfcf
The resulting diagram is shown below:
\nlscf
\centerline{\includegraphics[width=8cm]{examples/eps/ex_euclid_I_47.eps}}
}

\example{ex:nanotubes}{The conductivity of nanotubes}{
foo
\nlscf
\centerline{\includegraphics[width=9cm]{examples/eps/ex_nanotubes.eps}}
}

\subsection{The {\tt image} Command}

Graphical images in {\tt bmp}, {\tt gif}, {\tt jpeg} or {\tt png} format may be
placed on multiplot canvases using the \indcmdt{image}\footnote{To maintain
compatibility with historic versions of PyXPlot, the {\tt image} command may
also be spelt {\tt jpeg}, with the identical syntax thereafter.}. In its
simplest form, this has the syntax:
\begin{verbatim}
image 'filename' at x,y width w
\end{verbatim}

As an alternative to the \indkeyt{width} keyword the height of the image can be
specified, using the analogous \indkeyt{height} keyword.  An optional angle can
also be specified using the \indkeyt{rotate} keyword; this causes the included
image to be rotated counter-clockwise by a specified angle, measured in
degrees.  The keyword {\tt smooth} may optionally be supplied to cause the
pixels of the image to be interpolated\footnote{Many commonly-used postscript
display engines, including GhostScript, do not support this functionality.}.

Images which include transparency are supported. The optional keyword {\tt
notransparent} may be supplied to the \indcmdt{image} to cause transparent
regions to be filled with the image's default background colour. Alternatively,
the keyword {\tt transparent} may be followed by the name of an RGB colour in
the form {\tt rgb<r>:<g>:<b>} to cause a particular colour to be converted to
transparent.

\subsection{The {\tt eps} Command}

Vector graphic images in eps format may be placed on multiplot canvases
using the \indcmdt{eps}, which has a syntax analogous to the {\tt image}
command.  However neither height nor width need be specified; in this case the
image will be included at its native size.  For example:

\begin{verbatim}
eps 'filename' at 3,2 rotate 5
\end{verbatim}

\noindent will place the eps file with its bottom-left corner at position
$(3,2)$\,cm from the origin, rotated counter-clockwise through 5 degrees.

\subsection{The {\tt box} and {\tt circle} Commands}
\label{sec:rectangle}

Rectangular boxes and circles may be placed on multiplot canvases
using the {\tt box} and {\tt circle} commands\indcmd{box}\indcmd{circle}, as
in:

\begin{verbatim}
box from 0*unit(mm),0*unit(mm) to 25*unit(mm),70*unit(mm)
circle at 0*unit(mm),0*unit(mm) radius 70*unit(mm)
\end{verbatim}

\noindent In the former case, two corners of the rectangle are specified,
meanwhile in the latter case the centre of the circle and its radius are
specified. The \indcmdt{box} may also be invoked by the synonym {\tt
rectangle}\indcmd{rectangle}. Boxes may be rotated using an optional {\tt
rotate} modifier, which may be followed by a counter-clockwise rotational angle
which may either have dimensions of angle, or is assumed to be in degrees if
dimensionless. The rotation is performed about the centre of the rectangle:

\begin{verbatim}
box from 0,0 to 10,3 rotate 45
\end{verbatim}

The positions and dimensions of boxes may also be specified by giving the
position of one of the corners of the box, together with its width and height.
The specified corner is assumed to be the bottom-left corner if both the
specified width and height are positive; other corners may be specified if the
supplied width and/or height are negative. If such boxes are rotated, the
rotation is about the specified corner:

\begin{verbatim}
box at 0,0 width 10 height 3 rotate 45
\end{verbatim}

The line type, line width, and colour of line with which the outlines of boxes
and circles are drawn may be specified as in the {\tt arrow} command, for
example:

\begin{verbatim}
circle at 0,0 radius 5 with linetype 1 linewidth 2 colour red
\end{verbatim}

\noindent The shapes may be filled by specifying a {\tt fillcolour}:

\begin{verbatim}
circle at 0,0 radius 5 with lw 10 colour red fillcolour yellow
\end{verbatim}

\example{ex:noentry}{A simple no-entry sign}{
In this example script, we use PyXPlot's {\tt box} and {\tt circle} commands to
produce a no-entry sign warning passers by that code monkeys can turn nasty
when interrupted from their work.
\nlscf
\noindent {\tt set multiplot ; set nodisplay}\newline
\noindent {\tt }\newline
\noindent {\tt w = 10 \# Width of sign / cm}\newline
\\
\noindent {\tt \# Make no-entry sign}\newline
\noindent {\tt circle at 0,0 radius w with col null fillcol red}\newline
\noindent {\tt box from -(0.8*w),-(0.2*w) to (0.8*w),(0.2*w) $\backslash$}\newline
\noindent {\tt \phantom{xxxx}with col null fillcol white}\newline
\\
\noindent {\tt \# Put a warning beneath the sign}\newline
\noindent {\tt set fontsize 3}\newline
\noindent {\tt set texthalign centre ; set textvalign centre}\newline
\noindent {\tt text "$\backslash$bf Keep Out! Code Monkey at work!"}\newline
\\
\noindent {\tt \# Display sign}\newline
\noindent {\tt set display ; refresh}\newline
\vspace{2mm}\\
\nlfcf
The resulting sign is shown below:
\nlscf
\centerline{\includegraphics[width=5cm]{examples/eps/ex_noentry.eps}}
}

\example{ex:mandelbrot}{Producing an image of the Mandelbrot Set}{
The Mandelbrot set is a set of points in the complex plane, whose boundary
forms a fractal. It is an iconic image of the power of chaos theory to produce
complicated structure out of simple algorithms, which we implement in this
example using PyXPlot's loop constructs.
\nlnp
A point $c$ in the complex plane is defined to lie within the Mandelbrot set if
the complex sequence of numbers
\begin{displaymath}
z_{n+1} = z_n^2 + c,
\end{displaymath}
subject to the starting condition $z_0=0$, remains bounded. In practice, the
sequence is certain to diverge if $|z_n|$ ever exceeds~2.  It cannot generally
be proven not to diverge at any particular point, but is declared to remain
bounded in our algorithm if $|z_n|$ has not exceeded~2 within a certain number
{\tt StepsMax} of steps. To increase the aesthetic appeal of the fractal,
pixels outside of the fractal are assigned different colours depending upon how
many iterations were required before the sequence moved outside the circle
$|z_n|=2$.
\nlnp
This script makes use of the \indcmdt{rectangle} to paint each pixel of the
fractal one by one, which is highly inefficient, but demonstrates PyXPlot's
versatility.
\nlscf
\noindent {\tt set~multiplot }\newline
\noindent {\tt set~nodisplay }\newline
\noindent {\tt set~numerics~complex }\newline
\noindent {\tt StepsMax=10 }\newline
\noindent {\tt yn=0 }\newline
\noindent {\tt for~y=2~to~-2~step~-0.05 }\newline
\noindent {\tt \phantom{x}\{ }\newline
\noindent {\tt \phantom{xx}xn=0 }\newline
\noindent {\tt \phantom{xx}for~x=-2~to~2~step~0.05 }\newline
\noindent {\tt \phantom{xxx}\{ }\newline
\noindent {\tt \phantom{xxxx}zi~=~x+i*y;~z=zi;~iter=0; }\newline
\noindent {\tt \phantom{xxxx}while~(abs(z)<2)~and~(iter<StepsMax)~$\backslash$ }\newline
\noindent {\tt \phantom{xxxxxxxx}\{~;~z~=~z**2~+~zi~;~iter=iter+1~;~\} }\newline
\noindent {\tt \phantom{xxxx}cd~=~(1-iter/StepsMax)*256 }\newline
\noindent {\tt \phantom{xxxx}rectangle~from~xn,yn~to~xn+0.21,yn+0.21~with~col~null~$\backslash$ }\newline
\noindent {\tt \phantom{xxxxxxxxxxx}fillc~rgb(cd):(cd):(cd) }\newline
\noindent {\tt \phantom{xxxx}xn=xn+0.2 }\newline
\noindent {\tt \phantom{xxx}\} }\newline
\noindent {\tt \phantom{xx}yn=yn+0.2 }\newline
\noindent {\tt \phantom{x}\} }\newline
\noindent {\tt }\newline
\noindent {\tt \#~Display~output }\newline
\noindent {\tt set~display }\newline
\noindent {\tt refresh }\newline
\nlscf
The resulting image is shown below:
\nlscf
\begin{center}
\includegraphics[width=8cm]{examples/eps/ex_mandelbrot.eps}
\end{center}
}

\subsection{The {\tt arc} Command}
\label{sec:arc}

Partial arcs of circles may be drawn using \indcmdt{arc}. This has similar
syntax to \indcmdt{circle}, but takes two additional angle, measured clockwise
from the upward vertical direction, which specify the extent of the arc to be
drawn. The arc is drawn clockwise from start to end, and hence the following
two instructions draw two complementary arcs which together form a complete
circle:

\begin{verbatim}
set multiplot
arc at 0,0 radius 5 from -90 to   0 with lw 3 col red
arc at 0,0 radius 5 from   0 to -90 with lw 3 col green
\end{verbatim}

\noindent If a {\tt fillcolour} is specified, then a pie-wedge is drawn:

\begin{verbatim}
arc at 0,0 radius 5 from 0 to 30 with lw 3 fillcolour red
\end{verbatim}

\example{ex:triangle}{A labelled diagram of a triangle}{
In this example, we make a subroutine to draw labelled diagrams of the interior
angles of triangles, taking as its inputs the lengths of the three sides of the
triangle to be drawn and the position of its lower-left corner. The subroutine
calculates the positions of the three vertices of the triangle and then labels
them. We use PyXPlot's automatic handling of physical units to generate the
\LaTeX\ strings required to label the side lengths in centimetres and the
angles in degrees. We use PyXPlot's {\tt arc} command to draw angle symbols in
the three corners of a triangle.
\nlscf
{\footnotesize
\noindent {\tt \# Define subroutine for drawing triangles}\newline
\noindent {\tt subroutine TriangleDraw(Bx,By,AB,AC,BC)}\newline
\noindent {\tt \phantom{x}\{}\newline
\noindent {\tt \phantom{xx}\# Use cosine rule to find interior angles}\newline
\noindent {\tt \phantom{xx}ABC = acos((AB**2 + BC**2 - AC**2) / (2*AB*BC))}\newline
\noindent {\tt \phantom{xx}BCA = acos((BC**2 + AC**2 - AB**2) / (2*BC*AC))}\newline
\noindent {\tt \phantom{xx}CAB = acos((AC**2 + AB**2 - BC**2) / (2*AC*AB))}\newline
}\\{\footnotesize
\noindent {\tt \phantom{xx}\# Positions of three corners of triangle}\newline
\noindent {\tt \phantom{xx}Cx = Bx + BC~~~~~~~~~~; Cy = By}\newline
\noindent {\tt \phantom{xx}Ax = Bx + AB*cos(ABC) ; Ay = By + AB*sin(ABC)}\newline
}\\{\footnotesize
\noindent {\tt \phantom{xx}\# Draw triangle}\newline
\noindent {\tt \phantom{xx}line from Ax,Ay to Bx,By}\newline
\noindent {\tt \phantom{xx}line from Ax,Ay to Cx,Cy}\newline
\noindent {\tt \phantom{xx}line from Bx,By to Cx,Cy}\newline
}\\{\footnotesize
\noindent {\tt \phantom{xx}\# Draw angle symbols}\newline
\noindent {\tt \phantom{xx}ArcRad = 4*unit(mm) \# Radius of angle arcs}\newline
\noindent {\tt \phantom{xx}arc at Bx,By radius ArcRad from~~90*unit(deg)-ABC to~~90*unit(deg)}\newline
\noindent {\tt \phantom{xx}arc at Cx,Cy radius ArcRad from -90*unit(deg)~~~~~to -90*unit(deg)+BCA}\newline
\noindent {\tt \phantom{xx}arc at Ax,Ay radius ArcRad from~~90*unit(deg)+BCA to 270*unit(deg)-ABC}\newline
}\\{\footnotesize
\noindent {\tt \phantom{xx}\# Label lengths of sides}\newline
\noindent {\tt \phantom{xx}set unit of length cm \# Display lengths in cm}\newline
\noindent {\tt \phantom{xx}set numeric sigfig 3 display latex \# Correct to 3 significant figure}\newline
\noindent {\tt \phantom{xx}set fontsize 1.2 \# And in slightly bigger text than normal}\newline
\noindent {\tt \phantom{xx}TextGap = 1*unit(mm)}\newline
\noindent {\tt \phantom{xx}text "\%s"\%(BC) at (Bx+Cx)/2,(By+Cy)/2 gap TextGap hal c val t}\newline
\noindent {\tt \phantom{xx}text "\%s"\%(AB) at (Ax+Bx)/2,(Ay+By)/2 gap TextGap rot  ABC hal c val b}\newline
\noindent {\tt \phantom{xx}text "\%s"\%(AC) at (Ax+Cx)/2,(Ay+Cy)/2 gap TextGap rot -BCA hal c val b}\newline
}\\{\footnotesize
\noindent {\tt \phantom{xx}\# Label angles}\newline
\noindent {\tt \phantom{xx}set unit of angle degree \# Display angles in degrees}\newline
\noindent {\tt \phantom{xx}ArcRad2 = 1.4 * ArcRad \# Distance of text from apex of angles}\newline
\noindent {\tt \phantom{xx}text "\%s"\%(CAB) at Ax+ArcRad2*sin(ABC-BCA),Ay-ArcRad2*cos(ABC-BCA) hal c val t}\newline
\noindent {\tt \phantom{xx}text "\%s"\%(ABC) at Bx+ArcRad2*cos(ABC/2),By+ArcRad2*sin(ABC/2) hal l val c}\newline
\noindent {\tt \phantom{xx}text "\%s"\%(BCA) at Cx-ArcRad2*cos(BCA/2),Cy+ArcRad2*sin(BCA/2) hal r val c}\newline
}\\{\footnotesize
\noindent {\tt \phantom{xx}\# Label points ABC}\newline
\noindent {\tt \phantom{xx}text "A" at Ax,Ay gap TextGap hal c val b}\newline
\noindent {\tt \phantom{xx}text "B" at Bx,By gap TextGap hal r val c}\newline
\noindent {\tt \phantom{xx}text "C" at Cx,Cy gap TextGap hal l val c}\newline
\noindent {\tt \phantom{x}\}}\newline
}\\{\footnotesize
\noindent {\tt \# Display diagram with three triangles}\newline
\noindent {\tt set multiplot ; set nodisplay}\newline
\noindent {\tt call TriangleDraw(2.8*unit(cm),3.4*unit(cm), 3*unit(cm),4*unit(cm),4*unit(cm))}\newline
\noindent {\tt call TriangleDraw(0.0*unit(cm),0.0*unit(cm), 3*unit(cm),4*unit(cm),5*unit(cm))}\newline
\noindent {\tt call TriangleDraw(6.5*unit(cm),0.0*unit(cm), 3*unit(cm),3*unit(cm),3*unit(cm))}\newline
\noindent {\tt set display ; refresh}\newline
}
\nlfcf
The resulting diagram is shown below:
\nlscf
\centerline{\includegraphics{examples/eps/ex_triangle.eps}}
}

\example{ex:lens}{A labelled diagram of a converging lens forming a real image}{
In this example, we make a subroutine to draw labelled diagrams of converging
lenses forming real images.
\nlscf
{\footnotesize
\noindent {\tt subroutine LensDraw(x0,y0,u,h,f)}\newline
\noindent {\tt \phantom{x}\{}\newline
\noindent {\tt \phantom{xx}\# Use the thin-lens equation to find v and H}\newline
\noindent {\tt \phantom{xx}v = 1/(1/f - 1/u)}\newline
\noindent {\tt \phantom{xx}H = h * v / u}\newline
}\\{\footnotesize
\noindent {\tt \phantom{xx}\# Draw lens}\newline
\noindent {\tt \phantom{xx}lc = 5.5*unit(cm) \# Radius of curvature of lens}\newline
\noindent {\tt \phantom{xx}lt = 0.5*unit(cm) \# Thickness of lens}\newline
\noindent {\tt \phantom{xx}la = acos((lc-lt/2)/lc) \# Angular size of lens from centre of curvature}\newline
\noindent {\tt \phantom{xx}lh = lc*sin(la) \# Physical height of lens on paper}\newline
\noindent {\tt \phantom{xx}arc at x0-(lc-lt/2),y0 radius lc from~~90*unit(deg)-la to~~90*unit(deg)+la}\newline
\noindent {\tt \phantom{xx}arc at x0+(lc-lt/2),y0 radius lc from 270*unit(deg)-la to 270*unit(deg)+la}\newline
\noindent {\tt \phantom{xx}set texthalign right ; set textvalign top}\newline
\noindent {\tt \phantom{xx}point at x0-f,y0 label "\$f\$"}\newline
\noindent {\tt \phantom{xx}set texthalign left~~; set textvalign bottom}\newline
\noindent {\tt \phantom{xx}point at x0+f,y0 label "\$f\$"}\newline
}\\{\footnotesize
\noindent {\tt \phantom{xx}\# Draw object and image}\newline
\noindent {\tt \phantom{xx}arrow from x0-u,y0 to x0-u,y0+h with lw 2}\newline
\noindent {\tt \phantom{xx}arrow from x0+v,y0 to x0+v,y0-H with lw 2}\newline
\noindent {\tt \phantom{xx}text "\$h\$" at x0-u,y0+h/2 hal l val c gap unit(mm)}\newline
\noindent {\tt \phantom{xx}text "\$H\$" at x0+v,y0-H/2 hal l val c gap unit(mm)}\newline
}\\{\footnotesize
\noindent {\tt \phantom{xx}\# Draw construction lines}\newline
\noindent {\tt \phantom{xx}line from x0-u,y0 to x0+v,y0 with lt 2 \# Optic axis}\newline
\noindent {\tt \phantom{xx}line from x0-u,y0+h to x0+v,y0-H \# Undeflected ray through origin}\newline
\noindent {\tt \phantom{xx}line from x0-u,y0+h to x0,y0+h}\newline
\noindent {\tt \phantom{xx}line from x0,y0+h to x0+v,y0-H}\newline
\noindent {\tt \phantom{xx}line from x0+v,y0-H to x0,y0-H}\newline
\noindent {\tt \phantom{xx}line from x0,y0-H to x0-u,y0+h}\newline
}\\{\footnotesize
\noindent {\tt \phantom{xx}\# Label distances u and v}\newline
\noindent {\tt \phantom{xx}ylabel = y0-lh-2*unit(mm)}\newline
\noindent {\tt \phantom{xx}arrow from x0-u,ylabel to x0,ylabel with twoway lt 2}\newline
\noindent {\tt \phantom{xx}arrow from x0+v,ylabel to x0,ylabel with twoway lt 2}\newline
\noindent {\tt \phantom{xx}text "\$u\$" at x0-u/2,ylabel hal c val t gap unit(mm)}\newline
\noindent {\tt \phantom{xx}text "\$v\$" at x0+v/2,ylabel hal c val t gap unit(mm)}\newline
\noindent {\tt \phantom{x}\}}\newline
}\\{\footnotesize
\noindent {\tt \# Display diagram of lens}\newline
\noindent {\tt set multiplot ; set nodisplay}\newline
\noindent {\tt call LensDraw(0*unit(cm),0*unit(cm), 5*unit(cm),1.5*unit(cm),2*unit(cm))}\newline
\noindent {\tt set display ; refresh}\newline
}
\nlfcf
The resulting diagram is shown below:
\nlscf
\centerline{\includegraphics{examples/eps/ex_lenses.eps}}
}

\subsection{The {\tt point} Command}
\label{sec:point}

The \indcmdt{point} places a single point on a multiplot canvases, in the same
style which would be used when plotting a dataset on a graph with the {\tt
points} plotting style. It is useful for marking significant points on
technical diagrams with crosses or other motifs.

The \indcmdt{point} that the position of the point to be marked be specified
after the {\tt at} modifier. A text label to be attached next to the point may
optionally be specified using the same {\tt label} modifier as taken by the
{\tt plot} command. A {\tt with} modifier may then be supplied, followed by any
of the style modifiers: {\tt colour}, {\tt pointlinewidth}, {\tt pointsize},
{\tt pointtype}, {\tt style}.

The following example labels the origin as such:
\begin{verbatim}
set texthalign left
set textvalign centre
point at 0,0 label "The Origin" with ps 2
\end{verbatim}

\subsection{The {\tt ellipse} Command}
\label{sec:ellipse}

Ellipses may be placed on multiplot canvases using \indcmdt{ellipse}. The shape
of the ellipse may be specified in many different ways, by specifying

\begin{enumerate}[(i)]
\item the position of two corners of the smallest rectangle which can enclose
the ellipse when its major axis is horizontal, together with an optional
counter-clockwise rotation angle, applied about the centre of the ellipse.
For example:

\begin{verbatim}
ellipse from 0,0 to 4,1 rot 70
\end{verbatim}

\item the position of both the centre and one of the foci of the ellipse,
together with any one of the following pieces of information: the major axis
length, the semi-major axis length, the minor axis length, the semi-minor axis
length, the eccentricity, the latus rectum, or the semi-latus rectum.  For
example:

\begin{verbatim}
ellipse focus 0,0 centre 2,2 majoraxis 4
ellipse focus 0,0 centre 2,2 minoraxis 4
ellipse focus 0,0 centre 2,2 ecc 0.5
ellipse focus 0,0 centre 2,2 LatusRectum 6
ellipse focus 0,0 centre 2,2 slr 3
\end{verbatim}

\item the position of either the centre or one of the foci of the ellipse,
together with any two of the following pieces of information: the major axis
length, the semi-major axis length, the minor axis length, the semi-minor axis
length, the eccentricity, the latus rectum, or the semi-latus rectum. An
optional counter-clockwise rotation angle may also be specified, applied about
either the centre or one of the foci of the ellipse, whichever is specified. If
no rotation angle is given, then the major axis of the ellipse is horizontal.
For example:

\begin{verbatim}
ellipse centre 0,0 majoraxis 4 minoraxis 4
\end{verbatim}
\end{enumerate}

The line type, line width, and colour of line with which the outlines of
ellipses are drawn may be specified as in the {\tt box} and {\tt circle}
commands above. Likewise, ellipses may be filled in the same manner.

\example{ex:ellipse}{A labelled diagram of an ellipse}{
In this example script, we illustrate the text of Section~\ref{sec:ellipse} by
using PyXPlot's {\tt ellipse} command, together with arrows and text labels, to
produce a labelled diagram of an ellipse. We label the semi-major axis $a$, the
semi-minor axis $b$, the semi-latus rectum $L$, and the distance between the
centre of the ellipse and one of its foci with the length $ae$, where $e$ is
the eccentricity of the ellipse.
\nlscf
{\footnotesize
\noindent {\tt set multiplot ; set nodisplay }\newline
\noindent {\tt }\newline
\noindent {\tt a~~~= 6.0~~~~~~~~~~~~~~\# Semi-major axis }\newline
\noindent {\tt b~~~= 4.0~~~~~~~~~~~~~~\# Semi-minor axis }\newline
\noindent {\tt e~~~= sqrt(1-(b/a)**2)~\# Eccentricity }\newline
\noindent {\tt slr~= a*(1-e**2)~~~~~~~\# Length of semi-latus rectum }\newline
\noindent {\tt fd~~= a*e~~~~~~~~~~~~~~\# Distance of focus from centre }\newline
}\\{\footnotesize
\noindent {\tt \# Draw ellipse }\newline
\noindent {\tt ellipse centre 0,0 SemiMajor a SemiMinor b with lw 3 }\newline
\noindent {\tt }\newline
\noindent {\tt \# Draw points at centre and focus}\newline
\noindent {\tt set texthalign centre ; set textvalign top}\newline
\noindent {\tt set fontsize 1.5}\newline
\noindent {\tt point at 0,0 label "Centre" with pointsize 2 plw 2}\newline
\noindent {\tt point at -fd,0 label "Focus" with pointsize 2 plw 2}\newline
}\\{\footnotesize
\noindent {\tt \# Draw arrows and dotted lines on ellipse }\newline
\noindent {\tt arrow from 0,0 to 0,b with twohead lw 2 lt 3~~~\# Semi-minor axis}\newline
\noindent {\tt arrow from 0,0 to a,0 with twohead lw 2 lt 3~~~\# Semi-major axis}\newline
\noindent {\tt arrow from -fd,0 to -fd,slr with tw lw 2 lt 3~~\# SLR}\newline
\noindent {\tt arrow from 0,0 to -fd,0 with twohead lw 2 lt 3 \# Focus $<$-$>$ Centre}\newline
}\\{\footnotesize
\noindent {\tt \# Label ellipse }\newline
\noindent {\tt set texthalign centre ; set textvalign centre}\newline
\noindent {\tt text '\$ae\$' at -fd/2,-0.3}\newline
\noindent {\tt text '\$a\$' at a/2,+0.3}\newline
\noindent {\tt text '\$b\$' at 0.3,b/2}\newline
\noindent {\tt set texthalign left ; set textvalign centre}\newline
\noindent {\tt text '\$L=a(1-e\^{}2)\$' at  0.2-fd,slr/2}\newline
}\\{\footnotesize
\noindent {\tt \# Display diagram }\newline
\noindent {\tt set display ; refresh }\newline
}
\nlfcf
The resulting diagram is shown below:
\nlscf
\centerline{\includegraphics[width=8cm]{examples/eps/ex_ellipse.eps}}
}

\subsection{The {\tt piechart} Command}
\label{sec:piechart}

The \indcmdt{piechart} produces piecharts based upon single columns of data
read from \datafile s, which are taken to indicate the sizes of the pie wedges.
The \indcmdt{piechart} has the following syntax:
\begin{verbatim}
piechart ('<filename>'|<function>)
     [using <using specifier>]
     [select <select specifier>]
     [index <index specifier>]
     [every <every specifier>]
     [label <auto|key|inside|outside> <label>]
     [format <format string>]
     [with <style> [<style modifier> ... ] ]
\end{verbatim}

Immediately after the {\tt piechart} keyword, the file (or indeed, function)
from which the data is to be taken should be specified; any of the modifiers
taken by the {\tt plot} command -- i.e.\ {\tt using}, {\tt index}, etc.\ -- may
be used to specify which data from this \datafile\ should be used. The {\tt
label} modifier should be used to specify how a name for each pie wedge should
be drawn from the \datafile, and has a similar syntax to the equivalent
modifier in the {\tt plot} command, except that the name string may be
prefixed by a keyword to specify how the pie wedge names should be positioned.
Four options are available:

\noindent
\begin{itemize}
\item {\tt auto} -- specifies that the {\tt inside} positioning mode should be used on wide pie wedges, and the {\tt outside} positioning mode should be used on narrow pie wedges. {\bf [default]}
\item {\tt key} -- specifies that all of the labels should be arranged in a vertical list to the right-hand side of the piechart.
\item {\tt inside} -- specifies that the labels should be placed within the pie wedges themselves.
\item {\tt outside} -- specifies that the labels should be arranged around the circumference of the pie chart.
\end{itemize}

Having specified a name for each wedge using the {\tt label} modifier, the {\tt
format} modifier determines the final text which is printed along side each
wedge.  For example, a wedge with name `Europe' might be labelled as `27\%
Europe', applying the default format string:
\begin{verbatim}
"%.1d\%% %s"%(percentage,label)
\end{verbatim}
Three variables may be used in format strings: {\tt label} contains the name of
the wedge as specified by the {\tt label} modifier, {\tt percentage} contains
the numerical percentage size of the wedge, and {\tt wedgesize} contains the
absolute unnormalised size of the wedge, as read from the input \datafile,
before the sizes were renormalised to sum to 100\%.

The {\tt with} modifier may be followed by the keywords {\tt colour}, {\tt
linewidth}, {\tt style}, which all apply to the lines drawn around the
circumference of the piechart and between its wedges. The fill colour of the
wedges themselves are taken sequentially from the current palette, as set by
the {\tt set palette} command. Note that PyXPlot's default palette is optimised
more for producing plots with datasets in different and distinct colours than
for producing piecharts in asthetically pleasing shades, where a little more
subtly may be desirable. A suitable call to the {\tt set palette} command is
highly recommended before the \indcmdt{piechart} is used.

As with the {\tt plot} command, the position and size of the piechart are
governed by the {\tt set origin} and {\tt set size} commands. The former
determines where the centre of the piechart is positioned; the latter
determines its diameter.

\example{ex:piechart}{A piechart of the composition of the Universe}{
It is the beginning of many a news story about cosmology. Astronomers tell us
that of the material which makes up the Universe, only 4\% is in the form of
the baryonic matter which we know and love. The rest is composed of dark matter
(22\%) and dark energy (74\%). That is all very well, though, of course, rather
few people were actually aware until a few seconds ago that they were made up
of stuff called baryonic matter at all. And then, of course, comes the next
obvious question, ``So what is dark matter and dark energy?''. And there,
sadly, the story ends.  Dark energy is... well... it's the number in the
equations which makes the cosmology work.
\nlnp
But rest assured. While cosmologists work on the problem, PyXPlot can plot
piecharts of the composition of the Universe:
\nlscf
\noindent{\tt set palette grey40, grey60, grey80}\newline
\noindent{\tt set width 6}\newline
\noindent{\tt piechart '--' using \$1 label key "\%s"\%(\$2)}\newline
\noindent{\tt 0.22 Dark$\sim$Matter}\newline
\noindent{\tt 0.04 Baryonic$\sim$Matter}\newline
\noindent{\tt 0.74 Dark$\sim$Energy}\newline
\noindent{\tt END}
\nlscf
\centerline{\includegraphics{examples/eps/ex_piechart.eps}}
\nlfcf
Below, we show the change produced by replacing the line\vspace{2mm}\newline
\noindent{\tt piechart '--' using \$1 label key "\%s"\%(\$2)}\vspace{2mm}\newline
with\vspace{2mm}\newline
\noindent{\tt piechart '--' using \$1 label auto "\%s"\%(\$2)}\vspace{2mm}\newline
Note that the labels on the piechart are placed either within the pie, in the
cases of large wedges, and around the edge of the pie for those wedges which
are too narrow for this.
\nlscf
\centerline{\includegraphics{examples/eps/ex_piechart2.eps}}
}

\section{LaTeX and PyXPlot}

The \indcmdt{text} can straightforwardly be used to render simple one-line
\LaTeX\index{latex} strings, but sometimes the need arises to place more
substantial blocks of text onto a plot. For this purpose, it can be useful to
use the \LaTeX\ {\tt parbox} or {\tt minipage} environments\footnote{Remember,
any valid \LaTeX\ string can be passed to the \indcmdt{text} and \indcmdt{set
label}.}. For example:

\begin{verbatim}
text '\parbox[t]{6cm}{\setlength{\parindent}{1cm} \
\noindent There once was a lady from Hyde, \\ \
Who ate a green apple and died, \\ \
\indent While her lover lamented, \\ \
\indent The apple fermented, \\ \
and made cider inside her inside.}'
\end{verbatim}

\begin{center}
\fbox{\includegraphics{examples/eps/ex_text1.eps}}
\end{center}

If unusual mathematical symbols are required, for example those in the {\tt
amsmath} package\index{amsmath package@{\tt amsmath} package}, such a package
can be loaded using the \indcmdt{set preamble}. For example:

\begin{verbatim}
set preamble \usepackage{marvosym}
text "{\Huge\Dontwash\ \NoIroning\ \NoTumbler}$\;$ Do not \
wash, iron or tumble-dry this plot."
\end{verbatim}

\begin{center}
\fbox{\includegraphics{examples/eps/ex_text2.eps}}
\end{center}

