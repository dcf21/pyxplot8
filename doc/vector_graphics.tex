% VECTOR_GRAPHICS.TEX
%
% The documentation in this file is part of PyXPlot
% <http://www.pyxplot.org.uk>
%
% Copyright (C) 2006-9 Dominic Ford <coders@pyxplot.org.uk>
%               2009   Ross Church
%
% $Id$
%
% PyXPlot is free software; you can redistribute it and/or modify it under the
% terms of the GNU General Public License as published by the Free Software
% Foundation; either version 2 of the License, or (at your option) any later
% version.
%
% You should have received a copy of the GNU General Public License along with
% PyXPlot; if not, write to the Free Software Foundation, Inc., 51 Franklin
% Street, Fifth Floor, Boston, MA  02110-1301, USA

% ----------------------------------------------------------------------------

% LaTeX source for the PyXPlot Users' Guide

\chapter{Producing Vector Graphics in PyXPlot}
\label{ch:vector_graphics}

So far, we have talked exclusively about how to plot graphs with PyXPlot. In
this chapter, we discuss how to label graphs and place simple vector graphics
around them.

\section{Multi-plotting}
\label{sec:multiplot}
\index{multiplot}

Gnuplot has a plotting mode called {\it multiplot} which allows many graphs to
be plotted together and displayed side-by-side. The basic syntax of this mode
is reproduced in PyXPlot, but it is hugely extended.

The mode is entered by the command \indcmdts{set multiplot}.  This can be
compared to taking a blank sheet of paper on which to place plots.  Plots are
then placed on that sheet of paper, as usual, with the {\tt plot} command. The
position of each plot is set using the \indcmdt{set origin}, which takes a
comma-separated $(x,y)$ co-ordinate pair, measured in centimetres. The
following, for example, would plot a graph of $\sin(x)$ to the left of a plot
of $\cos(x)$:

\begin{verbatim} 
set multiplot
plot sin(x)
set origin 10,0
plot cos(x)
\end{verbatim}

The multiplot page may subsequently be cleared with the \indcmdt{clear}, and
multiplot mode may be left using the \indcmdt{set nomultiplot}.

\subsection{Deleting, Moving and Changing Plots}

Each time a plot is placed on the multiplot page in PyXPlot, it is allocated a
reference number, which is output to the terminal. Reference numbers count up
from zero each time the multiplot page is cleared. A number of commands exist
for modifying plots after they have been placed on the page, selecting them by
making reference to their reference numbers.

Plots may be removed from the page with the \indcmdt{delete}, and restored with
the \indcmdt{undelete}:

\begin{verbatim} 
delete <number>
undelete <number>
\end{verbatim}

The reference numbers of deleted plots are not reused until the page is
cleared, as they may always be restored with the \indcmdt{undelete}; plots
which have been deleted simply do not appear.

Plots may also be moved with the \indcmdt{move}. For example, the following
would move plot 23 to position $(8,8)$ measured in centimetres:

\begin{verbatim} 
move 23 to 8,8
\end{verbatim}

In multiplot mode, the \indcmdt{replot} can be used to modify the last plot
added to the page. For example, the following would change the title of the
latest plot to `foo', and add a plot of $\cos(x)$ to it:

\begin{verbatim} 
set title 'foo'
replot cos(x)
\end{verbatim}

Additionally, it is possible to modify any plot on the page, by first selecting
it with the \indcmdt{edit}. Subsequently, the \indcmdt{replot} will act upon
the selected plot. The following example would produce two plots, and then
change the colour of the text on the first:

\begin{verbatim} 
set multiplot
plot sin(x)
set origin 10,0
plot cos(x)
edit 0        # Select the first plot ...
set textcolour red
replot        # ... and replot it.
\end{verbatim}

The \indcmdt{edit} can also be used to view the settings which are applied to
any plot on the multiplot page -- after executing {\tt edit~0}, the
\indcmdt{show} will show the settings applied to plot zero.

When a new plot is added to the page, the \indcmdt{replot} always switches to
act upon this most recent plot.

\subsection{Listing Items on a Multiplot}

A listing of all of the items on a multiplot, giving their reference numbers
and the commands used to produce them, can be obtained using the
\indcmdt{list}. For example:

\begin{verbatim}
pyxplot> list
#  ID | Command 
    0   plot f(x) 
d   1   text 'Figure 1: A plot of f(x)' 
    2   text 'Figure 1: A plot of $f(x)$' 

# Items marked 'd' are deleted 
\end{verbatim}

In this example, the user has plotted a graph of $f(x)$, and added a caption to
it. The {\tt ID} column lists the reference numbers of each multiplot item.
Item {\tt 1} has been deleted, and this is indicated by the {\tt d} to the left
of its reference number.

\subsection{Linked Axes}

The axes of plots can be linked together, in such a way that they always share
a common scale. This can be useful when placing plots next to one another,
firstly, of course, if it is of intrinsic interest to ensure that they are on a
common scale, but also because the two plots then do not both need their own
axis labels, and space can be saved by one sharing the labels from the other.
In PyXPlot, an axis which borrows its scale and labels from another is called a
{\it linked axis}.

Such axes are declared by setting the label of the linked axis to a magic
string such as {\tt linkaxis 0}\label{sec:linked_axes}\index{axes!reserved
labels}\index{magic axis labels}. This magic label would set the axis to borrow
its scale from an axis from plot zero. The general syntax is `{\tt linkaxis}
$n$ $m$', where $n$ and $m$ are two integers, separated by a comma or
whitespace. The first, $n$, indicates the plot from which to borrow an axis;
the second, $m$, indicates whether to borrow the scale of axis $x1$, $x2$,
$x3$, etc. By default, $m=1$. The linking will fail, and a warning result, if
an attempt is made to link to an axis which doesn't exist.

\subsection{Text Labels, Arrows and Images}

In addition to placing plots on the multiplot page, text labels may also be
inserted independently of any plots, using the \indcmdt{text}. This has the
following syntax:

\begin{verbatim} 
text 'This is some text' at x,y
\end{verbatim}

In this case, the string `This is some text' would be rendered at position
$(x,y)$ on the multiplot. As with the \indcmdt{set label}, a colour may
optionally be specified with the {\tt with colour} modifier, as well as a
rotation angle to rotate text labels through any given angle, measured in
degrees counter-clockwise. For example:\indkey{rotate}

\begin{verbatim} 
text 'This is some text' at x,y rotate r with colour red
\end{verbatim}

The commands \indcmdts{set textcolour}, \indcmdts{set texthalign} and
\indcmdts{set textvalign}, which have already been described in the context in
the {\tt set label} command, can also be used to set the colour and alignment
of text produced with the \indcmdt{text}.  A useful application of this is to
produce centred headings at the top of multiplots.

As with plots, each text item has a unique identification number, and can be
moved around, deleted or undeleted with the \indcmdts{move},
\indcmdts{delete} and \indcmdts{undelete} commands.

It should be noted that the \indcmdt{text} can also be used outside of the
multiplot environment, to render a single piece of short text instead of a
graph. One obvious application is to produce equations rendered as graphical
files for inclusion in talks.\index{presentations}

Arrows may also be placed on multiplot pages, independently of any plots, using
the \indcmdt{arrow}, which has syntax:

\begin{verbatim} 
arrow from x,y to x,y
\end{verbatim}

As above, arrows receive unique identification numbers, and can be deleted and
undeleted.

The \indcmdt{arrow} may be followed by the \indmodt{with} keyword to specify to
style of the arrow. The style keywords which are accepted are identical to
those accepted by the {\tt set arrow} command (see
Section~\ref{sec:set_arrow}).  For example:

\begin{verbatim} 
arrow from x1,y1 to x2,y2 \
with twohead colour red
\end{verbatim}

Bitmap images in jpeg format may be placed on the multiplot using the
\indcmdt{jpeg}.  This has syntax:

\begin{verbatim}
jpeg 'filename' at x,y width w
\end{verbatim}

As an alternative to the \indkeyt{width} keyword the height of the image can be
specified, using the analogous \indkeyt{height} keyword.  An optional angle can
also be specified using the \indkeyt{rotate} keyword; this causes the included
image to be rotated counter-clockwise by a specified angle, measured in
degrees.

Vector graphic images in eps format may be placed on to a multiplot using the
\indcmdt{eps}, which has a syntax analogous to the {\tt jpeg} command.  However
neither height nor width need be specified; in this case the image will be
included at its native size.  For example:

\begin{verbatim}
eps 'filename' at 3,2 rotate 5
\end{verbatim}

\noindent will place the eps file with its bottom-left corner at position
$(3,2)$\,cm from the origin, rotated counter-clockwise through 5 degrees.

\subsection{Speed Issues}
\label{sec:set_display}

By default, whenever an item is added to a multiplot, or an existing item moved
or replotted, the whole multiplot is replotted to show the change. This can be
a time consuming process on large and complex multiplots. For this reason, the
\indcmdt{set nodisplay} is provided, which stops PyXPlot from producing any
output. The \indcmdt{set display} can subsequently be issued to return to
normal behaviour.

This can be especially useful in scripts which produce large multiplots. There
is no point in producing output at each step in the construction of a large
multiplot, and a great speed increase can be achieved by wrapping the script
with:

\begin{verbatim} 
set nodisplay
[...prepare large multiplot...]
set display
refresh
\end{verbatim}

\subsection{The \indcmdt{refresh}}

\index{replotting} The \indcmdt{refresh} is rather similar to the
\indcmdt{replot}, but produces an exact copy of the latest display. This can be
useful, for example, after changing the terminal type, to produce a second copy
of a multiplot page in a different format. But the crucial difference between
this command and {\tt replot} is that it doesn't replot anything. Indeed, there
could be only textual items and arrows on the present multiplot page, and no
graphs {\it to} replot.

\section{LaTeX and PyXPlot}

The \indcmdt{text} can straightforwardly be used to render simple one-line
\LaTeX\index{latex} strings, but sometimes the need arises to place more
substantial blocks of text onto a plot. For this purpose, it can be useful to
use the \LaTeX\ {\tt parbox} or {\tt minipage} environments\footnote{Remember,
any valid \LaTeX\ string can be passed to the \indcmdt{text} and \indcmdt{set
label}.} For example:

\begin{verbatim} 
text '\parbox[t]{6cm}{\setlength{\parindent}{1cm} \
\noindent There once was a lady from Hyde, \\ \
Who ate a green apple and died, \\ \
\indent While her lover lamented, \\ \
\indent The apple fermented, \\ \
and made cider inside her inside.}'
\end{verbatim}

If unusual mathematical symbols are required, for example those in the {\tt
amsmath} package\index{amsmath package@{\tt amsmath} package}, such a package
can be loaded using the \indcmdt{set preamble}. For example:

\begin{verbatim} 
set preamble \usepackage{marvosym}
text "{\Huge\Dontwash\ \NoIroning\ \NoTumbler}$\;$ Do not \
wash, iron or tumble-dry this plot."
\end{verbatim}

