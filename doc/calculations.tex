% CALCULATIONS.TEX
%
% The documentation in this file is part of PyXPlot
% <http://www.pyxplot.org.uk>
%
% Copyright (C) 2006-9 Dominic Ford <coders@pyxplot.org.uk>
%               2009   Ross Church
%
% $Id$
%
% PyXPlot is free software; you can redistribute it and/or modify it under the
% terms of the GNU General Public License as published by the Free Software
% Foundation; either version 2 of the License, or (at your option) any later
% version.
%
% You should have received a copy of the GNU General Public License along with
% PyXPlot; if not, write to the Free Software Foundation, Inc., 51 Franklin
% Street, Fifth Floor, Boston, MA  02110-1301, USA

% ----------------------------------------------------------------------------

% LaTeX source for the PyXPlot Users' Guide

\chapter{Performing Calculations}

This chapter describes how PyXPlot may be used as a sophisticated calculator.

\section{Variables}

In PyXPlot variables that hold numeric values can be assigned using any valid
mathematical expression.\footnote{PyXPlot also recognises string variables,
which are discussed in Section~\ref{sect:stringvars}.}  For example:

\begin{verbatim}
a = 5.2 * sqrt(64)
\end{verbatim}

\noindent would assign the value 41.6 to the variable {\tt a}.  Numerical variables can
subsequently be used in mathematical expressions themselves, for example:

\begin{verbatim}
a=2*pi
plot [0:1] sin(a*x)
\end{verbatim}

\section{Handling Numerical Errors}
\label{sec:num_errs}
\index{numerical errors}

By default, an error message is returned whenever calculations return values
which are infinite, as in the case of {\tt 1/0}, or when functions are
evaluated outside the range of parameter space in which they are defined, as in
the case of {\tt besseli(-1,1)}.  Sometimes this behaviour is desirable: it
flags up to the user that a calculation has gone wrong, and exactly what the
problem is.  At other times, however, these error messages can be undesirable
and may lead you to miss more genuine and serious errors buried in their midst.

For this reason, the issuing of explicit error messages when calculations
return non-finite numeric results can be switched off by typing
\indcmd{set numeric errors quiet}

\begin{verbatim}
set numeric errors quiet
\end{verbatim}

\noindent Having done this, expressions such as

\begin{verbatim}
x = besseli(-1,1)
\end{verbatim}

\noindent fail silently, and variables which contain non-finite numeric results
are displayed as {\tt NaN}\index{NaN}, which stands for {\it Not a
Number}\index{not a number}.

The issuing of explicit errors may subsequently be re-enabled by typing
\indcmd{set numeric errors explicit}

\begin{verbatim}
set numeric errors explicit
\end{verbatim}

\section{Working with Complex Numbers}
\label{sec:complex_numbers}
\index{complex numbers}

In all of the examples given thus far, algebraic expressions have only been
allowed to return real numbers: PyXPlot has not been handling any complex
numbers. Since there are many circumstances in which you may be analysing data
which you are certain is real, complex arithmetic is disabled by default.
Expressions such as {\tt sqrt(-1)} will return either an error or {\tt NaN}.
The most obvious example of this is the built-in variable {\tt i}, which is set
to equal {\tt sqrt(-1)}:

\begin{verbatim}
pyxplot> print i
nan
\end{verbatim}

Complex arithmetic may be enabled by typing
\indcmd{set numeric complex}

\begin{verbatim}
set numeric complex
\end{verbatim}

\noindent and then disabled again by typing
\indcmd{set numeric real}

\begin{verbatim}
set numeric real
\end{verbatim}

Once complex arithmetic is enabled, many of PyXPlot's built in mathematical
functions accept complex input arguments, including the logarithm function, all
of the trigonometric functions, and the exponential function.  A complete list
of functions which accept complex inputs can be found in
Appendix~\ref{ch:function_list}.

Complex number literals can be entered into algebraic expressions in either of
the following two forms:

\begin{verbatim}
print (2 + 3*i       )
print (2 + 3*sqrt(-1))
\end{verbatim}

\noindent The former version depends upon the built-in system variable {\tt i}
being defined to equal $\sqrt{-1}$; the user could stop this from working if he
had previously typed, for example

\begin{verbatim}
i=1
\end{verbatim}

\noindent However, the variable {\tt i} can straightforwardly be returned to
its default value by typing

\begin{verbatim}
i=sqrt(-1)
\end{verbatim}

Several functions are provided especially for performing manipulations on
complex numbers. the \indfunt{Re(z)} and \indfunt{Im(z)} functions return
respectively the real and imaginary parts of a complex number $z$, the
\indfunt{arg(z)} function returns the complex argument of $z$, and the
\indfunt{abs(z)} function returns the modulus of $z$.  The
\indfunt{conjugate(z)} command returns the complex conjugate of $z$.  For
example:

\begin{verbatim}
set numeric complex
x=0.5
print Re(exp(i*x))
print cos(x)        # This equals the above
print arg(exp(i*x)) # This equals x
\end{verbatim}

\section{Working with Physical Units}
\label{sec:units}
\index{physical units}\index{units}

PyXPlot has extensive facilities for handling data with a range of physical
units.  These features also make it a powerful desktop tool for converting
quantities between different imperial and metric units, and for doing simple
back-of-the-envelope calculations where numbers are substituted into equations
and you want to know what the dimensions of the quantity you've calculated are.

All numeric variables in PyXPlot have not only a magnitude, but also a
physical unit associated with them. In the case a pure number such as~2, the
quantity is said to be dimensionless: it has no physical unit. The special
function \indfunt{unit()} is used to specify the physical unit associated with a
quantity. For example, the expression

\begin{verbatim}
print 2*unit(s)
\end{verbatim}

\noindent takes the number~2 and multiplies it by the unit {\tt s}, which is
the SI abbreviation for seconds.  The resulting quantity then has dimensions of
time, and could, for example, be divided by the unit {\tt hr} to find the
dimensionless number of hours in two seconds:

\begin{verbatim}
print 2*unit(s)/unit(hr)
\end{verbatim}

\begin{boxout}\index{units!angle}\index{angles, handling of}
By convention, the SI system of units does not have a base unit of angle:
instead, the radian is considered to be a dimensionless unit.  There are some
strong mathematical reasons why this makes sense, since it makes it possible to
write equations such as
\begin{displaymath}
d=\theta r
\end{displaymath}
and
\begin{displaymath}
x = \exp(a+i\theta).
\end{displaymath}
However, it also has some disadvantages: some interesting quantities are
measured per unit angle or per unit solid angle, and the SI system offers no
way to dimensionally distinguish these from one another or from quantities with
no angular dependence.

On balance, we have decided that it is more useful if PyXPlot {\it does}
consider the radian to be a base unit, so that it can understand quantities
such as radiative fluxes measured per steradian. The unfortunate consequence of
this is that first equation above to be rewritten as:
\begin{displaymath}
d=\left(\frac{\theta}{2\pi\,\mathrm{rad}}\right) r
\end{displaymath}

As compensation, the $\exp()$ function and all of the trigonometric functions
accept either quantities with dimensions of angles, or dimensionless numbers as
inputs. The second equation above might need some modification, however, since
it is not possible to add a dimensionless number to an angle.
\caption{A note on the use of the radian is a base unit in PyXPlot.}
\end{boxout}

Compound units such as miles per hour, which is defined in terms of two other
units, can be used thus

\begin{verbatim}
print 2*unit(miles/hour)
\end{verbatim}

\noindent or, in many cases, have their own explicit abbreviations, in this
case {\tt mph}:

\begin{verbatim}
print 2*unit(mph)
\end{verbatim}

\noindent As these examples demonstrate, the {\tt unit()} function can be
passed a string of units either multiplied together with the {\tt *} operator,
or divided using the {\tt /} operator. Units may be raised to powers with the
{\tt **} operator\footnote{The {\tt \^{}} character may be used as an alias for
the {\tt **} operator, though this notation is arguably confusing, since the
same character is used for the binary exclusive or operator in PyXPlot's normal
arithmetic.}, as in

\begin{verbatim}
a = 2*unit(m**2)
print "An area of %f square feet"%(a/unit(ft**2))
\end{verbatim}

\noindent As these examples also demonstrate, units may be refered to either by
their abbreviated or full names, and either in their singular or plural forms.
A complete list of all of the units which PyXPlot recognises by default,
together with all of their recognised names can be found in
Appendix~\ref{ch:unit_list}.

SI units may also be preceded with SI prefixes\index{units!SI prefixes}, for
example:

\begin{verbatim}
a = 2*unit(um)
a = 2*unit(micrometres)
\end{verbatim}

When quantities with physical units are substituted into algebraic expressions,
PyXPlot automatically checks that the expression is dimensionally correct
before evaluating it. For example, the following expression is not
dimensionally correct and would return an error because the first term in the
sum has dimensions of velocity, meanwhile the second term is a length:
\index{units!dimensional analysis}

\begin{verbatim}
a = 2*unit(m)
b = 4*unit(s)
print a/b + a
\end{verbatim}

\noindent PyXPlot continues to throw an error in this case, even when explicit
numerical errors are turned off with the \indcmdt{set numeric errors quiet},
since it is deemed a serious error: the above expression would never be correct
for any values of {\tt a} and {\tt b} given their dimensions.

As a quick perusal of Appendix~\ref{ch:unit_list} will show, a large number of
units are pre-defined in PyXPlot by default. However, the need may occasionally
arise to define new units. It is not possible to do this from an interactive
PyXPlot terminal, but it is possible to do so from a configuration script which
PyXPlot runs upon start-up. This will be discussed in
Chapter~\ref{ch:configuration}, where we shall also see that new base units can
also be defined. Just as the Pluto mass can be defined as a new measure of
mass, so the potato can be defined as a new measure of number of vegetables.

\subsection{Converting between different Temperature Scales}
\index{temperature conversions}\index{units!temperature}

PyXPlot's facilities for converting quantities between different physical units
include the ability to convert temperatures between different temperature
scales, for example, between $^\circ\mathrm{C}$, $^\circ\mathrm{F}$ and K.
However, these conversions have some subtleties, unique to temperature
conversions, which mean that they should be used with some caution. Consider
the following two questions:
\begin{itemize}
\item How many degrees Kelvin corresponds to a temperature of $20^\circ$C?
\item How many degrees Kelvin corresponds to a temperature {\it rise} of $20^\circ$C?
\end{itemize}
The answers to these two questions are 293\,K and 20\,K respectively: we see
that although we are converting from $20^\circ$C in both cases, the
corresponding number of Kelvin depends upon whether we are talking about an
{\it absolute} temperature or a {\it relative} temperature. A heat capacity of
1\,J/$^\circ$C equals 1\,J/K, even though $1^\circ$C does not equal 1\,K.

The cause of this problem, and the reason why it rarely affects any physical
units other than temperatures is that there exists such a thing as absolute
temperature. Distances, for example, are very rarely absolute: they measure
relative distance gaps between points. Occasionally people might choose to
express all their displacements relative to a particular origin, but they
wouldn't expect PyXPlot to be able to convert these into displacements from
another origin. But they might expect it to be able to convert temperatures
between Celsius and Fahrenheit, even though the problem of doing so is
equivalent.

Times are occasionally expressed as absolute quantities: the year 1453, for
example, implicitly corresponds to 1453 years since the Christian epoch.
Similar problems would arise in trying to convert such a year into the Muslim
calendar, which counts from the year {\footnotesize AD} 622, to those of
encountered in converting between temperature scales.\footnote{PyXPlot can,
incidentally, make this conversion, as will be seen in
Section~\ref{sec:time_series}.}

As PyXPlot cannot distinguish between absolute and relative temperatures, it
takes a safe approach of performing algebra consistently with any unit of
temperature, never performing automatic conversions between different
temperature scales. A calculation based on temperatures measured in
$^\circ\mathrm{F}$ will produce an answer measured in $^\circ\mathrm{F}$.
However, as converting temperatures between temperature scales is a useful task
which is often wanted, this is allowed, when specifically requested, in the
specific case of dividing one temperature by another unit of temperature to get
a dimensionless number, as in the following example:

\begin{dodo}
print 98*unit(oF) / unit(oC)
\end{dodo}

\noindent Note that the two units of temperature must be placed in separate
{\tt unit(...)} functions. The following is not allowed:

\begin{dontdo}
print 98*unit(oF / oC)
\end{dontdo}

Note that such a conversion always assumes that the temperatures supplied are
{\it absolute} temperatures. PyXPlot has no facility for converting relative
temperatures between different scales. This must be done manually.

The conversion of derived units of temperature, such as $\mathrm{J}/\mathrm{K}$ or
$^\circ\mathrm{C}^2$, to derived units of other temperature scales, such as
$\mathrm{J}/^\circ\mathrm{F}$ or $\mathrm{K}^2$, is not permitted, since in
general these conversions are ill-defined. For example, a temperature squared
measured in $^\circ\mathrm{C}^2$ has the same value for $\pm
x^\circ\mathrm{C}$, but would have different values in $\mathrm{K}^2$.

The moral of this story is: pick what unit of temperature you want to work in,
convert all of your temperatures to that scale, and then stick to it.

\section{Configuring how Numbers are Displayed}
\label{sec:unitdisp}

\subsection{Units}

By default, when a number which has physical dimensions is displayed PyXPlot
searches through its database of physical units for the most appropriate unit
in which to represent it. The name of the adopted unit is printed after the
value. By default, quantities are displayed by preference in SI units, and SI
prefixes such as milli- or kilo- are applied where appropriate. All of the
behaviour, however, can be configured.

PyXPlot has a number of different {\it units schemes}\index{units!unit
schemes}, each of which comprises a list of units which are to be used in
preference to all others. For example, in the CGS unit scheme\index{CGS
units}\index{units!CGS}, all lengths are displayed in centimetres, all masses
are displayed in grammes, and so forth. In the imperial unit
scheme\index{imperial units}\index{units!imperial}, quantities are displayed in
British imperial units, meanwhile in the US unit scheme, US customary units are
used. The current unit scheme can be changed using the \indcmdt{set unit
scheme}:

\begin{verbatim}
pyxplot> vol = 3*unit(m**3)
pyxplot> set unit scheme si ; print vol
3 cubic_m
pyxplot> set unit scheme cgs ; print vol
3000000 cubic_cm
pyxplot> set unit scheme imperial ; print vol
82.488468 bushels_(UK)
pyxplot> set unit scheme us ; print vol
85.13278 bushels_(US)
\end{verbatim}

\noindent A complete list of PyXPlot's unit schemes can be found in
Table~\ref{tab:unit_schemes}.\index{natural units}\index{units!natural}

\begin{table}
\framebox[\textwidth]{
\begin{tabular}{lp{9cm}}
{\bf Name} & {\bf Description} \\
\hline
{\tt ancient} & Ancient units, especially those used in the Authorised Version of the Bible. \\
{\tt CGS} & CGS units. \\
{\tt Imperial} & British imperial units. \\
{\tt Planck} & Planck units, also known as natural units, which make several physical constants equal unity. \\
{\tt SI} & SI units. \\
{\tt US} & US customary units. \\
\hline
\end{tabular}}
\caption{A list of PyXPlot's unit schemes.}
\label{tab:unit_schemes}
\end{table}

In some cases, one may want to broadly use one of these unit schemes, but
override one of the units in favour of another.  Astronomers, for example, may
wish to use SI or CGS units to express all quantities with the exception of
distances, which they wish to express in parsecs or astronomical units. Another
astronomer might wish to express masses in solar or Jupiter masses, or
luminosities in solar luminosities. This level of control is made available
through the \indcmdt{set unit of} command, for example:

\begin{verbatim}
set unit of length parsec
set unit of mass Mjupiter
\end{verbatim}

The more unusual case of the astronomer who wishes to express masses in Pluto
masses is more complicated: the Pluto mass is not a pre-defined unit in
PyXPlot, and it must first be defined before it can be set as a default unit.
In Chapter~\ref{ch:configuration}, we shall see how to define new units in a
configuration script.

By default, units are displayed in their abbreviated forms, for example {\tt A} for amperes, and SI prefixes such as milli- and kilo- are applied to SI units where they are appropriate.\index{SI prefixes}\index{units!SI prefixes} In both cases, this behaviour
can be turned on or off, in the former case with the commands:

\begin{verbatim}
set unit display abbreviated
set unit display full
\end{verbatim}

\noindent and in the latter case using the following pair of commands:

\begin{verbatim}
set unit display prefix
set unit display noprefix
\end{verbatim}

\subsection{Changing the Accuracy to which Numbers are Displayed}

By default, when a number is displayed, it is printed accurate to eight
significant figures, although fewer figures may actually be displayed if the
final digits are zeros or nines.

This is generally a helpful convention: PyXPlot's internal arithmetic is
generally accurate to around 16 significant figures, and so it is quite
conceivable that a calculation which is supposed to return, say $1$, may in
fact return 0.999\,999\,999\,999\,999\,9. Likewise, when complex arithmetic is
enabled, routines which are expected to return real numbers may in fact return
results with imaginary parts at the level of one part in $10^{16}$.  By
displaying numbers to only eight significant figures in such cases, the user is
usually shown the `right' answer, instead of a noisy and unattractive one.

However, there may also be cases where more accuracy is desirable, in which
case, the number of significant figures to which output is displayed can be set
using the command\indcmd{set numerics sigfig}

\begin{verbatim}
n = 12
set numerics sigfig n
\end{verbatim}

\noindent where {\tt n} can be any number in the range 1-30. It should be noted
that the number supplied is the {\it minimum} number of significant figures to
which numbers are displayed; on occasion an extra figure may be displayed.

Alternatively, the string substitution operator, described in
Section~\ref{sec:string_subs_op} may be used to specify how a number should be
displayed on a one-by-one basis, for example:

\begin{verbatim}
print "%d"  %(pi) # Print the integer part of pi
print "%.5f"%(pi) # Print pi in non-scientific format, to
                  #   5 decimal places
print "%.5e"%(pi) # Print pi in scientific format, to
                  #   5 decimal places
print "%s"  %(pi) # Print pi as normal
\end{verbatim}

\subsection{Creating Pastable Text}
\label{sec:pastable}

PyXPlot's default convention of displaying numbers in a format such as

\begin{verbatim}
(2+3i) metres
\end{verbatim}

\noindent is well-suited for creating text which is readable by human users, but
is less well-suited for creating text which can be copied and pasted into
another calculation in another PyXPlot terminal, or for creating text which
could be used in a \LaTeX\ text label on a plot. For this reason, the
\indcmdt{set numerics display} command allows the user to choose between three
different ways in which numbers can be displayed:

\begin{verbatim}
pyxplot> set numerics display natural
pyxplot> print phy_c
299792 km/s
pyxplot> set numerics display typeable
pyxplot> print phy_c
299792*unit(km/s)
pyxplot> set numerics display latex
pyxplot> print phy_c
$299792\,\mathrm{km}/\mathrm{s}$
\end{verbatim}

The first case is the default way in which PyXPlot displays numbers. The second
case produces text which forms a valid algebraic expression which could be
pasted into another PyXPlot calculation. The final case produces a string of
\LaTeX\ text which could be used as a label on a plot.

\section{Physical Constants}
\label{sec:constants}
\index{physical constants}\index{constants}

A wide range of mathematical and physical constants are defined by default in
PyXPlot: most of the physical constants are prefixed with {\tt phy\_} to
minimise name clashes with variables which the user may wish to define,
although it is not an error for the user to redefine any of the built-in
variables.

These physical constants make it easy to evaluate physical formulae without
explicitly looking up their values, as well as having a wide range of other
uses. For example, the default aspect ratio of plots is the golden ratio,
$\left((1+\sqrt{5})/2\right)^{-1}$, whose value is also contained in the
constant {\tt GoldenRatio}.\index{golden ratio} Thus, the statement

\begin{verbatim}
set size ratio 1/GoldenRatio
\end{verbatim}

\noindent is entirely equivalent to setting plots to have the automatic aspect
ratio. To obtain a square grid on a plot of the default aspect ratio, it is
necessary that the range of the vertical axis should be {\tt 1/GoldenRatio}
times that of the horizontal axis:

\begin{verbatim}
x0 = 0 ; y0 = 0; xspan = 2
set xrange[x0 - xspan/2            :x0 + xspan/2             ]
set yrange[y0 - xspan/2/GoldenRatio:y0 + xspan/2/GoldenRatio ]
\end{verbatim}

\section{User-defined Functions}
\index{function splicing}
\index{splicing functions}

In PyXPlot, as in \gnuplot, user-defined functions may be declared on the
command line:

\begin{verbatim}
f(x) = x*sin(x)
\end{verbatim}

\noindent It is also possible to declare functions which are valid only over
certain ranges of argument space. For example, the following function would
only be valid within the range $-2<x<2$:\footnote{The syntax {\tt [-2:2]} can
also be written {\tt [-2 to 2]}.}

\begin{verbatim}
f(x)[-2:2] = x*sin(x)
\end{verbatim}

\noindent The following function would only be valid when all of ${a,b,c}$ were
in the range $-1 \to 1$:

\begin{verbatim}
f(a,b,c)[-1:1][-1:1][-1:1] = a+b+c
\end{verbatim}

If an attempt is made to evaluate a function outside of its specified range,
then an error results. This may be useful, for example, for plotting a function
only within some specified range. The following would plot the function
$\sinc(x)$, but only in the range $-2<x<7$:

\begin{verbatim}
f(x)[-2:7] = sin(x)/x
plot f(x)
\end{verbatim}

\example{ex:funcsplice1}{A simple example of the use of function splicing to truncate a function}{
A simple example of the use of function splicing to truncate the function $\sinc(x)$ at $x=-2$ and $x=7$. See details in the text.\\
\begin{center}
\includegraphics{examples/eps/ex_funcsplice1.eps}
\end{center}
}

\noindent The output of this particular example can be seen in
Example~\ref{ex:funcsplice1}. A similar effect could also have been achieved
with the {\tt select} keyword; see Section~\ref{sec:select_modifier}.

It is possible to make multiple declarations of the same function, over
different regions of argument space; if there is an overlap in the valid
argument space for multiple definitions, then later declarations take
precedence. This makes it possible to use different functional forms for
functions in different parts of parameter space, and is especially useful when
fitting functions to data, if different functional forms are to be spliced
together to fit different regimes in the data.

Another application of function splicing is to work with functions which do not
have analytic forms, or which are, by definition, discontinuous, such as
top-hat functions or Heaviside functions. The following example would define
$f(x)$ to be a Heaviside function:

\begin{verbatim}
f(x) = 0
f(x)[0:] = 1
\end{verbatim}

\noindent The following example would define $fib(x)$ to follow the Fibonacci
sequence, though it is not at all computationally efficient, and it is
inadvisable to evaluate it for $x\gtrsim25$:

\begin{verbatim}
fib(x) = 1
fib(x)[2:] = fib(x-1) + fib(x-2)
plot [0:8] fib(x)
\end{verbatim}

\example{ex:funcsplice2}{An example of the use of function splicing to define a function which does not have an analytic form}{
An example of the use of function splicing to define a function which does not have an analytic form -- in this case, the Fibonacci sequence. See the text for details.\\
\begin{center}
\includegraphics{examples/eps/ex_funcsplice2.eps}
\end{center}
}

\noindent The output of this example can be seen in Example~\ref{ex:funcsplice2}

\section{Numerical Integration and Differentiation}

\index{differentiation}\index{integration} Two special functions,
\indfunt{int\_dx()} and \indfunt{diff\_dx()}, may be used to numerically
integrate or differentiate an algebraic expressions.  In each case, the letter
{\tt x} is the dummy variable which is to be used in the integration or
differentiation, and may be replaced by any valid variable name of up to
16~characters in length.

The function {\tt int\_dx()} takes three parameters -- firstly the expression
to be integrated, which may optionally be placed in quotes, followed by the
minimum and maximum integration limits. These may have any physical dimensions,
so long as they match, but must both be real numbers. For example, the
following would plot the integral of the function $\sin(x)$:

\begin{verbatim}
plot int_dt('sin(t)',0,x)
\end{verbatim} 

The function {\tt diff\_dx()} takes two obligatory parameters plus one further
optional parameters. The first is the expression to be differentiated, which,
as above, may optionally placed in quotes for clarity. This should be followed
by the numerical value of the dummy variable at the point where the given
expression is to be differentiated. This value may have any physical
dimensions, and may be a complex number. When complex arithmetic is enabled,
PyXPlot checks that the function being differentiated satisfied the
Cauchy-Riemann equations, and returns an error if it does not to indicate that it
is not differentiable.  The final, optional, parameter to the {\tt diff\_dx()}
function is an approximate step size which indicates the range of argument
values over which PyXPlot should take samples to determine the gradient. If no
value is supplied, a value of $10^{-6}$ is used.  The following example would
evaluate the differential of the function $\cos(x)$ with respect to $x$ at
$x=1.0$:

\begin{dodo}
print diff\_dx('cos(x)', 1.0)
\end{dodo}

The following is an example of a function which is not differentiable, and
which throws an error because the Cauchy-Riemann equations are not satisfied:

\begin{dontdo}
pyxplot> set num comp
pyxplot> print diff\_dx(Re(sin(x)),1)
\end{dontdo}

Advanced users may be interested to know that \indfunt{int\_dx()} function is
implemented using the {\tt gsl\_integration\_qags()} function of the Gnu
Scientific Library\index{GSL}, and the \indfunt{diff\_dx()} function is
implemented using the {\tt gsl\_deriv\_central()} function of the same library.
Any caveats which apply to the use of these routines also apply to PyXPlot's
numerical calculus routines.

\section{Equation Solving and Searching for Minima and Maxima}

The \indcmdt{solve} can be used to numerically solve simple systems of
equations, as in the following example, which solves a simple pair of
simultaneous equations of two variables:

\begin{verbatim}
solve x+y=10, x-y=3 via x,y
\end{verbatim}

\noindent As in the {\tt fit} command, the list of variables which should be
varied to achieve a numerical fit should be supplied as a comma-separated list
after the word {\tt via}. No output is returned if the numerical solver
succeeds, otherwise an error message is displayed. If any of the fitting
variables are already defined prior to the \indcmdt{solve} being called, their
values are used as initial guesses, otherwise an initial guess of one for each
fitting variable is assumed.

The \indcmdt{solve} works by minimised the square residuals of all of the
equations supplied, and so even when no exact solution can be found, the best
compromise is returned. The following example has no solution -- a system of
three equations with two variables is over-constrained -- but PyXPlot
nonetheless finds a compromise solution:

\begin{verbatim}
solve x+y=10, x-y=3, 2*x+y=16 via x,y
\end{verbatim}

When complex arithmetic is enabled, the \indcmdt{solve} allows each fitting
variable to take any value in the complex plane, and thus the number of
dimensions of the fitting problem is effectively doubled, as in the following
example:

\begin{verbatim}
set numerics complex
solve abs(x)=2, arg(x)=90*unit(deg) via x
\end{verbatim}

The related \indcmd{minimise}\indcmd{maximise} {\tt minimise} and {\tt
maximise} commands can be used to find the minima or maxima of algebraic
expressions, for example:

\begin{verbatim}
x=0.1
minimise cos(x) via x
\end{verbatim}

\noindent Note that this example doesn't work when complex arithmetic is
enabled, since $\cos(x)$ diverges to $-\infty$ at $x=\pi+\infty i$.

Various caveats apply to all of the commands discussed in this section, which
are common to almost all numerical equation solvers. The {\tt solve}, {\tt
minimise} and {\tt maximise} commands can only claim to return the {\it
locally} optimum solution which is closest to the supplied initial guess, and
this may or may not be the {\it globally} optimum solution.

Furthermore, these commands usually work best when the optimum value they are
searching for is of order one, and if the convergence radius of this solution
is of at least similar magnitude. PyXPlot does attempt to correct cases where
the supplied initial guess turns out to be many orders of magnitude different
from the true solution, but cannot be gauranteed in these cases not to wildly
overshoot and produce unexpected results.

Wherever possible, it is strongly advised that the results returned by these
numerical solvers should be sanity checked.

\section{Working with Time-Series Data}
\label{sec:time_series}

\begin{verbatim}
set calendar british
x = time_juliandate(2000,1,1,0,0,0)
print time_diff(x, time_now())
print time_diff_string(x, time_now())

x = time_now()
print time_string(x)

set calendar julian

print time_string(x)
print time_string(time_now())

set calendar input russian output british
x = time_juliandate(1828,8,28,0,0,0)
print time_string(x)

set calendar islamic
print time_string(time_now())

set calendar hebrew
print time_string(time_now())
\end{verbatim}
