% INTRODUCTION.TEX
%
% The documentation in this file is part of PyXPlot
% <http://www.pyxplot.org.uk>
%
% Copyright (C) 2006-2010 Dominic Ford <coders@pyxplot.org.uk>
%               2008-2010 Ross Church
%
% $Id$
%
% PyXPlot is free software; you can redistribute it and/or modify it under the
% terms of the GNU General Public License as published by the Free Software
% Foundation; either version 2 of the License, or (at your option) any later
% version.
%
% You should have received a copy of the GNU General Public License along with
% PyXPlot; if not, write to the Free Software Foundation, Inc., 51 Franklin
% Street, Fifth Floor, Boston, MA  02110-1301, USA

% ----------------------------------------------------------------------------

% LaTeX source for the PyXPlot Users' Guide

\chapter{Introduction}

\label{ch:introduction}

{\sc PyXPlot} is a multi-purpose command-line tool for performing simple data
processing and for producing graphs and vector graphics. The central philosophy
of PyXPlot's interface is that common tasks -- for example, plotting labelled
graphs of data -- should be accessible via short, simple and intuitive commands
which require minimal typing to produce a first draft result.  At the same
time, these commands also take a sufficient range of optional arguments and
settings to allow these figures to be subsequently fine-tuned into a wide range
of different styles, appropriate for inclusion in reports, talks or academic
journals.

As well as being a graph-plotting package, PyXPlot also has facilities for
fitting mathematical functions to data, for numerically solving simple systems
of equations, and for converting \datafile s between different formats.  Its
mathematical environment can interpolate datasets, integrate and differentiate
them, and take Fourier transforms.  PyXPlot's ability to keep track of the
physical units in which data are expressed, and to convert data between
different units of measurement, mean that it can be used as a powerful desktop
calculator.

\section{PyXPlot's Heritage: \gnuplot}

PyXPlot's interface bears some striking similarities with that of \gnuplot.
Specifically, the commands used for plotting {\it simple} graphs in the two
programs are virtually identical, though the syntax used for more advanced
plotting often differs and PyXPlot's mathematical environment is hugely
extended over that of \gnuplot. This means that \gnuplot\ users will have a
head start with PyXPlot: simple \gnuplot\ scripts will often run in PyXPlot
with minimal modification.

\section{The Structure of this Manual}

This manual aims to serve both as a tutorial guide to PyXPlot, and also as a
reference manual. Part~I provides a step-by-step tutorial and overview of
PyXPlot's features, including numerous worked examples. Part~II provides a more
detailed survey of PyXPlot's plotting and vector graphics commands. Part~III
provides alphabetical reference guides to all of PyXPlot's commands,
mathematical functions and plotting options.  Finally, the appendices provide
information which is likely to be of more specialist interest.

Broadly speaking, Chapter~\ref{ch:first_steps} covers those commands which are
common between PyXPlot and \gnuplot, and those experiences in working with
\gnuplot\ may find that they can skim rather briefly over this chapter. An
approximate list of those features of \gnuplot\ which are either not supported,
or which are substantially different, in PyXPlot can be found in
Appendix~\ref{ch:gnuplot_diffs}.

\section{A Whirlwind Tour}

Before beginning a more systematic tutorial in Chapter~\ref{ch:first_steps}, we
provide a brief tour of a subset of PyXPlot's features, with references to
those chapters of this manual where further details can be found. This section
should provide some flavour of the wide range of tasks for which PyXPlot can be
used. This is not the place for long-winded explanations of the syntax of each
of the quoted PyXPlot commands, but most of the examples will work if pasted
directly into a PyXPlot command prompt.

We will assume that the user has already successfully installed PyXPlot, and
has just opened a new PyXPlot command prompt. For instructions on how to
install PyXPlot, see Chapter~\ref{ch:installation}.

The first command which any such tour must visit -- the workhorse command of
PyXPlot -- is the \indcmdt{plot}. This can be used to plot graphs of either
mathematical functions, by typing, for example

\begin{verbatim}
plot log(x)
\end{verbatim}

\noindent or \datafile s, by typing\footnote{This example requires you to have a
plain text data file called {\tt datafile.dat} in your current working directory,
and is the only example in this section which may not work out of the box.}

\begin{verbatim}
plot 'datafile.dat'
\end{verbatim}

\noindent There are many commands which allow you to configure the appearance
of the plots produced by the \indcmdt{plot}, or to select which data from a
\datafile\ are plotted; these will be discussed at length in
Chapters~\ref{ch:first_steps} and~\ref{ch:plotting}.

PyXPlot has extensive facilities for converting \datafile s between different
physical units -- for example, you can tell it that a column of a \datafile\
represents lengths measured in inches, and request it to plot those lengths on
a graph in millimetres. These facilities can also be applied to numerical
quantities entered by the user.  For example, you can define a variable which
has physical dimensions of length, and then display its value in different
units as follows:

\begin{verbatim}
x = 2 * unit(m)
print x / unit(inch)
\end{verbatim}

When arithmetic operations are applied to numerical quantities which have
physical units, the units propagate intuitively: in the above example, {\tt
x*x} would compute to four square metres. However, the expression {\tt x*x+x}
would throw an error message because it is not dimensionally correct: the first
term has dimensions of area whilst the second term has dimensions of length,
and these cannot be added.  More details of the use of physical units in
PyXPlot are given in Section~\ref{sec:units}, and Appendix~\ref{ch:unit_list}
lists all of the physical units which PyXPlot recognises by default.

Users can add their own units to those recognised by PyXPlot by means of a
configuration file, and these can be declared either as alternative measures of
existing quantities such as length or mass, or as measures of new base
quantities such as man-hours of labour or numbers of CPU cycles. More details
about how to do this are given in Chapter~\ref{ch:configuration}.

The way in which physical units are displayed can be extensively configured --
for example, the automatic use of SI prefixes such as milli- and kilo- is
optional, and the user can request that quantities be displayed in CGS or
imperial units by default. Other settings instruct PyXPlot to display numerical
results in a way which can be pasted into future PyXPlot sessions -- {\tt
2*unit(m)} instead of $2\,\mathrm{m}$ -- or in LaTeX source code, as {\tt
\$$2\backslash$,$\backslash$mathrm\{m\}\$}.

PyXPlot can perform algebra on complex as well as real numbers. By default,
evaluation of {\tt sqrt(-1)} throws an error, as the emergence of complex
numbers is often an indication that a calculation has gone wrong.  Complex
arithmetic can be enabled by typing\indcmd{set numerics complex}

\begin{verbatim}
set numerics complex
print sqrt(-1)
\end{verbatim}

\noindent Many of the mathematical functions which are built into PyXPlot, a
complete list of which can be found in Appendix~\ref{ch:function_list} or by
typing {\tt show functions}, can take complex arguments, for example

\begin{verbatim}
print exp(2+3*i)
print sin(i)
\end{verbatim}

\noindent For more details, see Section~\ref{sec:complex_numbers}.

In the above example, the variable {\tt i} is a pre-defined constant in
PyXPlot, in this case set to equal $\sqrt{-1}$. PyXPlot has many other
pre-defined physical and mathematical constants, and complete list of which can
found in Appendix~\ref{ch:constants} or by typing {\tt show variables}. These,
together with the physical units which are built into PyXPlot make it easy to
answer a wide range of questions very quickly. In the following examples, and
throughout this Users' Guide, we show the commands typed by the user in {\tt\bf
bold face}, preceded by PyXPlot prompts {\tt pyxplot>} and followed by the text
returned by PyXPlot. Any comments are shown in {\tt\it italic face} preceded by
a hash character ({\tt\it \#}).

\begin{itemize}
\item What is $80^\circ$F in Celsius?

\vspace{3mm}
\noindent{\tt pyxplot> {\bf print 80*unit(oF) / unit(oC)}}\newline
\noindent{\tt 26.666667}
\vspace{3mm}

\item How long does it take for light to travel from the Sun to the Earth?

\vspace{3mm}
\noindent{\tt pyxplot> {\bf print unit(AU) / phy\_c}}\newline
\noindent{\tt 499.00478 s}
\vspace{3mm}

\item What wavelength of light corresponds to the ionisation energy of hydrogen (13.6\,eV)?

\vspace{3mm}
\noindent{\tt pyxplot> {\bf print phy\_c * phy\_h / (13.6 * unit(eV))}}\newline
\noindent{\tt 91.164844 nm}
\vspace{3mm}

\item What is the escape velocity of the Earth?

\vspace{3mm}
\noindent{\tt pyxplot> {\bf print sqrt(2 * phy\_G * unit(Mearth) / unit(Rearth))}}\newline
\noindent{\tt 11.186948 km/s}
\vspace{3mm}
\end{itemize}

In addition, PyXPlot provides extensive functions for numerically solving
equations, which will be described in detail in Chapter~\ref{ch:numerics}. The
following example evaluates $\int_{0\,\mathrm{s}}^{2\,\mathrm{s}}
x^2\,\mathrm{d}x$:

\vspace{3mm}
\noindent{\tt pyxplot> {\bf print int\_dx(x**2,0*unit(s),2*unit(s))}}\newline
\noindent{\tt 2.6666667 s**3}
\vspace{3mm}

\noindent This example solves a simple pair of simultaneous equations of two variables:

\vspace{3mm}
\noindent{\tt pyxplot> {\bf solve x+y=1 , 2*x+3*y=7 via x,y}}\newline
\noindent{\tt pyxplot> {\bf print "x=\%s; y=\%s"\%(x,y)}}\newline
\noindent{\tt x=-4; y=5}
\vspace{3mm}

\noindent And this third example searches for the minimum of the function $\cos(x)$ closest to $x=0.5$:

\vspace{3mm}
\noindent{\tt pyxplot> {\bf x=0.5}}\newline
\noindent{\tt pyxplot> {\bf minimise cos(x) via x}}\newline
\noindent{\tt pyxplot> {\bf print x}}\newline
\noindent{\tt 3.1415927}
\vspace{3mm}

This tour has touched briefly upon a few areas of PyXPlot's functionality, but
has not described its facilities for producing vector graphics, which will be
discussed in detail in Chapter~\ref{ch:vector_graphics} with numerous examples.

\section{Acknowledgments}

The inspiration for PyXPlot came from two sources, to which PyXPlot owes a
considerable historical debt.  PyXPlot's interface was heavily motivated by
\gnuplot's simple and intuitive interface, which was devised by Thomas Williams
and Colin Kelley and has more recently been developed by many others. PyXPlot's
graphical output engine was heavily motivated by the PyX\index{PyX} graphics
library for Python, originally written by J\"org Lehmann and Andr\'e Wobst and
more recently developed by a larger team.  Versions of PyXPlot prior to $0.8$
used PyX to produce their graphical output, and though version~$0.8$ uses its
own graphics engine, it continues to bear many similarities to PyX.

Several other people have made very substantial contributions to PyXPlot's
development. Matthew Smith provided extensive advice on many mathematical
matters which arose during its development, provided C implementations of the
Airy functions and the Riemann zeta function for general complex inputs, and
suggested many other mathematical functions which ought to be made available.
Dave Ansell provided many good ideas which have helped to shape PyXPlot's
interface. The writing of PyXPlot's postscript engine was substantially eased
thanks to the help of Michael Rutter, who happily shared his code and past
experiences with us; the implementation of the {\tt image} command is
substantially his work.

We are also very grateful to our team of alpha testers, without whose work this
release of PyXPlot would doubtless contain many more bugs than it does:
especial thanks go to Ian Abel, Rachel Holdforth, XXX. Of course, the authors
remain solely responsible for any bugs which remain.

Finally, we would like to think all of the users who have got in touch with us
by email since PyXPlot was first released on the web in~2006. Your feedback and
suggestions have been gratefully received.

\section{License}

This manual and the software which it describes are both copyright \copyright\
Dominic Ford~2006--2010, Ross Church~2008--2010. They are distributed under the
GNU General Public License (GPL) Version~2, a copy of which is provided in the
{\tt COPYING} file in this distribution.\index{General Public
License}\index{license} Alternatively, it may be downloaded from the web, from
the following location:\\ \url{http://www.gnu.org/copyleft/gpl.html}.

