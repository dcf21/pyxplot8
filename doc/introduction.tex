% INTRODUCTION.TEX
%
% The documentation in this file is part of PyXPlot <http://www.pyxplot.org.uk>
%
% Copyright (C) 2006-8 Dominic Ford <coders@pyxplot.org.uk>
%               2008   Ross Church
%
% $Id$
%
% PyXPlot is free software; you can redistribute it and/or modify it under the
% terms of the GNU General Public License as published by the Free Software
% Foundation; either version 2 of the License, or (at your option) any later
% version.
%
% You should have received a copy of the GNU General Public License along with
% PyXPlot; if not, write to the Free Software Foundation, Inc., 51 Franklin
% Street, Fifth Floor, Boston, MA  02110-1301, USA

% ----------------------------------------------------------------------------

% LaTeX source for the PyXPlot Users' Guide

\chapter{Introduction} \pagenumbering{arabic}

\label{introduction}

\section{Introduction to PyXPlot and Ode to \gnuplot}

{\sc PyXPlot} is a stand-alone command-line graphing package that is simple to
use yet produces high-quality attractive output suitable for use in
publications. For ease of use, its interface is based loosely upon that of the
ever-popular {\sc \gnuplot}\index{gnuplot} plotting program. This means that
\gnuplot\ users will find most simple tasks can be achieved with the same
commands that they are used to in PyXPlot, and that they can get started
straight away. However, PyXPlot produces considerably neater output than
\gnuplot; for example, it renders all text labels automatically in the \LaTeX\
typesetting environment, making it straightforward to label graphs with
mathematical expressions and to use the same fonts which you use elsewhere.
PyXPlot also extends \gnuplot's interface to provide new commands to make it a
powerful tool for producing vector graphics, and to automate many
commonly-required data-processing and numerical computing tasks.  Later in this
manual, we will see that PyXPlot may even be useful as a simple calculator to
users who have no intention of using its plotting facilities: it can perform
dimensional analysis on equations and convert quantities between different
units -- for example, between kilograms and ounces -- in a way which most other
simple calculators cannot.

So, perhaps you're wondering why we've been so mad as to write a plotting
program which borrows its interface from another, 25-year-old program. Isn't it
time for a shiny new twenty-first century plotting package? Well, yes and no.
\gnuplot\ may have its flaws, but its interface isn't one of them. It's telling
that, 25 years after its first appearance, \gnuplot\ remains the default choice
for so many people. Part of that may be because it's so readily available as a
free component of most Linux systems, but it also has a uniquely simple and
intuitive interface. If you have a datafile called {\tt data.dat} that you'd
like to plot, you can make a first attempt at plotting it by simply typing:
\begin{verbatim}
plot 'data.dat'
\end{verbatim}
That's very short and easy to remember, and if it's still too much typing for
you, you're even allowed to abbreviate {\tt plot} to {\tt p}. More advanced commands are barely any more difficult to remember, and there are no prizes for guessing what
\begin{verbatim}
set title 'A plot of the population of London against time'
\end{verbatim}
might do.

The strengths of \gnuplot\ become particularly apparent when it is compared
with some of its commercial rivals, for example, {\sc Maple}\index{Maple}, {\sc
Mathematica}\index{Mathematica}, {\sc Origin}\index{Origin}, and {\sc
SuperMongo}\index{SuperMongo}. It has to be granted that with a bit of work,
these can almost invariably produce prettier results than \gnuplot, though, of
course, the come with considerable price tags and licensing restrictions. But
as well as being more powerful, these tools are also considerably more fiddly
and time-consuming to use.

Of course, \gnuplot\ also has some open-source rivals. Back in the 1980s, one
of its main rivals was {\sc Pgplot}\index{Pgplot}: a somewhat more flexible
program, but one which could only be called from within another program. The
user was expected to have some programming experience, and required to put it
into practice to produce every plot. The same is true of several of \gnuplot's
more modern rivals, including the {\sc MatPlotLib}\index{MatPlotLib} and {\sc
PyX}\index{PyX} packages for the Python programming language. Both of these
significantly improve upon the quality of the plots produced by \gnuplot, but
both of these have interfaces more akin to programming languages than a simple
command language.

So, in conclusion, we heartily congratulate the authors of \gnuplot\ for the
work they did back in the mid-1980s, and our borrowing of some parts of their
interface is a tribute to their work. The graphic output which \gnuplot\
produces may not be the most attractive, but its interface is a fine piece of
design.

\section{The geekier reasons why \gnuplot\ is great}

We wouldn't want to give the impression that the only reason why we like
\gnuplot's interface is its ease of use. There are some subtler reasons why it
turns out to work very well in practice. First of all, as with any
command-driven UNIX program, it can be controlled in many different ways
depending upon what you want to do. The user can type commands into a live
terminal when experimenting to begin with, and then later speed up the process
by writing a series of commands into a text file to be run as a script. The
really brave can send \gnuplot\ commands through a UNIX pipe from another
process. In this sense, \gnuplot\ wins over many of its graphically-driven
rivals, which rarely have facilities to automatically produce a hundred similar
plots of a hundred similar \datafile s.

\gnuplot's interface also has another distinct advantage over other plotting
packages which insist upon being called from within a programming language. It
requires that data be written to a file on disk before it can be plotted. By
contrast, when plotting is done from within a programming language, it is often
tempting for the user to write programs which both perform calculations and
plot the results immediately. This sounds really neat, but it can also be
dangerous. Remembering to store a copy of the data used to produce a graph
becomes a secondary priority. Months later, when the need arises to replot the
same data in a different form, or to compare it with newer data, remembering
how to use a hurriedly written program can prove tricky -- especially if the
program was originally written by someone else. But a simple \datafile\ is
quite straightforward to plot.

\section{The structure of this manual}

The similarity of PyXPlot's interface to that of \gnuplot\ is such that simple
scripts written for \gnuplot\ should work with PyXPlot with minimal
modification; \gnuplot\ users should be able to get started very quickly.
However, PyXPlot is still a work in progress, and a small number of \gnuplot's
features are still missing.  A detailed list of which features are supported
can be found in Section~\ref{missing_features}. The new features which have
been added to the interface are described in
Chapters~\ref{gnuplot_ext_first}\,--\,\ref{gnuplot_ext_last}.

A brief overview of \gnuplot's interface is provided for novice users in
Chapter~\ref{gnuplot_intro}. Past \gnuplot\ users may skip over this chapter,
though their attention is drawn to one of the key changes to the interface --
namely that all textual labels on plots are now rendered using the \LaTeX\
typesetting environment. This does unfortunately introduce some incompatibility
with \gnuplot, since some strings which were valid before are no longer valid
(see Section~\ref{sec:latex_incompatibility} for more details). For
example:\index{latex}

\begin{dontdo}
set xlabel 'x\^{}2'
\end{dontdo}

\noindent would have been valid in \gnuplot, but now needs to be written in
\LaTeX\ mathmode as:

\begin{dodo}
set xlabel '\$x\^{}2\$'
\end{dodo}

\noindent The nuisance of this incompatibility is surely far outweighed by the
power that \LaTeX\ brings, however. For users with no prior knowledge of
\LaTeX\ we recommend Tobias Oetiker's\index{Tobias Oetiker} excellent
introduction, {\it The Not So Short Guide to \LaTeX $2\epsilon$}\index{Not So
Short Guide to \LaTeX $2\epsilon$, The}\footnote{Download from:\\
\url{http://www.ctan.org/tex-archive/info/lshort/english/lshort.pdf}}.

\section{System Requirements}

PyXPlot works on most UNIX-like operating systems. We have tested it under
Linux, Solaris\index{Solaris} and MacOS X\index{MacOS X}, and believe that it
should work on other similar POSIX systems. It requires that the following
software packages (not included) be installed:\index{system requirements}

\vspace{0.5cm}
\begin{itemize}
\item Ghostview (or {\tt ggv}) \index{Ghostview}
\item The Gnu Scientific Library (version 1.10+) \index{gsl}
\item The Gnu Readline Library (version 5+) \index{readline}
\item latex (version $2\epsilon$) \index{latex}
\item libxml2 (version 2.6+) \index{libxml}
\item libjpeg (version 6.2+) \index{libjpeg}
\end{itemize}
\vspace{0.5cm}

\noindent In order for PyXPlot to display plots live on the screen, it is also neccessary to install the following package, though it is still possible for PyXPlot to generate image files on disk without it:

\vspace{0.5cm}
\begin{itemize}
\item GhostScript
\end{itemize}
\vspace{0.5cm}

Debian and Ubuntu users can find the above software in the packages {\tt gs},
{\tt gv}, {\tt imagemagick}, {\tt libgsl0-dev}, {\tt libjpeg62-dev}, {\tt
libreadline5-dev}, {\tt libxml2-dev}, {\tt tetex-extra}.\index{Debian
Linux}\index{Ubuntu Linux}\index{installation!under
Debian}\index{installation!under Ubuntu}

\section{Installation}
\index{installation}

\subsection{Installation within Linux Distributions}

PyXPlot is available as a user-installable package within some Linux
distributions. Gentoo\index{Gentoo Linux}\index{installation!under
Gentoo}\footnote{See \url{http://gentoo-portage.com/sci-visualization/pyxplot}}
and Ubuntu\index{Ubuntu Linux}\index{installation!under Ubuntu}\footnote{Note
that there is an error in the packaging of PyXPlot in {\it Hardy Herron} and
{\it Intrepid Ibex}, which means that the {\tt tetex-extra} package, upon which
PyXPlot depends, is not automatically installed with PyXPlot.} already have
such packages. Debian will have such a package in their next release, {\it
Lenny}, which is scheduled for late 2008. Alternatively, and to ensure that
they are using the latest version, Debian and Ubuntu users can download the
package from the PyXPlot website and install it manually by typing:

\begin{verbatim}
dpkg -i pyxplot_0.8.0.deb
\end{verbatim}

\subsection{Installation as User}

The following steps describe the installation of PyXPlot from a {\tt .tar.gz}
archive by a user without superuser (i.e.\ root) access to his machine. It is
assumed that the packages listed above have already been installed; if they are
not, you will need to contact your system
administrator.\index{installation!user-level}

\begin{itemize}
\item Unpack the distributed .tar.gz:

\begin{verbatim}
tar xvfz pyxplot_0.8.0.tar.gz
cd pyxplot
\end{verbatim}

\item Run the installation script:

\begin{verbatim}
./configure
make
\end{verbatim}

\item Finally, start PyXPlot:

\begin{verbatim}
./bin/pyxplot
\end{verbatim}

\end{itemize}

\subsection{System-wide Installation}

Having completed the steps described above, PyXPlot may be installed
system-wide by a superuser with the following additional
step:\index{installation!system-wide}

\begin{verbatim}
make install
\end{verbatim}

By default, the PyXPlot executable installs to {\tt /usr/local/bin/pyxplot}.
If desired, this installation path may be modified in the file {\tt
Makefile.skel}, by changing the variable {\tt USRDIR} in the first line to an
alternative desired installation location.

PyXPlot may now be started by any system user, simply by typing:

\begin{verbatim}
pyxplot
\end{verbatim}

\section{Credits}

%We would like to express our gratitude to several people who have contributed
%to PyXPlot -- first and foremost to J\"org Lehmann\index{Lehmann, J\"org},
%Andr\'e Wobst\index{Wobst, Andr\'e} and Michael Schindler\index{Schindler,
%Michael} for writing the PyX\index{PyX} graphics library for Python, upon which
%this software is heavily built. We would also like to think all of the users
%who have got in touch with us by email since PyXPlot was first released on the
%web.  Your feedback and suggestions have been gratefully received.

\section{License}

This manual and the software which it describes are both copyright \copyright\
Dominic Ford 2006-9, Ross Church 2009. They are distributed under the GNU
General Public License (GPL) Version~2, a copy of which is provided in the {\tt
COPYING} file in this distribution.\index{General Public
License}\index{license} Alternatively, it may be downloaded from the web, from
the following location:\\ \url{http://www.gnu.org/copyleft/gpl.html}.

