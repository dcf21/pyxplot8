% PROGRAMMING.TEX
%
% The documentation in this file is part of PyXPlot
% <http://www.pyxplot.org.uk>
%
% Copyright (C) 2006-2010 Dominic Ford <coders@pyxplot.org.uk>
%               2009-2010 Ross Church
%
% $Id$
%
% PyXPlot is free software; you can redistribute it and/or modify it under the
% terms of the GNU General Public License as published by the Free Software
% Foundation; either version 2 of the License, or (at your option) any later
% version.
%
% You should have received a copy of the GNU General Public License along with
% PyXPlot; if not, write to the Free Software Foundation, Inc., 51 Franklin
% Street, Fifth Floor, Boston, MA  02110-1301, USA

% ----------------------------------------------------------------------------

% LaTeX source for the PyXPlot Users' Guide

\chapter{Programming and Flow Control}

In this chapter we describe the facilities which PyXPlot has for automating
repetitive tasks. We begin by introducing string variables, which can be useful
for auto-generating titles and labels for graphs which are being produced in
batch jobs. Then we discuss the loop constructs which make it possible to
produce similar plots of many different \datafile s, or to perform calculations
in an iterative fashion.  Finally, we introduce a simple framework which
automatically re-executing PyXPlot scripts whenever they change, allowing plots
to be automatically regenerated whenever the scripts used to produce them are
modified.

\section{String variables}
\label{sec:stringvars}

String variables can be assigned in a manner analogous to numeric variables,
using the $=$ command followed by a string enclosed in quotation marks. They can
then be used wherever a quoted string could be used, for example as a filename
or a plot title, as in:\index{variables!string}

\begin{verbatim}
plotname = "The Growth of a Rabbit Population"
set title plotname
\end{verbatim}

\subsection{The String Substitution Operator}

Strings may also be put together using the string substitution operator, {\tt
\%}\index{\% operator@{\tt \%} operator}, which works in a similar fashion to
the Python string substitution operator\index{string operators!substitution}. This
is described in detail in Section~\ref{sec:string_subs_op}.  For example, to
concatenate the two strings contained in variables {\tt a} and {\tt b} into
variable {\tt c} one would run:\index{string operators!concatenation}

\begin{verbatim}
c = "%s%s"%(a,b)
\end{verbatim}

One common practical application of these string operators is to label plots
with the title of the \datafile\ being plotted, as in:

\begin{verbatim}
filename="data_file.dat"
title="A plot of the data in {\tt %s}."%(filename)
title=~s/_/\_/g # Underscore is a reserved character in LaTeX
set title title
plot filename
\end{verbatim}

\subsection{Regular Expressions}

String variables can be modified using the search-and-replace string
operator\index{string operators!search and replace}\footnote{Programmers with
experience of {\tt perl} will recognise this syntax.}, =$\sim$\index{=$\sim$
operator}, which takes a regular expression with a syntax similar to that
expected by the shell command {\tt sed}\index{sed shell command@{\tt sed} shell
command} and applies it to the relevant string.\footnote{Regular expression
syntax is a massive subject, and is beyond the scope of this manual. The
official GNU documentation for the {\tt sed} command is heavy reading, but
there are many more accessible tutorials on the web.}\index{regular
expressions} For example:

\begin{verbatim}
twister="seven silver soda syphons"
twister =~ s/s/th/
print twister
\end{verbatim}

Note that only the {\tt s} (substitute) command of {\tt sed} is implemented in
PyXPlot. Any character can be used in place of the {\tt /} characters in the
above example, for example:

\begin{verbatim}
twister =~ s@s@th@
\end{verbatim}

\noindent Flags can be passed, as in {\tt sed} or {\tt perl}, for example:

\begin{verbatim}
twister =~ s@s@th@g
\end{verbatim}

\noindent Table~\ref{tab:re_flags} lists all of the regular expression flags
recognised by the =$\sim$ operator.

\begin{table}
\framebox[\textwidth]{\footnotesize
\begin{tabular}{p{5mm}p{10.5cm}}
{\tt g} & Replace {\it all} matches of the pattern; by default, only the first match is replaced. \\
{\tt i} & Perform case-insensitive matching, such that expressions like {\tt [A-Z]} will match lowercase letters, too. \\
{\tt l} & Make {\tt $\backslash$w}, {\tt $\backslash$W}, {\tt $\backslash$b}, {\tt $\backslash$B}, {\tt $\backslash$s} and {\tt $\backslash$S} dependent on the current locale. \\
{\tt m} & When specified, the pattern character {\tt \^{}} matches the beginning of the string and the beginning of each line immediately following each newline. The pattern character {\tt \$} matches at the end of the string and the end of each line immediately preceding each newline. By default, {\tt \^{}} matches only the beginning of the string, and {\tt \$} only the end of the string and immediately before the newline, if present, at the end of the string. \\
{\tt s} & Make the {\tt .} special character match any character at all, including a newline; without this flag, {\tt .} will match anything except a newline. \\
{\tt u} & Make {\tt $\backslash$w}, {\tt $\backslash$W}, {\tt $\backslash$b}, {\tt $\backslash$B}, {\tt $\backslash$s} and {\tt $\backslash$S} dependent on the Unicode character properties database. \\
{\tt x} & This flag allows the user to write regular expressions that look nicer. Whitespace within the pattern is ignored, except when in a character class or preceded by an unescaped backslash. When a line contains a {\tt \#}, neither in a character class or preceded by an unescaped backslash, all characters from the leftmost such {\tt \#} through to the end of the line are ignored. \\
\end{tabular}}
\caption{A list of the flags accepted by the =$\sim$ operator. Most are rarely used, but the {\tt g} flag is very useful.}
\label{tab:re_flags}
\end{table}

\section{Conditionals}

The {\tt if} statement\indcmd{if} can be used to conditionally execute a series
of commands only when a certain criterion is satisfied. In its simplest form,
its syntax is

\begin{verbatim}
if <expression> {
  ....
 }
\end{verbatim}

\noindent where the expression can take the form of, for example, {\tt x<0} or
{\tt y==1}. Note that the operator {\tt ==} is used to test the equality of two
algebraic expressions; the operator {\tt =} is only used to assign values to
variables and functions. A full list of the operators available can be found in
Table~\ref{tab:operators_table}. As in many other programming languages,
algebraic expressions are deemed to be true if they evaluate to any non-zero
value, and false if they exactly equal zero. Thus, the following two examples
are entirely legal syntax, and the first {\tt print} statement will execute,
but the second will not:

\begin{verbatim}
if 2*3 {
  print "2*3 is True"
 }

if 2-2 {
  print "2-2 is False"
 }
\end{verbatim}

\noindent The variables {\tt True} and {\tt False} are predefined constants,
which evaluate to~1 and~0 respectively, making the following syntax legal:

\begin{verbatim}
if False {
  print "Never gets here"
 }
\end{verbatim}

As in C, the block the commands which are to be executed if the conditional is
true are enclosed in braces (i.e. {\tt \{~\}}).  There are, however, some rules
about the arrangement of whitespace.  The block of commands must begin on a new
line after the {\tt if} statement. The closing brace must be on a line by
itself at the end of the block. Alternatively, semi-colons may, as always, be
used in place of new lines. The opening brace may be placed either on the same
line as the {\tt if} statement, or on the following line:

\begin{verbatim}
if (x==0)
 {
  print "x is zero"
 }

if (x==0) { ; print "x is zero" ; }
\end{verbatim}

After such an {\tt if} clause, it is possible to string together further
conditions in {\tt else if} clauses, perhaps with a final {\tt else} clause.
For example:

\begin{verbatim}
if (x==0)
 {
  print "x is zero"
 } else if (x>0) {
  print "x is positive"
 } else {
  print "x is negative"
 }
\end{verbatim}

Here, as previously, the first script block is executed if the first
conditional, {\tt x==0}, is true. If this script block is not executed, the
second conditional, {\tt x>0}, is then tested. If this is true, then the second
script block is executed.  The final script block, following the {\tt else}, is
executed if none of the preceeding conditionals have been true. Any number of
{\tt else if} statements can be chained after one another, and a final {\tt
else} statement can always be optionally supplied. The {\tt else} and {\tt else
if} statements must always be placed on the same line as the closing brace of
the preceeding script block.

The precise way in which a string of {\tt else if} statements are arranged in a
PyXPlot script is a matter of taste: the following is a more compact but
equivalent version of the example given above:

\begin{verbatim}
if      (x==0) { ; print "x is zero"     ; } \
else if (x> 0) { ; print "x is positive" ; } \
else           { ; print "x is negative" ; }
\end{verbatim}

\section{For Loops}

For loops may used to execute a series of commands multiple times. PyXPlot's
\indcmdt{for} command has the syntax:

\begin{verbatim}
for <variable> = <start> to <end> {step <step>} {loopname <name>}
 {
  ....
 }
\end{verbatim}

\noindent The first time that the script block is executed, the variable named
at the start of the {\tt for} statement has the value given for {\tt start}.
Upon each iteration of the loop, this is incremented by amount {\tt step}. The
loop finishes when the value equals or exceeds {\tt end}. If {\tt step} is
negative, then {\tt end} is required to be less than {\tt start}. A step size
of zero is considered to be an error.  The iterator variable can have any
physical dimensions, so long as {\tt start}, {\tt end} and {\tt step} all have
the same dimensions, but the iterator variable must always be a real number. If
no step size is given then a step size of unity is assumed.

As an example, the following script would print the numbers 0, 2, 4, 6 and 8:

\begin{verbatim}
for x = 0 to 10 step 2
 {
  print x
 }
\end{verbatim}

The same rules concerning the placement of brace characters apply to the
\indcmdt{for} command as to the {\tt if} command.

The optional {\tt loopname} which can be specified in the {\tt for} statement
is used in conjunction with the {\tt break} and {\tt continue} statements which
will be introduced in Section~\ref{sec:breakcontinue}.

\section{Foreach Loops}
\indcmd{foreach}
\index{wildcards}

Foreach loops may be used to run a script block once for each item in a list.
The list may either take the form of an explicit bracketed comma-separated list
of items, or the form of a filename wildcard, as in the examples:

\begin{verbatim}
foreach x in (-1,pi,10)
 {
  print x
 }

foreach x in "*.dat"
 {
  print x
 }
\end{verbatim}

The first of these loops would iterate three times, with the variable {\tt x}
holding the values $-1$, $\pi$ and $10$ in turn. The second of these loops
would search for any \datafile s in the user's current directory with filenames
ending in {\tt .dat} and iterate for each of them. As previously, the wildcard
character {\tt *} matches any string of characters, and the character {\tt ?}
matches any single character. Thus, {\tt foo?.dat} would match {\tt foo1.dat}
and {fooX.dat}, but not {\tt foo.dat} or {\tt foo10.dat}. The effect of the
{\tt print} statement in this particular example would be rather similar to
typing:

\begin{verbatim}
!ls *.dat
\end{verbatim}

The quotes around each supplied search string are compulsory if any of the
characters in the search string are alphanumeric, but optional otherwise. Since
both of the wildcard characters {\tt *} and {\tt ?} are non-alphanumeric, the
quotes are compulsory in most useful cases.  An error is returned if there are
no files in the present directory which match the supplied wildcard. The
following example would produce plots of all of the \datafile
s in the current directory as {\tt eps} files with matching filenames:

\begin{verbatim}
set terminal eps
foreach x in "*.dat"
 {
  outfilename =  x
  outfilename =~ s/dat/eps/
  set output outfilename
  plot x using 1:2
 }
\end{verbatim}

\section{Foreach Datum Loops}
\label{sec:foreach_datum}

Foreach datum loops are similar to foreach loops in that they run a script
block once for each item in a list.  In this case, however, the list in
question is the list of \datapoint s in a file. The syntax of the
\indcmdt{foreach datum} is similar to that of the commands met in the previous
chapter for acting upon \datafile s: the standard modifiers {\tt every}, {\tt
index}, {\tt select} and {\tt using} can be used to select which columns of the
\datafile, and which subset of the datapoints, should be used:

\begin{verbatim}
foreach datum i,j,name in "data.dat" using 1:2:"%s"%($3)
 {
  ...
 }
\end{verbatim}

The \indcmdt{foreach datum} is followed by a comma-separated list of the
variable(s) which are to be read from the \datafile on each iteration of the
loop. The {\tt using} modifier specifies the columns or rows of data which are
to be used to set the values of each variable. In this example, the third
variable, {\tt name}, is set using a quoted string, indicating that it will be
set to equal whatever string of text is found in the third column of the
\datafile.

\section{While and Do Loops}

The \indcmdt{while} command may be used to continue running a script block
until some stopping criterion is met. Two types of while loop are supported:

\begin{verbatim}
while <criterion> {loopname <name>}
 {
  ....
 }

do {loopname <name>}
 {
  ....
 } while <criterion>
\end{verbatim}
\indcmd{do}

In the former case, the enclosed script block is executed repeatedly, and the
algebraic expression supplied to the \indcmdt{while} command is tested
immediately before each repetition. If it tests false, then the loop finishs.
The latter case is very similar, except that the supplied algebraic expression
is tested immediately {\it after} each repetition. Thus, the former example may
never actually execute the supplied script block if the looping criterion tests
false upon the first iteration, but the latter example is always guaranteed to
run its script block at least once.

The following example would continue looping indefinitely until stopped by the
user, since the value {\tt 1} is considered to be true:

\begin{verbatim}
while (1)
 {
  print "Hello, world!"
 }
\end{verbatim}

\section{The {\tt break} and {\tt continue} statements}
\label{sec:breakcontinue}
\indcmd{break}
\indcmd{continue}

The {\tt break} and {\tt continue} statements may be placed within loop
structures to interrupt their iteration. The {\tt break} statement terminates
execution of the smallest loop currently being executed, and PyXPlot resumes
execution at the next statement after the closing brace which marks the end of
that loop structure. The {\tt continue} statement terminates execution of the
{\tt current iteration} of the smallest loop currently being executed, and
execution proceeds with the next iteration of that loop, as demonstrated by the
following pair of examples.

\begin{verbatim}
pyxplot> for i=0 to 4 {
for... >  if (i==2) { ; break; }
for... >  print i
for... > }
0
1
pyxplot> for i=0 to 4 {
for... >  if (i==2) { ; continue ; }
for... >  print i
for... > }
0
1
3
\end{verbatim}

Note that if several loops are nested, the {\tt break} and {\tt continue}
statements only act upon the innermost loop. If either statement is encountered
outside of a loop structure, an error results.

Optionally, the {\tt for}, {\tt foreach}, {\tt do} and {\tt while}, may be
supplied with a name for the loop, prefixed by the word {\tt loopname}, as in
the examples:

\begin{verbatim}
for i=0 to 4 loopname iloop
...
foreach i in "*.dat" loopname DatafileLoop
...
\end{verbatim}

When loops are given names, the {\tt break} and {\tt continue} statements may
be followed by the name of the loop to be broken out of, allowing the user to
act upon loops other than the innermost one.

\section{Subroutines}
\indcmd{subroutine}
\label{sec:subroutines}

Subroutines are similar to mathematical functions (see
Section~\ref{sec:functions}), and once defined, can be used anywhere in
algebraic expressions, just as functions can be.  However, instead of being
defined by a single algebraic expression, whenever a subroutine is evaluated, a
block of PyXPlot commands of arbitrary length is executed. This gives much
greater flexibility for implementing complex algorithms. Subroutines are
defined using the following syntax:
\begin{verbatim}
subroutine <name>(<variable1>,...)
 {
  ...
  return <value>
 }
\end{verbatim}
Where {\tt name} is the name of the subroutine, {\tt variable1} is an argument
taken by the subroutine, and the value given to the {\tt return} statement is
the value returned to the caller. Once the {\tt return} statement is reached,
execution of the subroutine is terminated. The following two examples would
produce entirely equivalent results:
\begin{verbatim}
f(x,y) = x*sin(y)

subroutine f(x,y)
 {
  return x*sin(y)
 }
\end{verbatim}
In either case, the function/subroutine could be evaluated by typing:
\begin{verbatim}
print f(1,pi/2)
\end{verbatim}
If a subroutine ends without any value being returned using the {\tt return}
statement, then a value of zero is returned.

Subroutines may serve one of two purposes. In many cases they are used to
implement complicated mathematical functions for which no simple algebraic
expression may be given. Secondly, they may be used to repetitively execute a
set of commands whenever they are required. In the latter case, the subroutine
may not have a return value, but may merely be used as a mechanism for
encapsulating a block of commands.  In this case, the \indcmdt{call} may be
used to execute a subroutine, discarding any return value which it may produce,
as in the example:
\begin{verbatim}
pyxplot> subroutine f(x,y)
subrtne>  {
subrtne>   print "%s - %s = %s"%(x,y,x-y)
subrtne>  }
pyxplot> 
pyxplot> call f(2,1)
2 - 1 = 1
pyxplot> call f(5*unit(inch), 10*unit(mm))
127 mm - 10 mm = 117 mm
\end{verbatim}

\section{The \indcmdt{exec}}

The \indcmdt{exec} can be used to execute PyXPlot commands contained within
string variables. For example:

\begin{verbatim}
terminal="eps"
exec "set terminal %s"%(terminal)
\end{verbatim}

It can also be used to write obfuscated PyXPlot scripts.

\section{Shell Commands}

Shell commands\index{shell commands!executing} may be executed directly from
within PyXPlot by prefixing them with an \indcmdts{!} character. The
remainder of the line is sent directly to the shell, for example:

\begin{verbatim}
!ls -l
\end{verbatim}

\noindent Semi-colons cannot be used to place further PyXPlot commands after a
shell command on the same line.

\begin{dontdo}
!ls -l ; set key top left
\end{dontdo}

It is also possible to substitute the output of a shell command into a PyXPlot
command. To do this, the shell command should be enclosed in back-quotes (`).
For example:\index{backquote character}\index{shell commands!substituting}

\begin{verbatim}
a=`ls -l *.ppl | wc -l`
print "The current directory contains %d PyXPlot scripts."%(a)
\end{verbatim}

It should be noted that back-quotes can only be used outside quotes. For
example:

\begin{dontdo}
set xlabel '`ls`'
\end{dontdo}

\noindent will not work. The best way to do this would be:

\begin{dodo}
set xlabel `echo "'" ; ls ; echo "'"`
\end{dodo}

Note that it is not possible to change the current working directory by sending
the {\tt cd} command to a shell, as this command would only change the working
directory of the shell in which the single command is executed:

\begin{dontdo}
!cd ..
\end{dontdo}

PyXPlot has its own \indcmdt{cd} for this purpose, as well as its own
\indcmdt{pwd}:

\begin{dodo}
cd ..
\end{dodo}

\section{Script Watching: pyxplot\_watch}

PyXPlot includes a simple tool for watching command script files and executing
them whenever they are modified. This may be useful when developing a command
script, if one wants to make small modifications to it and see the results in a
semi-live fashion. This tool is invoked by calling the {\tt
pyxplot\_watch}\index{pyxplot\_watch}\index{watching scripts} command from a
shell prompt. The command-line syntax of {\tt pyxplot\_watch} is similar to
that of PyXPlot itself, for example:

\begin{verbatim}
pyxplot_watch script.ppl
\end{verbatim}

\noindent would set {\tt pyxplot\_watch} to watch the command script file
{\tt script.ppl}. One difference, however, is that if multiple script files are
specified on the command line, they are watched and executed independently,
\textit{not} sequentially, as PyXPlot itself would do. Wildcard characters can
also be used to set {\tt pyxplot\_watch} to watch multiple
files.\footnote{Note that {\tt pyxplot\_watch *.script} and
{\tt pyxplot\_watch $\backslash$*.script} will behave differently in most
UNIX shells.  In the first case, the wildcard is expanded by your shell, and a
list of files passed to {\tt pyxplot\_watch}. Any files matching the
wildcard, created after running {\tt pyxplot\_watch}, will not be picked up.
In the latter case, the wildcard is expanded by {\tt pyxplot\_watch} itself,
which {\it will} pick up any newly created files.}

This is especially useful when combined with \ghostview's\index{Ghostview}
watch facility. For example, suppose that a script {\tt foo.ppl} produces
postscript output {\tt foo.ps}. The following two commands could be used to
give a live view of the result of executing this script:

\begin{verbatim}
gv --watch foo.ps &
pyxplot_watch foo.ppl
\end{verbatim}

