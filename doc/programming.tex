% PROGRAMMING.TEX
%
% The documentation in this file is part of PyXPlot
% <http://www.pyxplot.org.uk>
%
% Copyright (C) 2006-8 Dominic Ford <coders@pyxplot.org.uk>
%               2008   Ross Church
%
% $Id$
%
% PyXPlot is free software; you can redistribute it and/or modify it under the
% terms of the GNU General Public License as published by the Free Software
% Foundation; either version 2 of the License, or (at your option) any later
% version.
%
% You should have received a copy of the GNU General Public License along with
% PyXPlot; if not, write to the Free Software Foundation, Inc., 51 Franklin
% Street, Fifth Floor, Boston, MA  02110-1301, USA

% ----------------------------------------------------------------------------

% LaTeX source for the PyXPlot Users' Guide

\chapter{PyXPlot as a programming language}
\label{gnuplot_ext_first}

This chapter describes ways of programming PyXPlot.

\section{Variables}

As has already been hinted at in Section~\ref{string_subs_op}, PyXPlot
recognises two types of variables: numeric variables and string variables.  The
former can be assigned using any valid mathematical expression. For example:

\begin{verbatim}
a = 5.2 * sqrt(64)
\end{verbatim}

\noindent would assign the value 41.6 to the variable {\tt a}.  Numerical variables can
subsequently be used in mathematical expressions themselves, for example:

\begin{verbatim}
a=2*pi
plot [0:1] sin(a*x)
\end{verbatim}

\noindent String variables can be assigned in an analogous manner, by enclosing
the string in quotation marks. They can then be used wherever a quoted string
could be used, for example as a filename or a plot title, as
in:\index{variables!string}

\begin{verbatim}
plotname = "The Growth of a Rabbit Population"
set title plotname
\end{verbatim}

String variables can be modified using the search-and-replace string
operator\index{string operators!search and replace}\footnote{Programmers with
experience of {\tt perl} will recognise this syntax.}, =$\sim$\index{=$\sim$
operator}, which takes a regular expression with a syntax similar to that
expected by the shell command {\tt sed}\index{sed shell command@{\tt sed} shell
command} and applies it to the relevant string.\footnote{Regular expression
syntax is a massive subject, and is beyond the scope of this manual. The
official GNU documentation for the {\tt sed} command is heavy reading, but
there are many more accessible tutorials on the web.}\index{regular
expressions} For example:

\begin{verbatim}
twister="seven silver soda syphons"
twister =~ s/s/th/
print twister
\end{verbatim}

Note that only the {\tt s} (substitute) command of {\tt sed} is implemented in
PyXPlot. Any character can be used in place of the {\tt /} characters in the
above example, for example:

\begin{verbatim}
twister =~ s@s@th@
\end{verbatim}

\noindent Flags can be passed, as in {\tt sed} or {\tt perl}, for example:

\begin{verbatim}
twister =~ s@s@th@g
\end{verbatim}

\noindent Table~\ref{re_flags} lists all of the regular expression flags
recognised by the =$\sim$ operator.

\begin{table}
\framebox[\textwidth]{\footnotesize
\begin{tabular}{p{5mm}p{10.5cm}}
{\tt g} & Replace {\it all} matches of the pattern; by default, only the first match is replaced. \\
{\tt i} & Perform case-insensitive matching, such that expressions like {\tt [A-Z]} will match lowercase letters, too. \\
{\tt l} & Make {\tt $\backslash$w}, {\tt $\backslash$W}, {\tt $\backslash$b}, {\tt $\backslash$B}, {\tt $\backslash$s} and {\tt $\backslash$S} dependent on the current locale. \\
{\tt m} & When specified, the pattern character {\tt \^{}} matches the beginning of the string and the beginning of each line immediately following each newline. The pattern character {\tt \$} matches at the end of the string and the end of each line immediately preceding each newline. By default, {\tt \^{}} matches only the beginning of the string, and {\tt \$} only the end of the string and immediately before the newline, if present, at the end of the string. \\
{\tt s} & Make the {\tt .} special character match any character at all, including a newline; without this flag, {\tt .} will match anything except a newline. \\
{\tt u} & Make {\tt $\backslash$w}, {\tt $\backslash$W}, {\tt $\backslash$b}, {\tt $\backslash$B}, {\tt $\backslash$s} and {\tt $\backslash$S} dependent on the Unicode character properties database. \\
{\tt x} & This flag allows the user to write regular expressions that look nicer. Whitespace within the pattern is ignored, except when in a character class or preceded by an unescaped backslash. When a line contains a {\tt \#}, neither in a character class or preceded by an unescaped backslash, all characters from the leftmost such {\tt \#} through the end of the line are ignored. \\
\end{tabular}}
\caption{A list of the flags accepted by the =$\sim$ operator. Most are rarely used, but the {\tt g} flag is very useful. This table is adapted from Guido van Rossum's\index{van Rossum, Guido} {\it Python Library Reference}\index{Python Library Reference}: \protect\url{http://docs.python.org/lib/node46.html}.}
\label{re_flags}
\end{table}

Strings may also be put together using the string substitution operator, {\tt
\%}\index{\% operator@{\tt \%} operator}, which works in a similar fashion to
Python string substitution operator\index{string operators!substitution}. This
is described in detail in Section~\ref{string_subs_op}.  For example, to
concatenate the two strings contained in variables {\tt a} and {\tt b} into
variable {\tt c} one would run:\index{string operators!concatenation}

\begin{verbatim}
c = "%s%s"%(a,b)
\end{verbatim}

One common practical application of these string operators is to label plots
with the title of the \datafile\ being plotted, as in:

\begin{verbatim}
filename="data_file.dat"
title="A plot of the data in {\tt %s}."%(filename)
title=~s/_/\_/g # Underscore is a reserved character in LaTeX
set title title
plot filename
\end{verbatim}

\section{The \indcmdt{exec}}

The \indcmdt{exec} can be used to execute PyXPlot commands contained within
string variables. For example:

\begin{verbatim}
terminal="eps"
exec "set terminal %s"%(terminal)
\end{verbatim}

It can also be used to write obfuscated PyXPlot scripts.

\section{Function Splicing}
\index{function splicing}
\index{splicing functions}

In PyXPlot, as in \gnuplot, user-defined functions may be declared on the
command line:

\begin{verbatim}
f(x) = x*sin(x)
\end{verbatim}

\noindent It is also possible to declare functions which are valid only over
certain ranges of argument space. For example, the following function would
only be valid within the range $-2<x<2$:\footnote{The syntax {\tt [-2:2]} can
also be written {\tt [-2 to 2]}.}

\begin{verbatim}
f(x)[-2:2] = x*sin(x)
\end{verbatim}

\noindent The following function would only be valid when all of ${a,b,c}$ were
in the range $-1 \to 1$:

\begin{verbatim}
f(a,b,c)[-1:1][-1:1][-1:1] = a+b+c
\end{verbatim}

If an attempt is made to evaluate a function outside of its specified range,
then an error results. This may be useful, for example, for plotting a function
only within some specified range. The following would plot the function
$\sinc(x)$, but only in the range $-2<x<7$:

\begin{verbatim}
f(x)[-2:7] = sin(x)/x
plot f(x)
\end{verbatim}

\begin{example}
\begin{center}
\includegraphics{examples/eps/ex_funcsplice1.eps}
\end{center}
\caption[A simple example of the use of function splicing to truncate a function]
{A simple example of the use of function splicing to truncate the function $\sinc(x)$ at $x=-2$ and $x=7$. See details in the text.}
\label{fig:ex_funcsplice1}
\end{example}

\label{splice} \noindent The output of this particular example can be seen in
Figure~\ref{fig:ex_funcsplice1}. A similar effect could also have been achieved
with the {\tt select} keyword; see Section~\ref{select_modifier}.

It is possible to make multiple declarations of the same function, over
different regions of argument space; if there is an overlap in the valid
argument space for multiple definitions, then later declarations take
precedence. This makes it possible to use different functional forms for
functions in different parts of parameter space, and is especially useful when
fitting functions to data, if different functional forms are to be spliced
together to fit different regimes in the data.

Another application of function splicing is to work with functions which do not
have analytic forms, or which are, by definition, discontinuous, such as
top-hat functions or Heaviside functions. The following example would define
$f(x)$ to be a Heaviside function:

\begin{verbatim}
f(x) = 0
f(x)[0:] = 1
\end{verbatim}

\noindent The following example would define $f(x)$ to follow the Fibonacci
sequence, though it is not at all computationally efficient, and it is
inadvisable to evaluate it for $x\gtrsim8$:

\begin{verbatim}
f(x) = 1
f(x)[2:] = f(x-1) + f(x-2)
plot [0:8] f(x)
\end{verbatim}

\begin{example}
\begin{center}
\includegraphics{examples/eps/ex_funcsplice2.eps}
\end{center}
\caption[An example of the use of function splicing to define a function which does not have an analytic form]
{An example of the use of function splicing to define a function which does not have an analytic form -- in this case, the Fibonacci sequence. See the text for details.}
\label{fig:ex_funcsplice2}
\end{example}

\noindent The output of this example can be seen in Figure~\ref{fig:ex_funcsplice2}

\section{Working with physical units}

units schemes

\section{Physical constants}

\section{Flow control}

\section{Script Watching: pyxplot\_watch}

PyXPlot includes a simple tool for watching command script files and executing
them whenever they are modified. This may be useful when developing a command
script, if one wants to make small modifications to it and see the results in a
semi-live fashion. This tool is invoked by calling the {\tt
pyxplot\_watch}\index{pyxplot\_watch}\index{watching scripts} command from a
shell prompt. The command-line syntax of {\tt pyxplot\_watch} is similar to
that of PyXPlot itself, for example:

\begin{verbatim}
pyxplot_watch script.ppl
\end{verbatim}

\noindent would set {\tt pyxplot\_watch} to watch the command script file
{\tt script.ppl}. One difference, however, is that if multiple script files are
specified on the command line, they are watched and executed independently,
\textit{not} sequentially, as PyXPlot itself would do. Wildcard characters can
also be used to set {\tt pyxplot\_watch} to watch multiple
files.\footnote{Note that {\tt pyxplot\_watch *.script} and
{\tt pyxplot\_watch $\backslash$*.script} will behave differently in most
UNIX shells.  In the first case, the wildcard is expanded by your shell, and a
list of files passed to {\tt pyxplot\_watch}. Any files matching the
wildcard, created after running {\tt pyxplot\_watch}, will not be picked up.
In the latter case, the wildcard is expanded by {\tt pyxplot\_watch} itself,
which {\it will} pick up any newly created files.}

This is especially useful when combined with \ghostview's\index{Ghostview}
watch facility. For example, suppose that a script {\tt foo.ppl} produces
postscript output {\tt foo.ps}. The following two commands could be used to
give a live view of the result of executing this script:

\begin{verbatim}
gv --watch foo.ps &
pyxplot_watch foo.ppl
\end{verbatim}

