% DATA.TEX
%
% The documentation in this file is part of PyXPlot
% <http://www.pyxplot.org.uk>
%
% Copyright (C) 2006-9 Dominic Ford <coders@pyxplot.org.uk>
%               2009   Ross Church
%
% $Id$
%
% PyXPlot is free software; you can redistribute it and/or modify it under the
% terms of the GNU General Public License as published by the Free Software
% Foundation; either version 2 of the License, or (at your option) any later
% version.
%
% You should have received a copy of the GNU General Public License along with
% PyXPlot; if not, write to the Free Software Foundation, Inc., 51 Franklin
% Street, Fifth Floor, Boston, MA  02110-1301, USA

% ----------------------------------------------------------------------------

% LaTeX source for the PyXPlot Users' Guide

\chapter{Working with Data}
\label{ch:numerics}

In this chapter, we describe more sophisticated ways of dealing with data files
and outline the facilities provided for data processing within PyXPlot.

\section{Input filters}

\section{Reading data from a pipe}

PyXPlot usually reads data from a file, but it possible to read data via a pipe
from standard input.  To do this one uses the magic filename `{\tt -}':

\begin{verbatim}
plot '-' with lines
\end{verbatim}

This makes it possible to call PyXPlot from within a program and to pass it
some data to plot without ever storing the data on disk.  However, we would
generally discourage you from doing this. Unless you really {\it need} to do
otherwise, we would recommend that you write your data to a file on disk, and
then to plot it from inside PyXPlot in a separate step.  Whilst it is often
tempting to write programs which both perform calculations and plot the results
immediately, this can be a dangerous path. A few months after the event, when
the need arises to replot the same data in a different form or in a different
style, remembering how to use a sizeable program can prove tricky -- especially
if the person struggling to do so is not you! But a simple \datafile\ is quite
straightforward to plot time and again.

\section{Including data within PyXPlot command scripts}

\section{Tabulating Functions and Slicing Data Files}

PyXPlot's \indcmdt{tabulate} can be used to produce a text file containing the
values of a function at a set of points.  For example, the following would
produce a data file called {\tt sine.dat} containing a list of values of the
sine function:

\begin{verbatim}
set output 'sine.dat'
tabulate [-pi:pi] sin(x)
\end{verbatim}

\noindent Multiple functions may be tabulated into the same file, either by
using the \indmodt{using} modifier:

\begin{verbatim}
tabulate [0:2*pi] sin(x):cos(x):tan(x) u 1:2:3:4
\end{verbatim}

\noindent or by placing them in a comma-separated list, as in the {\tt plot}
command:

\begin{verbatim}
tabulate [0:2*pi] sin(x), cos(x), tan(x)
\end{verbatim}

The {\tt samples} setting can be used to control the number of points that are
inserted into the data file:\indcmd{set samples}

\begin{verbatim}
set samples 200
\end{verbatim}

\noindent If the $x$-axis is set to be logarithmic then the points at which the
functions are evaluated are spaced logarithmically.

The {\tt tabulate} command can also be used to select portions of data files
for output into a new file.  For example, the following would write out the
third, sixth and ninth columns of the datafile {\tt input.dat}, but only when
the arcsine of the value in the fourth column is positive:

\begin{verbatim}
set output 'filtered.dat'
tabulate 'input.dat' u 3:6:9 select (asin($4)>0)
\end{verbatim}

\noindent The \indmodt{select}, \indmodt{using} and \indmodt{every} modifiers
operate in the same manner as with the {\tt plot} command.

The format used in each column of the output file is chosen automatically with
integers and small numbers treated intelligently to produce output which
preserves accuracy, but is also easily human-readable. If desired, however, a
format statement may be specified using the {\tt with format} specifier. The
syntax for this is similar to that expected by the Python string substitution
operator ({\tt \%})\index{\% operator@{\tt \%} operator}\index{string
operators!substitution}\footnote{Note that this operator can also be used
within PyXPlot; see Section~\ref{sec:string_subs_op} for details.}.  For example,
to tabulate the values of $x^2$ to very many significant figures one could use:

\begin{verbatim}
tabulate x**2 with format "%27.20e"
\end{verbatim}

If there are not enough columns present in the supplied format statement it
will be repeated in a cyclic fashion; e.g. in the example above the single
supplied format is used for both columns.

\section{Function Fitting}
\label{sec:fit_command}

It is possible to fit functional forms to \datapoint s read from files by using
the \indcmdt{fit}. A simple example might be:\footnote{In \gnuplot, this
example would have been written {\tt fit f(x) ...}, rather than {\tt fit f()
...}. This syntax is supported in PyXPlot, but is deprecated.}

\begin{verbatim}
f(x) = a*x+b
fit f() 'data.dat' index 1 using 2:3 via a,b
\end{verbatim}

The first line specifies the functional form which is to be used.  The
coefficients within this function which are to be varied during the fitting
process are listed after the keyword \indkeyt{via} in the {\tt fit} command.
The modifiers \indmodt{index}, \indmodt{every}, \indmodt{select} and
\indmodt{using} have the same meanings here as in the plot command.  For
example, given the following data file which contains a sampled square wave,
entitled {\tt square.dat}:

\begin{verbatim}
    0.314159          1
    0.942478          1
    1.570796          1
    2.199115          1
    2.827433          1
    3.455752         -1
    4.084070         -1
    4.712389         -1
    5.340708         -1
    5.969026         -1
\end{verbatim}

\noindent the following script fits a truncated Fourier series to it.  The
output can be found in Example~\ref{ex:fitting}.

\begin{verbatim}
f(x) = a1*sin(x) + a3*sin(3*x) + a5*sin(5*x)
fit f() 'square.dat' via a1, a3, a5
set xlabel '$x$' ; set ylabel '$y$'
plot 'square.dat' title 'data' with points pointsize 2, \
     f(x) title 'Fitted function' with lines
\end{verbatim}

\example{ex:fitting}{The output from a script that fits a truncated Fourier series to a sampled square wave}{
The output from a script that fits a truncated Fourier series to a
sampled square wave.  Even with only three terms the Gibbs pheonomenon is
becoming apparent (see \url{http://en.wikipedia.org/wiki/Gibbs_phenomenon} for
an explanation).\\
\centerline{\includegraphics{examples/eps/ex_fitting.eps}}
}

This is useful for producing best-fit lines\index{best fit
lines}\footnote{Another way of producing best-fit lines is to use a cubic
spline; more details are given in Section~\ref{sec:spline_command}}, and also has
applications for estimating the gradients of datasets.  The syntax is
essentially identical to that used by \gnuplot, though a few points are worth
noting:

\begin{itemize}
\item When fitting a function of $n$ variables, at least $n+1$ columns (or
rows -- see Section~\ref{sec:horizontal_datafiles}) must be specified after the
{\tt using} modifier. By default, the first $n+1$ columns are used. These
correspond to the values of each of the $n$ inputs to the function, plus
finally the value which the output from the function is aiming to match.
\item If an additional column is specified, then this is taken to contain the
standard error in the value that the output from the function is aiming to
match, and can be used to weight the \datapoint s which are input into the
{\tt fit} command.
\item By default, the starting values for each of the fitting parameters is
$1.0$. However, if the variables to be used in the fitting process are already
set before the {\tt fit} command is called, these initial values are used
instead. For example, the following would use the initial values
$\{a=100,b=50\}$:
\begin{verbatim}
f(x) = a*x+b
a = 100
b = 50
fit f() 'data.dat' index 1 using 2:3 via a,b
\end{verbatim}

\item As with all numerical fitting procedures, the {\tt fit} command comes
with caveats. It uses a generic fitting algorithm, and may not work well with
poorly behaved or ill-constrained problems. It works best when all of the
values it is attempting to fit are of order unity. For example, in a problem
where $a$ was of order $10^{10}$, the following might fail:
\begin{verbatim}
f(x) = a*x
fit f() 'data.dat' via a
\end{verbatim}
However, better results might be achieved if $a$ were artificially made of
order unity, as in the following script:
\begin{verbatim}
f(x) = 1e10*a*x
fit f() 'data.dat' via a
\end{verbatim}

\item A series of ranges may be specified after the {\tt fit} command, using
the same syntax as in the {\tt plot} command, as described in
Section~\ref{sec:plot_ranges}. If ranges are specified then only \datapoint s
falling within these ranges are used in the fitting process; the ranges refer
to each of the $n$ variables of the fitted function in order.

\item For those interested in the mathematical details, the workings of the
{\tt fit} command is discussed in more detail in Appendix~\ref{ch:fit_maths}.

\end{itemize}

At the end of the fitting process, the best-fitting values of each parameter
are output to the terminal, along with an estimate of the uncertainty in each.
Additionally, the Hessian, covariance and correlation matrices are output in
both human-readable and machine-readable formats, allowing a more complete
assessment of the probability distribution of the parameters.

\section{Datafile Interpolation}
\label{sec:spline_command}
\index{best fit lines}

The \indcmdt{interpolate} can be used to generate a special function within
PyXPlot which interpolates a set of \datapoint s supplied from a data file.
Various different types of interpolation are supported: linear interpolation,
power-law interpolation\footnote{Power-law interpolation is equivalent to
linear interpolation in log-log space.}, polynomial interpolation and
cubic-spline interpolation. The use of polynomial interpolation with large
datasets is strongly discouraged, as polynomial fits tend to show severe
oscillations between \datapoint s.

The \indcmdt{interpolate} has similar syntax to the \indcmdt{fit}:

\begin{verbatim}
interpolate ( linear | loglinear | polynomial | spline )
            [<range specification>] <function name>() '<filename>' 
            [index <index specification>]
            [every <every specification>]
            [using <using specification>]
\end{verbatim}

A very common application of the \indcmdt{interpolate} is to perform arithmetic
functions such as addition or subtraction on datasets which are not sampled at
the same abscissa values. The following example would plot the difference
between two such datasets:

\begin{verbatim}
interpolate linear f() 'data1.dat'
interpolate linear g() 'data2.dat'
plot [min:max] f(x)-g(x)
\end{verbatim}

\noindent Note that it is advisable to supply a range to the {\tt plot} command
in this example: because the two datasets have been turned into continuous
functions, the {\tt plot} command has to guess a range to plot them over unless
one is explicitly supplied.

The functions have not, however, lost the range of the original data: because
it is generally very inadvisable to extrapolate interpolated function fits
beyond the limits of the \datapoint s used to constain the fit, PyXPlot returns
an error or NaN if this is attempted.

Smoothed splines can also be produced:

\begin{verbatim}
interpolate spline f() 'data1.dat' smooth 1.0
\end{verbatim}

\noindent where the value $1.0$ determines the degree of smoothing to apply;
the higher the value, the more smoothing is applied. The default behaviour is
not to smooth at all -- equivalent to {\tt smooth 0.0} -- and a value of $1.0$
corresponds to a moderate degree of smoothing.

The \indcmdt{spline} command is an alias for {\tt interpolate spline}; the
following two statements are equivalent:

\begin{verbatim}
spline f() 'data1.dat'
interpolate spline f() 'data1.dat'
\end{verbatim}

\section{Fourier Transforms}

\section{Histograms}

The \indcmdt{histogram} takes data from a file and bins it, producing a
function that represents the frequency distribution of the supplied data.  A
histogram is defined as a function consisting of discrete intervals, the area
under each of which is equal to the number of points binned in that interval.
For example:

\begin{verbatim}
histogram f() 'input.dat'
\end{verbatim}

\noindent bins the points in the first column of the file {\tt input.dat}
into bins of unit width and produces a function $f()$, the value of which at any
given point is equal to the number of items in the bin at that point.

Modifiers can be supplied to the \indcmdt{histogram} command to control the bins
that it uses.  The \indmodt{binwidth} modifier sets the width of the bins used
and the \indmodt{binorigin} modifier their origin.  For example:

\begin{verbatim}
histogram wabbitcount() 'rabbits.dat' binorigin 0.5 binwidth 2
\end{verbatim}

\noindent bins the rabbit data into bins between $0.5$ and $2.5$, $2.5$ and
$4.5$, etc.  Alternatively the \indmodt{bins} modifier allows an arbitrary set
of bins to be specified. For example the command:

\begin{verbatim}
histogram g() 'input.dat' bins (1, 2, 4)
\end{verbatim}

\noindent would bin the points in the first column of the file {\tt input.dat}
into two bins, $x=1\to 2$ and $x=2\to 4$.

A range can be supplied immediately following the command, using the same
syntax as in the {\tt plot} and {\tt fit} commands; only points that fall in
that range will be binned.  In the same way as for the {\tt plot} command,
the \indmodt{index}, \indmodt{every}, \indmodt{using} and \indmodt{select}
modifiers can also be used to bin different portions of a datafile.

Note that, although a histogram is similar to a bar chart, they are subtly
different.  A bar chart has the {\it height} of the bar equal to the number of
points that it represents; for a histogram the {\it area} of the bar is equal to
the number of points.  To produce a bar chart use the histogram
command and then multiply by the bin width when plotting.

If the function produced by the histogram command is plotted using the
\indpst{boxes} plot style, box boundaries will be drawn to coincide with the
bins into which the data were sorted.

\section{Random Data Generation}

