% CHANGELOG.TEX
%
% The documentation in this file is part of PyXPlot
% <http://www.pyxplot.org.uk>
%
% Copyright (C) 2006-2011 Dominic Ford <coders@pyxplot.org.uk>
%               2008-2011 Ross Church
%
% $Id$
%
% PyXPlot is free software; you can redistribute it and/or modify it under the
% terms of the GNU General Public License as published by the Free Software
% Foundation; either version 2 of the License, or (at your option) any later
% version.
%
% You should have received a copy of the GNU General Public License along with
% PyXPlot; if not, write to the Free Software Foundation, Inc., 51 Franklin
% Street, Fifth Floor, Boston, MA  02110-1301, USA

% ----------------------------------------------------------------------------

% LaTeX source for the PyXPlot Users' Guide

\chapter{ChangeLog}
\index{ChangeLog}

\subsection*{2011 Jan 7: PyXPlot 0.8.4}

\subsubsection*{Summary:}

This is a minor bugfix release.

\subsubsection*{Details:}

\begin{itemize}
\item Two-dimensional parametric grid plotting implemented.
\item Bugfix to the dots plot style; filled triangles replaces with filled circles.
\item Bugfix to linewidths used when drawing line icons on graph legends.
\item Bugfix to Makefile to ensure libraries link correctly under Red Hat and SUSE.
\item Code cleanup to ensure correct compilation with {\tt -O2} optimisation.
\end{itemize}

\subsection*{2010 Sep 15: PyXPlot 0.8.3}

\subsubsection*{Summary:}

This is a minor bugfix release.

\subsubsection*{Details:}

\begin{itemize}
\item @ macro expansion operator implemented.
\item assert command implemented.
\item for command behaviour changed such that {\tt for i=1 to 10} includes a final iteration with {\tt i}=10.
\item Point types rearranged into a more logical order.
\item Improved support for newer Windows bitmap images.
\item Bugfix to the {\tt set unit preferred} command.
\item Binary not operator bugfixed.
\item Bugfix to handling of comma-separated horizontal datafiles.
\item Mathematical function {\tt finite()} added.
\end{itemize}

\subsection*{2010 Aug 4: PyXPlot 0.8.2}

\subsubsection*{Summary:}

This release introduces three-dimensional plotting, as well as the ability to plot two-dimensional maps of functions as either colour maps, contour plots, or as three-dimensional surfaces. A large number of bugs have also been fixed.

\subsubsection*{Details:}

\begin{itemize}
\item 3D plotting implemented.
\item New plot styles colourmap, contourmap and surface added.
\item Interpolation of 2D datagrids and bitmap images implemented.
\item Stepwise interpolation mode added.
\item Dependency on libkpathsea relaxed to make installation under MacOS easier; linking to the library is still strongly recommended on systems where it is readily available.
\item Mathematical functions {\tt frac-tal\_\-julia()}, {\tt frac\-tal\_\-man\-del\-brot()} and {\tt prime()} added.
\item Many bug fixes, especially to the ticking of axes.
\end{itemize}

\subsection*{2010 Jun 1: PyXPlot 0.8.1}

\subsubsection*{Summary:}

This release has no major new features, but fixes several significant bugs in version 0.8.0.

\subsubsection*{Details:}

\begin{itemize}
\item Mathematical functions {\tt time\_\-from\-unix()}, {\tt time\_\-unix()}, {\tt zernike()} and {\tt zernikeR()} added.
\item Bug fix to the ticking of linked axes.
\item Bug fix to the ticking of axes with blank axis tick labels.
\item Makefile and configure script improved for portability.
\end{itemize}

\subsection*{2010 May 19: PyXPlot 0.8.0}

\subsubsection*{Summary:}

This release is a major update, for which PyXPlot's original python code has
been completely rewritten in C with the addition of many new features. Because
of the scale of this update, there is some minor syntax incompatibility with
previous versions where features have undergone particularly heavy change. The
most apparent change is the increase in speed and efficiency resultant from the
use of a compiled language: especially when handling large \datafile s, PyXPlot
0.8.0 can run more than an order-of-magnitude faster than previous versions.

\subsubsection*{Details:}

\begin{itemize}
\item The handling of large \datafile s has been streamlined to require around an order-of-magnitude less time and memory.
\item PyXPlot's mathematical environment has been extended to handle complex numbers and quantities with physical units.
\item The range of mathematical functions built into PyXPlot has been massively extended.
\item The {\tt solve} command has been added to allow the solution of systems of equations.
\item The {\tt maximise} and {\tt minimise} commands have been added to allow searches for local extrema of functions.
\item An {\tt fft} command has been added for performing Fourier transforms on data.
\item New plot styles -- {\tt filledregion} and {\tt yerrorshaded} -- have been added for plotting filled error regions.
\item The configuration of linked axes has been entirely redesigned.
\item Parametric function plotting has been implemented.
\item Colours can now be specified by RGB, HSB or CMYK components, as well as by name.
\item Several commands, e.g. {\tt box}, {\tt circle}, {\tt ellipse}, etc., have been added to allow vector graphics to be produced in PyXPlot's multiplot environment.
\item The {\tt jpeg} command has been generalised to allow the incorporation of not only {\tt jpeg} images, but also {\tt bmp}, {\tt gif} and {\tt png} images, onto multiplot canvases. The command has been renamed {\tt image} in recognition of its wider applicability. Image transparency is now supported in {\tt gif} and {\tt png} images.
\item The {\tt spline} command, now renamed the {\tt interpolate} command, has been extended up provide many types of interpolation between datapoints.
\item A wide range of conditional and flow control structures have been added to PyXPlot's command language -- these are the {\tt do}, {\tt for}, {\tt foreach}, {\tt if} and {\tt while} commands and the {\tt cond\-ition\-alS} and {\tt con\-dition\-alN} mathematical functions.
\item Input filters have been introduced as a mechanism by which datafiles in arbitrary formats can be read.
\item PyXPlot's commandline interface now supports tab completion.
\item The {\tt show} command has been reworked to produce pastable output.
\item Many minor bugs have been fixed.
\end{itemize}

\subsection*{2009 Nov 17: PyXPlot 0.7.1}

\subsubsection*{Summary:}

This release has no major new features, but fixes several serious bugs in version 0.7.0.

\subsubsection*{Details:}

\begin{itemize}
\item The {\tt exec} command did not work in PyXPlot 0.7.0; this issue has been resolved.
\item The {\tt xyerrorrange} plot style did not work in PyXPlot 0.7.0; this issue has been resolved.
\item PyXPlot 0.7.0 produces large numbers of python deprecation error messages when run under python 2.6; the code has been updated to remove references to deprecated python functions.
\end{itemize}

\subsubsection*{Details -- Change of System Requirements:}

\begin{itemize}
\item In order to fix some of the bugs listed above, it has been necessary to
fix bugs in the PyX graphics library as well as those in PyXPlot. As a result,
and to ensure that these bugfixes reach users as quickly as possible, we have
opted to ship our own modified version of PyX 0.10, called dcfPyX with PyXPlot.
\end{itemize}

\subsection*{2008 Oct 14: PyXPlot 0.7.0}

\subsubsection*{Summary:}

Third PyXPlot beta-release. The code has undergone significant streamlining,
and now runs approximately twice as fast as version 0.6.3 when handling large
datafiles. Memory usage has also been radically reduced. Two new data
processing commands have been introduced. The {\tt tabulate} command can be
used to produce textual datafiles, allowing the user to read data in from
files, apply some analysis, and then write the processed data back to file. The
{\tt histogram} command can be used to estimate the frequency densities of sets
of data points, either by binning them into a bar chart, or by fitting a
functional form to their frequency density.

\subsubsection*{Details -- New and Extended Commands:}

\begin{itemize}
\item {\tt tabulate}
\item {\tt histogram}
\item {\tt set label} and {\tt text} commands extended to allow a colour to be
specified.
\end{itemize}

\subsubsection*{Details -- API changes}

\begin{itemize}
\item {\tt diff\_dx()} and {\tt int\_dx()} functions -- the function to be
differentiated or integrated must now be placed in quotation marks.
\end{itemize}

\subsubsection*{Details -- Change of System Requirements:}

\begin{itemize}
\item Requirement of PyX version 0.9 has been updated to PyX version 0.10. Note that new versions of the PyX graphics library are not generally backwardly compatible.
\end{itemize}

\subsection*{2007 Feb 26: PyXPlot 0.6.3}

\subsubsection*{Summary:}

Second PyXPlot beta-release. The most significant change is the introduction of
a new command-line parser, with greatly improved handling of complex
expressions and much more meaningful syntax error messages. Multi-platform
compatibility has also been massively improved, and dependencies loosened.  A
small number of new commands have been added; most notable among them are the
{\tt jpeg} and {\tt eps} commands, which embed images in multiplots.

\subsubsection*{Details -- New and Extended Commands:}

\begin{itemize}
\item {\tt jpeg}
\item {\tt eps}
\item {\tt set xtics} and {\tt set mxtics}
\item {\tt text} and {\tt set label} commands extended to allow text rotation.
\item {\tt set log} command extended to allow the use of logarithms with bases other than 10.
\item {\tt set preamble}
\item {\tt set term enlarge | noenlarge}
\item {\tt set term pdf}
\item {\tt set term x11\_persist}
\end{itemize}

\subsubsection*{Details -- Eased System Requirements:}

\begin{itemize}
\item Requirement on Python 2.4 minimum eased to version 2.3 minimum.
\item Requirements on scipy and readline eased; PyXPlot will now work in reduced form when they are absent, though they are still strongly recommended.
\item dvips and Ghostscript are no longer required.
\end{itemize}

\subsubsection*{Details -- Removed Commands:}

Due to a general refinement of PyXPlot's API, some of the less sensible pieces
of syntax from Version~0.5 are no longer supported. The author apologises for
any inconvenience caused.

\begin{itemize}
\item The {\tt delete\_arrow}, {\tt delete\_text}, {\tt move\_text}, {\tt undelete\_arrow} and {\tt undelete\_text} commands have been removed from the PyXPlot API. The {\tt move}, {\tt delete} and {\tt undelete} commands should now be used to act upon all types of multiplot objects.
\item The {\tt set terminal} command no longer accepts the {\tt enhanced} and {\tt noenhanced} modifiers. The {\tt postscript} and {\tt eps} terminals should be used instead.
\item The {\tt select} modifier, used after the {\tt plot}, {\tt replot}, {\tt fit} and {\tt spline} command can now only be used once; to specify multiple {\tt select} criteria, use the {\tt and} logical operator.
\end{itemize}

\subsection*{2006 Sep 09: PyXPlot 0.5.8}

First beta-release.
