% REFERENCE.TEX
%
% The documentation in this file is part of PyXPlot
% <http://www.pyxplot.org.uk>
%
% Copyright (C) 2006-2010 Dominic Ford <coders@pyxplot.org.uk>
%               2008-2010 Ross Church
%
% $Id$
%
% PyXPlot is free software; you can redistribute it and/or modify it under the
% terms of the GNU General Public License as published by the Free Software
% Foundation; either version 2 of the License, or (at your option) any later
% version.
%
% You should have received a copy of the GNU General Public License along with
% PyXPlot; if not, write to the Free Software Foundation, Inc., 51 Franklin
% Street, Fifth Floor, Boston, MA  02110-1301, USA

% ----------------------------------------------------------------------------

% LaTeX source for the PyXPlot Users' Guide

\chapter{Command Reference}
\label{ch:reference}

This chapter contains an alphabetically ordered list of all of PyXPlot's
commands. The syntax of each is specified in a variant of Backus-Naur notation,
in which angle brackets {\tt <>} are used to indicate replaceable tokens,
parentheses {\tt ()} are used to indicate mutually-exclusive options which are
separated by vertical lines {\tt |}, square brackets {\tt []} are used to
indicate optional items, and braces {\tt \{\}} are used to indicate items which
may be repeated. Dots {\tt ...} are used to indicate arbitrary strings of text.
Where any of these punctuation marks appear as a part of PyXPlot's syntax, they
are placed in double quotes, as in {\tt "\{"}. Strings, such as filenames,
which should be placed in quotes are shown in single quotes, as in {\tt
'<filename>'}, though in practice, double and single quotes can always be used
interchangeably. In cases where there is likely to be any ambiguity, worked
examples are usually shown beneath the syntax specification.

\section{?}\indcmd{?}

\begin{verbatim}
? [<topic> {<sub-topic>} ]
\end{verbatim}

The {\tt ?} symbol is a shortcut to the {\tt help} command.


\section{!}\indcmd{!}

\begin{verbatim}
! <shell command>
... `<shell command>` ...
\end{verbatim}

Shell commands can be executed within PyXPlot by prefixing them with
pling (!) characters, as in the example:

\begin{verbatim}
!mkdir foo
\end{verbatim}

\noindent As an alternative, back-quotes (`) can be used to substitute the
output of a shell command into a PyXPlot command, as in the example:

\begin{verbatim}
set xlabel `echo "'" ; ls ; echo "'"`
\end{verbatim}

\noindent Note that back-quotes cannot be used inside quote characters, and so
the following would \textit{not} work:

\begin{verbatim}
set xlabel '`ls`'
\end{verbatim}


\section{arc}\indcmd{arc}

\begin{verbatim}
arc [ item <id> ] [at] <x>, <y> radius <r>
    from <start> to <finish> [ with {<option>} ]
\end{verbatim}

Arcs (curves with constant radius of curvature, that is, segments of circles)
may be drawn on multiplot canvases using the \indcmdt{arc}.  The {\tt at}
modifier specifies the coordinates of the centre of curvature, from which all
points on the arc are at the distance given following the {\tt radius} modifier.
The angles {\tt start} and {\tt finish}, measured clockwise from the vertical,
control where the arc begins and ends.  For example, the command

\begin{verbatim}
arc at 0,0 radius 2 from 90 to 270
\end{verbatim}

\noindent would draw a semi-circle beneath the line $x=0$, centred on the
origin with radius $2\,\mathrm{cm}$.  The usual style modifiers for lines may
be passed after the keyword {\tt with}; if the {\tt fillcolour} modifier is
specified then the arc will be filled to form a pie-chart slice.

All vector graphics objects placed on multiplot canvases receive unique
identification numbers which count sequentially from one, and which may be
listed using the {\tt list} command.  By reference to these numbers, they can
be deleted and subsequently restored with the {\tt delete} and {\tt undelete}
commands respectively.


\section{arrow}\indcmd{arrow}

\begin{verbatim}
arrow [ item <id> ] [from] <x>, <y> [to] <x>, <y>
                  [ with {<option>} ]
\end{verbatim}

Arrows may be drawn on multiplot canvases using the \indcmdt{arrow}. The style
of the arrows produced may be specified by following the \indmodt{with}
modifier by one of the style keywords \indkeyt{nohead}, \indkeyt{head}
(default) or \indkeyt{twohead}. In addition, keywords such as \indkeyt{colour},
\indkeyt{linewidth} and \indkeyt{linetype} have the same syntax and meaning
following the keyword \indmodt{with} as in the {\tt plot} command. The
following example would draw a bidirectional blue arrow:

\begin{verbatim}
arrow from x1,y1 to x2,y2 with twohead linetype 2 colour blue
\end{verbatim}

The {\tt arrow} command has a twin, the {\tt line} command, which has the same
syntax, but uses the default arrow style of \indkeyt{nohead}, producing short
line segments.

All vector graphics objects placed on multiplot canvases receive unique
identification numbers which count sequentially from one, and which may be
listed using the {\tt list} command.  By reference to these numbers, they can
be deleted and subsequently restored with the {\tt delete} and {\tt undelete}
commands respectively.


\section{box}\indcmd{box}

\begin{verbatim}
box [ item <id> ] at <x>, <y> width <w> height <h>
         [ rotate <r> ] [ with {<option>} ]

box [ item <id> ] from <x1>, <y1> to <x2>, <y2>
         [ rotate <r> ] [ with {<option>} ]
\end{verbatim}

The \indcmdt{box} is used to draw and fill rectangular boxes on multiplot
canvases.  The position of each box may be specified in one of two ways.  In the
first, the coordinates of one corner of the box are specified, along with its
width and height. If both the width and the height are positive then the
coordinates are taken to be those of the bottom left-hand corner of the box;
other corners may be specified if the supplied width and/or height are
negative. If a rotation angle is specified then the box is rotated about the
specified corner.  The {\tt with} modifier allows the style of the box to be
specified using similar options to those accepted by the {\tt plot} command.

The second syntax allows two pairs of coordinates to be specified.  PyXPlot
will then draw a rectangular box with opposing corners at the specified
locations.  If an angle is specified the box will be rotated about its centre.
Hence the following two commands both draw a square box centred on the origin:

\begin{verbatim}
box from -1, -1 to 1,1
box at 1, -1 width -2 height 2
\end{verbatim}

All vector graphics objects placed on multiplot canvases receive unique
identification numbers which count sequentially from one, and which may be
listed using the {\tt list} command.  By reference to these numbers, they can
be deleted and subsequently restored with the {\tt delete} and {\tt undelete}
commands respectively.


\section{break}\indcmd{break}

\begin{verbatim}
break [<loopname>]
\end{verbatim}

The \indcmdt{break} terminates execution of {\tt do}, {\tt while}, {\tt for}
and {\tt foreach} loops in an analogous manner to the {\tt break} statement in
the C programming language.  Execution resumes at the statement following the
end of the loop. For example, the following loop would only print the numbers~1
and~2:

\begin{verbatim}
for i = 1 to 10
 {
  print i
  if (i==2)
   {
    break
   }
 }
\end{verbatim}

If several loops are nested, the {\tt break} statement only acts on the
innermost loop. If the {\tt break} statement is encountered outside of any loop
structure, an error results. Optionally, the {\tt for}, {\tt foreach}, {\tt do}
and {\tt while} commands may be supplied with a name for the loop, prefixed by
the word {\tt loopname}, as in the examples:

\begin{verbatim}
for i=0 to 4 loopname iloop

foreach i in "*.dat" loopname DatafileLoop
\end{verbatim}

\noindent When loops are given such names, the {\tt break} statement may be
followed by the name of the loop whose iteration is to be broken, allowing it
to act upon loops other than the innermost one.

See also the {\tt continue} command.


\section{cd}\indcmd{cd}

\begin{verbatim}
cd <directory>
\end{verbatim}

PyXPlot's \indcmdt{cd} is very similar to the shell {\tt cd} command; it
changes the current working directory. The following example would enter the
subdirectory {\tt foo}:

\begin{verbatim}
cd foo
\end{verbatim}


\section{circle}\indcmd{circle}

\begin{verbatim}
circle [ item <id> ] [at] <x>, <y> radius <r>
       [ with {<option>} ]
\end{verbatim}

The \indcmdt{circle} is used to draw circles on multiplot canvases.  The
coordinates of the circle's centre and its radius are specified. The {\tt with}
modifier allows the style of the circle to be specified using similar options
to those accepted by the {\tt plot} command.  The example

\begin{verbatim}
circle at 2,2 radius 1 with colour red fillcolour blue
\end{verbatim}

\noindent would draw a red circle of unit radius filled in blue, centred
$2\,\mathrm{cm}$ above and to the right of the origin.

All vector graphics objects placed on multiplot canvases receive unique
identification numbers which count sequentially from one, and which may be
listed using the {\tt list} command.  By reference to these numbers, they can
be deleted and subsequently restored with the {\tt delete} and {\tt undelete}
commands respectively.


\section{clear}\indcmd{clear}

\begin{verbatim}
clear
\end{verbatim}

In multiplot mode, the \indcmdt{clear} removes all plots, arrows and text
objects from the working multiplot canvas. Outside of multiplot mode, it is not
especially useful; it removes the current plot to leave a blank canvas.  The
{\tt clear} command should not be followed by any parameters.


\section{continue}\indcmd{continue}

\begin{verbatim}
continue [<loopname>]
\end{verbatim}

The \indcmdt{continue} terminates execution of the current iteration of {\tt
for}, {\tt foreach}, {\tt do} and {\tt while} loops in an analogous manner to
the {\tt continue} statement in the C programming language. Execution resumes
at the first statement of the next iteration of the loop, or at the first
statement following the end of the loop in the case of the last iteration of
the loop.  For example, the following script will not print the number~2:

\begin{verbatim}
for i = 0 to 5
 {
  if (i==2)
   {
    continue
   }
  print i
 }
\end{verbatim}

If several loops are nested, the {\tt continue} statement only
acts on the innermost loop. If the {\tt continue} statement is encountered outside of any
loop structure, an error results. Optionally, the {\tt for}, {\tt foreach},
{\tt do} and {\tt while} statements may be supplied with a name for the loop, prefixed by
the word {\tt loopname}, as in the examples:

\begin{verbatim}
for i=0 to 4 loopname iloop

foreach i in "*.dat" loopname DatafileLoop
\end{verbatim}

\noindent When loops are given such names, the {\tt continue} statement may be
followed by the name of the loop whose iteration is to be broken, allowing it
to act upon loops other than the innermost one.

See also the {\tt break} command.


\section{delete}\indcmd{delete}

\begin{verbatim}
delete <item number> {, <item number>}
\end{verbatim}

The \indcmdt{delete} removes vector graphics objects such as plots, arrows or
text items from the current multiplot canvas. All vector graphics objects
placed on multiplot canvases receive unique identification numbers which count
sequentially from one, and which may be listed using the {\tt list} command.
The items to be deleted should be identified using a comma-separated list of
their identification numbers. The example

\begin{verbatim}
delete 1,2,3
\end{verbatim}

\noindent would remove item numbers~1,~2 and~3.

Having been deleted, multiplot items can be restored using the {\tt undelete}
command.


\section{do}\indcmd{do}

\begin{verbatim}
do [loopname <loopname>]
 "{"
  ...
 "}" while <condition>
\end{verbatim}

The \indcmdt{do} executes a block of commands repeatedly, checking the
condition given in the {\tt while} clause at the end of each iteration.  If the
condition is true then the loop executes again. This is similar to a {\tt
while} loop, except that the contents of a {\tt do} loop are {\emph always}
executed at least once.  The following example prints the numbers~1, 2 and~3:

\begin{verbatim}
i=1
do
 {
  print i
  i = i + 1
 } while (i < 4)
\end{verbatim}

\noindent Note that there must always be a newline following the opening brace
after the \indcmdt{do}, and the while clause must always be on the same line as
the closing brace.


\section{ellipse}\indcmd{ellipse}

Ellipses may be drawn on multiplot canvases using the \indcmdt{ellipse}. The shape
of the ellipse may be specified in many different ways, by specifying

\begin{enumerate}[(i)]
\item the position of two corners of the smallest rectangle which can enclose
the ellipse when its major axis is horizontal, together with an optional
counter-clockwise rotation angle, applied about the centre of the ellipse.
For example:

\begin{verbatim}
ellipse from 0,0 to 4,1 rot 70
\end{verbatim}

\item the position of both the centre and one of the foci of the ellipse,
together with any one of the following additional pieces of information: the
ellipse's major axis length, its semi-major axis length, its minor axis length,
its semi-minor axis length, its eccentricity, its latus rectum, or its
semi-latus rectum.  For example:

\begin{verbatim}
ellipse focus 0,0 centre 2,2 majoraxis 4
ellipse focus 0,0 centre 2,2 minoraxis 4
ellipse focus 0,0 centre 2,2 ecc 0.5
ellipse focus 0,0 centre 2,2 LatusRectum 6
ellipse focus 0,0 centre 2,2 slr 3
\end{verbatim}

\item the position of either the centre or one of the foci of the ellipse,
together with any two of the following additional pieces of information: the
ellipse's major axis length, its semi-major axis length, its minor axis length,
its semi-minor axis length, its eccentricity, its latus rectum, or its
semi-latus rectum. An optional counter-clockwise rotation angle may also be
specified, applied about either the centre or one of the foci of the ellipse,
whichever is specified. If no rotation angle is given, then the major axis of
the ellipse is horizontal.  For example:

\begin{verbatim}
ellipse centre 0,0 majoraxis 4 minoraxis 4
\end{verbatim}
\end{enumerate}

The line type, line width, and colour of line with which the outlines of
ellipses are drawn may be specified after the keyword {\tt with}, as in the
{\tt box} and {\tt circle} commands above. Likewise, ellipses may be filled in
the same manner.

All vector graphics objects placed on multiplot canvases receive unique
identification numbers which count sequentially from one, and which may be
listed using the {\tt list} command.  By reference to these numbers, they can
be deleted and subsequently restored with the {\tt delete} and {\tt undelete}
commands respectively.


\section{else}\indcmd{else}

The {\tt else} statement is described in the entry for the {\tt if}
statement, of which it forms part.


\section{eps}\indcmd{eps}

\begin{verbatim}
eps [ item <id> ] '<filename>' [at <x>, <y>] [rotate <angle>]
                            [width <width>] [height <height>]
\end{verbatim}

The \indcmdt{eps} allows Encapsulated PostScript (EPS) images to be inserted
onto multiplot canvases.  The {\tt at} modifier can be used to specify where
the bottom-left corner of the image should be placed; if it is not, then the
image is placed at the origin. The {\tt rotate} modifier can be used to rotate
the image by any angle, measured in degrees counter-clockwise.  The {\tt width}
or {\tt height} modifiers can be used to specify the width or height with which
the image should be rendered; both should be specified in centimetres. If
neither is specified then the image will be rendered with the native dimensions
specified within the PostScript.  The {\tt eps} command is often useful in
multiplot mode, allowing PostScript images to be combined with plots, text
labels, etc.

All vector graphics objects placed on multiplot canvases receive unique
identification numbers which count sequentially from one, and which may be
listed using the {\tt list} command.  By reference to these numbers, they can
be deleted and subsequently restored with the {\tt delete} and {\tt undelete}
commands respectively.


\section{exec}\indcmd{exec}

\begin{verbatim}
exec <command>
\end{verbatim}

The \indcmdt{exec} can be used to execute PyXPlot commands contained within
string variables, as in the following example:

\begin{verbatim}
terminal="eps"
exec "set terminal %s"%(terminal)
\end{verbatim}

\noindent It can also be used to write obfuscated PyXPlot scripts, and its use
should be minimised wherever possible.


\section{exit}\indcmd{exit}

\begin{verbatim}
exit
\end{verbatim}

The \indcmdt{exit} can be used to quit PyXPlot. If multiple command files,
or a mixture of command files and interactive sessions, were specified on
PyXPlot's command line, then PyXPlot moves onto the next command-line item
after receiving the {\tt exit} command.

PyXPlot may also be quit be pressing CTRL-D or using the {\tt quit} command. In
interactive mode, CTRL-C terminates the current command, if one is running.
When running a script, CTRL-C terminates execution of the script.


\section{fft}\indcmd{fft}

\begin{verbatim}
fft {<range>} <function>"()"
    of ( '<filename>' | <function>"()" )
    [using <expression> {:<expression>} ]

ifft {<range>} <function>"()"
    of ( '<filename>' | <function>"()" )
    [using <expression> {:<expression>} ]
\end{verbatim}

The \indcmdt{fft} calculates Fourier transforms of \datafile s or functions.
Transforms can be performed on datasets with arbitrary numbers of dimensions.
To transform an algebraic expression with $n$~degrees of freedom, it must be
wrapped in a function of the form $f(i_2,i_2,\ldots,i_n)$. To transform an
$n$-dimensional dataset stored in a \datafile, the samples must be arranged on
a regular linearly-spaced grid and stored in row-major order.  For each
dimension of the transform, a range specification must be provided to the {\tt
fft} command in the form

\begin{verbatim}
[ <minimum> : <maximum> : <step> ]
\end{verbatim}

When data from a \datafile\ is being transformed, the specified range(s) must
precisely match those of the samples read from the file; the first $n$~columns
of data should contain the values of the $n$~real-space coordinates, and the
$n+1$th column should contain the data to be transformed.  After the range(s),
a function name should be provided for the output transform: a function of
$n$~arguments with this name will be generated to represent the transformed
data.  Note that this function is in general complex -- i.e.\ it has a non-zero
imaginary component. Complex numerics can be enabled using the {\tt set
numerics complex} command and the {\tt fft} command is of little use without
doing so. The {\tt using}, {\tt index}, {\tt every} and {\tt select} modifiers
can be used to specify how data will be sampled from the input function or
\datafile\ in an analogous manner to how they are used in the {\tt plot}
command.

The {\tt ifft} command calculates inverse Fourier transforms; it has the same
syntax as the {\tt fft} command.

\section{fit}\indcmd{fit}

\begin{verbatim}
fit [{<range>}] <function>"()" [withouterrors] '<datafile>'
    [index <value>]
    [using <expression> {:<expression>} ]
    via <variable> {, <variable>}
\end{verbatim}

The \indcmdt{fit} can be used to fit arbitrary functional forms to \datapoint s
read from files. It can be used to produce best-fit lines for datasets or to
determine gradients and other mathematical properties of data by looking at the
parameters associated with the best-fitting functional form.  The following
simple example fits a straight line to data in a file called {\tt data.dat}:

\begin{verbatim}
f(x) = a*x+b
fit f() 'data.dat' index 1 using 2:3 via a,b
\end{verbatim}

\noindent The first line specifies the functional form which is to be used.
The coefficients within this function, {\tt a} and {\tt b}, which are to be
varied during the fitting process are listed after the keyword \indkeyt{via}
in the {\tt fit} command.  The modifiers \indmodt{index}, \indmodt{every},
\indmodt{select} and \indmodt{using} have the same meanings in the {\tt fit}
command as in the {\tt plot} command. When fitting a function of $n$
variables, at least $n+1$ columns (or rows -- see
Section~\ref{sec:horizontal_datafiles}) of data must be specified after the {\tt using}
modifier. By default, the first $n+1$ columns are used. These correspond to the
values of each of the $n$ arguments to the function, plus finally the value which
the output from the function is aiming to match.  If an additional column is
specified, then this is taken to contain the standard error in the value that
the output from the function is aiming to match, and can be used to weight the
\datapoint s which are being used to constrain the fit.

As the {\tt fit} command works, it displays statistics including the best-fit
values of each of the fitting parameters, the uncertainties in each of them,
and the covariance matrix. These can be useful for analysing the security of
the fit achieved, but calculating the uncertainties in the best-fit parameters
and the covariance matrix can be time consuming, especially when many
parameters are being fitted simultaneously. The optional keyword {\tt
withouterrors} can be included immediately before the filename of the
\datafile\ to be fitted to substantially speed up cases where this information
is not required.

By default, the starting values for each of the fitting parameters is
$1.0$. However, if the variables to be used in the fitting process are already
set before the {\tt fit} command is called, these initial values are used
instead. For example, the following would use the initial values
$\{a=100,b=50\}$:
\begin{verbatim}
f(x) = a*x+b
a = 100
b = 50
fit f() 'data.dat' index 1 using 2:3 via a,b
\end{verbatim}

More details can be found in Section~\ref{sec:fit_command}.


\section{for}\indcmd{for}

\begin{verbatim}
for <variable> = <start> to <end> [step <step>]
                          [loopname <loopname>]
 "{"
  ...
 "}"
\end{verbatim}

The \indcmdt{for} executes a set of commands repeatedly, with a specified
variable taking a different value on each iteration. The variable takes the
value {\tt start} on the first iteration, and increases by a fixed value {\tt
step} on each iteration; {\tt step} may be negative if {\tt end} $<$ {\tt
start}. If {\tt step} is not specified then a value of unity is assumed. The
loop terminates when the variable exceeds or equals {\tt end}; note that the
loop is not executed with the variable equalling {\tt end}.  The following
example prints the squares of the first five natural numbers:

\begin{verbatim}
for i = 1 to 6
 {
  print i**2
 }
\end{verbatim}


\section{foreach}\indcmd{foreach}

\begin{verbatim}
foreach <variable> in ( <filename expression> |
                         "("<value> {, <value>}")" )
                      [loopname <loopname>]
 "{"
  ...
 "}"
\end{verbatim}

The \indcmdt{foreach} can be used to run a block of commands repeatedly, once
for each item in a list.  The list of items can be specified in one of two
ways.  In the first case, a set of filenames or filename wildcards is supplied,
and the {\tt foreach} loop iterates once for each supplied filename, with a
string variable set to each filename in succession.  For example, the following
loop would plot the data in the set of files whose names end with {\tt .dat}:

\begin{verbatim}
plot     # Create blank plot
foreach file in *.dat
 {
  replot file with lines
 }
\end{verbatim}

The second form of the command takes a list of string or numerical values
provided explicitly by the user, and the {\tt foreach} loop iterates once for
each value, with a variable set to each value in succession.  For example, the
following script would plot normal distributions of three different widths:

\begin{verbatim}
plot     # Create blank plot
foreach sigma in (1, 2, 3)
 {
  replot 1/sigma*exp(-x**2/(2*sigma**2))
 }
\end{verbatim}


\section{help}\indcmd{help}

\begin{verbatim}
help [<topic> {<sub-topic>} ]
\end{verbatim}

The \indcmdt{help} provides an hierarchical source of information which is
supplementary to this Users' Guide.  To obtain information on any particular
topic, type {\tt help} followed by the name of the topic, as in the following
example

\begin{verbatim}
help commands
\end{verbatim}

\noindent which provides information on PyXPlot's commands. Some topics have
sub-topics; these are listed at the end of each help page. To view them, add
further words to the end of the help request, as in the example:

\begin{verbatim}
help commands help
\end{verbatim}

Information is arranged with general information about PyXPlot under the
heading {\tt about} and information about PyXPlot's commands under {\tt
commands}.  Information about the format that input \datafile s should take can
be found under {\tt datafile}.  Other categories are self-explanatory.

To exit any help page, press the {\tt q} key.


\section{histogram}\indcmd{histogram}

\begin{verbatim}
histogram [<range>] <function name>"()" '<datafile>'
     [every <expression> {:<expression>} ]
     [index <value>]
     [select <expression>]
     [using <expression> {:<expression>} ]
     ( [binwidth <value>] [binorigin <value>] |
       [bins (x1, x2, ...)] )
\end{verbatim}

The \indcmdt{histogram} takes a single column of data from a file and produces
a function that represents the frequency distribution of the supplied data
values. The output function consists of a series of discrete intervals which we
term {\it bins}. Within each interval the output function has a constant value,
determined such that the area under each interval -- i.e.\ the integral of the
function over each interval -- is equal to the number of datapoints found
within that interval.  The following simple example

\begin{verbatim}
histogram f() 'input.dat'
\end{verbatim}

\noindent produces a frequency distribution of the data values found in the
first column of the file {\tt input.dat}, which it stores in the function
$f(x)$. The value of this function at any given point is equal to the number of
items in the bin at that point, divided by the width of the bins used. If the
input datapoints are not dimensionless then the output frequency distribution
adopts appropriate units, thus a histogram of data with units of length has
units of one over length.

The number and arrangement of bins used by the \indcmdt{histogram} can be
controlled by means of various modifiers.  The \indmodt{binwidth} modifier sets
the width of the bins used. The \indmodt{binorigin} modifier controls where
their boundaries lie; the \indcmdt{histogram} selects a system of bins which,
if extended to infinity in both directions, would put a bin boundary at the
value specified in the {\tt binorigin} modifier. Thus, if {\tt binorigin 0.1}
were specified, together with a bin width of~20, bin boundaries might lie
at~$20.1$, $40.1$, $60.1$, and so on. Alternatively global defaults for the bin
width and the bin origin can be specified using the {\tt set binwidth} and {\tt
set binorigin} commands respectively. The example

\begin{verbatim}
histogram h() 'input.dat' binorigin 0.5 binwidth 2
\end{verbatim}

\noindent would bin data into bins between $0.5$ and $2.5$, between $2.5$ and
$4.5$, and so forth.

Alternatively the set of bins to be used can be controlled more precisely using
the \indmodt{bins} modifier, which allows an arbitrary set of bins to be
specified. The example

\begin{verbatim}
histogram g() 'input.dat' bins (1, 2, 4)
\end{verbatim}

\noindent would bin the data into two bins, $x=1\to 2$ and $x=2\to 4$.

A range can be supplied immediately following the {\tt histogram} command,
using the same syntax as in the {\tt plot} and {\tt fit} commands; if such a
range is supplied, only points that fall within that range will be binned.  In
the same way as in the {\tt plot} command, the \indmodt{index},
\indmodt{every}, \indmodt{using} and \indmodt{select} modifiers can be used to
specify which subsets of a \datafile\ should be used.

Two points about the {\tt histogram} command are worthy of note. First,
although histograms are similar to bar charts, they are not the same.  A bar
chart conventionally has the height of each bar equal to the number of points
that it represents, whereas a histogram is a continuous function in which the
area underneath each interval is equal to the number of points within it.
Thus, to produce a bar chart using the {\tt histogram} command, the end result
should be multiplied by the bin width used.

Second, if the function produced by the {\tt histogram} command is plotted
using the {\tt plot} command, samples are automatically taken not at evenly
spaced intervals along the ordinate axis, but at the centres of each bin. If
the \indpst{boxes} plot style is used, the box boundaries are be conveniently
drawn to coincide with the bins into which the data were sorted.


\section{history}\indcmd{history}

\begin{verbatim}
history [<number of items>]
\end{verbatim}

The \indcmdt{history} prints a list of the most recently executed commands on
the terminal.  The optional parameter, {\tt N}, if supplied, causes only the
latest $N$ commands to be displayed.


\section{if}\indcmd{if}\indcmd{else if}\indcmd{else}

\begin{verbatim}
if <criterion>
 "{"
    ...
 "}" { else if <criterion>
 "{"
    ...
 "}" } [ else
 "{"
    ...
 "}" ]
\end{verbatim}

The \indcmdt{if} allows conditional execution of blocks of commands.  The code
enclosed in braces following the {\tt if} statement is executed if, and only
if, the {\tt criterion} is satisfied.  An arbitrary number of subsequent {\tt
else if} statements can optionally follow the initial {\tt if} statement; these
have their own criteria for execution which are only considered if all of the
previous criteria have tested false -- i.e.\ if none of the previous command
blocks have been executed.  A final optional {\tt else} statement can be
provided; the block of commands which follows it are executed only if none of
the preceding criteria have tested true.  The following example illustrates a
chain of {\tt else if} clauses:

\begin{verbatim}
if (x==2)
 {
  print "x is two!"
 } else if (x==3) {
  print "x is three!"
 } else if (x>3) {
  print "x is greater than three!"
 } else {
  x=2
  print "x didn't used to be two, but it is now!"
 }
\end{verbatim}


\section{ifft}\indcmd{ifft}

\begin{verbatim}
ifft {<range>} <function>"()"
    of ( '<filename>' | <function>"()" )
    [using <expression> {:<expression>} ]
\end{verbatim}

See {\tt fft}.


\section{image}\indcmd{image}

\begin{verbatim}
image [ item <id> ] '<filename>' [at <x>, <y>] [rotate <angle>]
                     [width <width>] [height <height>] [smooth]
                   [notransparent] [transparent rgb<r>:<g>:<b>]
\end{verbatim}

The \indcmdt{image} allows graphical images to be inserted onto the current
multiplot canvas from files on disk. Input graphical images may be in bitmap,
gif, jpeg or png formats; the file type of each image is automatically
detected. The {\tt at} modifier can be used to specify where the bottom-left
corner of each image should be placed; if no position is specified then this
corner of the image is placed at the origin. The {\tt rotate} modifier can be
used to rotate images by any angle, measured in degrees counter-clockwise.  The
{\tt width} or {\tt height} modifiers can be used to specify the width or
height with which images should be rendered; both should be specified in
centimetres. If neither is specified then images are rendered with the native
dimensions specified within the metadata present in the image file (if any). If
both are specified, then the aspect ratio of the image is changed.

The keyword {\tt smooth} may optionally be supplied to cause the pixels of
images to be interpolated\footnote{Many commonly-used PostScript display
engines, including Ghostscript, do not support this functionality.}.  Images
which include transparency are supported. The optional keyword {\tt
notransparent} may be supplied to cause transparent regions to be filled with
the image's default background colour. Alternatively, an RGB colour may be
specified in the form {\tt rgb<r>:<g>:<b>} after the keyword {\tt transparent}
to cause that particular colour to become transparent; the three components of
the RGB colour should be in the range~0 to~255.

All vector graphics objects placed on multiplot canvases receive unique
identification numbers which count sequentially from one, and which may be
listed using the {\tt list} command.  By reference to these numbers, they can
be deleted and subsequently restored with the {\tt delete} and {\tt undelete}
commands respectively.


\section{interpolate}\indcmd{interpolate}

\begin{verbatim}
interpolate ( akima | linear | loglinear | polynomial |
              spline | stepwise |
              2d [( bmp_r | bmp_g | bmp_b )] )
            [<range specification>] <function name>"()"
            '<filename>'
            [ every <expression> {:<expression} ]
            [ index <value> ]
            [ select <expression> ]
            [ using <expression> {:<expression} ]
\end{verbatim}

The \indcmdt{interpolate} can be used to generate a special function within
PyXPlot's mathematical environment which interpolates a set of \datapoint s
supplied from a \datafile. Either one- or two-dimensional interpolation is
possible.

In the case of one-dimensional interpolation, various different types of
interpolation are supported: linear interpolation, power law interpolation,
polynomial interpolation, cubic spline interpolation and akima spline
interpolation. Stepwise interpolation returns the value of the datapoint
nearest to the requested point in argument space. The use of polynomial
interpolation with large datasets is strongly discouraged, as polynomial fits
tend to show severe oscillations between \datapoint s.  Except in the case of
stepwise interpolation, extrapolation is not permitted; if an attempt is made
to evaluate an interpolated function beyond the limits of the \datapoint s
which it interpolates, PyXPlot returns an error or value of not-a-number.

In the case of two-dimensional interpolation, the type of interpolation to be
used is set using the {\tt interpolate} modifier to the \indcmdt{set samples},
and may be changed at any time after the interpolation function has been
created.  The options available are nearest neighbour interpolation -- which is
the two-dimensional equivalent of stepwise interpolation, inverse square
interpolation -- which returns a weighted average of the supplied \datapoint s,
using the inverse squares of their distances from the requested point in
argument space as weights, and Monaghan Lattanzio interpolation, which uses the
weighting function (Monaghan \& Lattanzio 1985)
\begin{eqnarray*}
w(x) & = 1 - \nicefrac{3}{2}v^2 + \nicefrac{3}{4}v^3 & \,\mathrm{for~}0\leq v\leq 1 \\
     & = \nicefrac{1}{4}(2-v)^3                      & \,\mathrm{for~}1\leq v\leq 2
\end{eqnarray*}
where $v=r/h$ for $h=\sqrt{A/n}$, $A$ is the product
$(x_\mathrm{max}-x_\mathrm{min})(y_\mathrm{max}-y_\mathrm{min})$ and $n$ is the
number of input datapoints. These are selected as follows:

\begin{verbatim}
set samples interpolate NearestNeighbour
set samples interpolate InverseSquare
set samples interpolate MonaghanLattanzio
\end{verbatim}

Finally, data can be imported from graphical images in bitmap ({\tt .bmp}) 
format to produce a function of two arguments returning a value in the
range~$0\to1$ which represents the data in one of the image's three colour
channels. The two arguments are the horizontal and vertical position within the
bitmap image, as measured in pixels.

A very common application of the \indcmdt{interpolate} is to perform arithmetic
functions such as addition or subtraction on datasets which are not sampled at
the same abscissa values. The following example would plot the difference
between two such datasets:

\begin{verbatim}
interpolate linear f() 'data1.dat'
interpolate linear g() 'data2.dat'
plot [min:max] f(x)-g(x)
\end{verbatim}

\noindent Note that it is advisable to supply a range to the {\tt plot} command
in this example: because the two datasets have been turned into continuous
functions, the {\tt plot} command has to guess a range over which to plot them
unless one is explicitly supplied.

The \indcmdt{spline} is an alias for {\tt interpolate spline}; the following
two statements are equivalent:

\begin{verbatim}
spline f() 'data1.dat'
interpolate spline f() 'data1.dat'
\end{verbatim}


\section{jpeg}\indcmd{jpeg}

\begin{verbatim}
jpeg [ item <id> ] '<filename>' [at <x>, <y>] [rotate <angle>]
                    [width <width>] [height <height>] [smooth]
                  [notransparent] [transparent rgb<r>:<g>:<b>]
\end{verbatim}

See {\tt image}.


\section{let}\indcmd{let}

\begin{verbatim}
let <varname> = <value>
\end{verbatim}

The \indcmdt{let} sets the variable {\tt varname} to equal {\tt value}.


\section{list}\indcmd{list}

\begin{verbatim}
list
\end{verbatim}

The \indcmdt{list} returns a list of all of the items on the current multiplot
canvas, giving their identification numbers and the commands used to produce
them.  The following is an example of the output produced:

{\footnotesize
\begin{verbatim}
pyxplot> list
# ID   Command
    1  plot item 1 f(x) using columns
    2  [deleted] text item 2 "Figure 1: A plot of f(x)" at 0,0 rotate 0 gap 0
    3  text item 3 "Figure 1: A plot of $f(x)$" at 0,0 rotate 0 gap 0
\end{verbatim}
}

In this example, the user has plotted a graph of $f(x)$ and added a caption to
it. The {\tt ID} column lists the reference numbers of each multiplot item.
Item {\tt 1} has been deleted.


\section{load}\indcmd{load}

\begin{verbatim}
load '<filename>'
\end{verbatim}

The \indcmdt{load} executes a PyXPlot command script file, just as if its
contents had been typed into the current terminal. For example:

\begin{verbatim}
load 'foo'
\end{verbatim}

\noindent would have the same effect as typing the contents of the file {\tt
foo} into the present terminal.  Filename wildcard can be supplied, in which
case {\it all} command files matching the given wildcard are executed, as in
the example:

\begin{verbatim}
load '*.script'
\end{verbatim}


\section{maximise}\indcmd{maximise}

\begin{verbatim}
maximise <expression> via <variable> {, variable}
\end{verbatim}

The \indcmdt{maximise} can be used to find the maxima of algebraic expressions.
A single algebraic expression should be supplied for optimisation, together
with a comma-separated list of the variables with respect to which it should be
optimised.  In the following example, a maximum of the sinusoidal function
$\cos(x)$ is sought:

\vspace{3mm}
\noindent{\tt pyxplot> {\bf set numerics real}}\newline
\noindent{\tt pyxplot> {\bf x=0.1}}\newline
\noindent{\tt pyxplot> {\bf maximise cos(x) via x}}\newline
\noindent{\tt pyxplot> {\bf print x/pi}}\newline
\noindent{\tt 0}
\vspace{3mm}

\noindent Note that this particular example doesn't work when complex
arithmetic is enabled, since $\cos(x)$ diverges to $\infty$ at $x=\infty i$.

Various caveats apply the {\tt maximise} command, as well as to the {\tt
minimise} and {\tt solve} commands.  All of these commands operate by searching
numerically for optimal sets of input parameters to meet the criteria set by
the user. As with all numerical algorithms, there is no guarantee that the {\it
locally} optimum solutions returned are the {\it globally} optimum solutions.
It is always advisable to double-check that the answers returned agree with
common sense.

These commands can often find solutions to equations when these solutions are
either very large or very small, but they usually work best when the solution
they are looking for is roughly of order unity.  PyXPlot does have mechanisms
which attempt to correct cases where the supplied initial guess turns out to be
many orders of magnitude different from the true solution, but it cannot be
guaranteed not to wildly overshoot and produce unexpected results in such
cases.  To reiterate, it is always advisable to double-check that the answers
returned agree with common sense.


\section{minimise}\indcmd{minimise}

\begin{verbatim}
minimise <expression> via <variable> {, variable}
\end{verbatim}

The \indcmdt{minimise} can be used to find the minima of algebraic expressions.
A single algebraic expression should be supplied for optimisation, together
with a comma-separated list of the variables with respect to which it should be
optimised. In the following example, a minimum of the sinusoidal function
$\cos(x)$ is sought:

\vspace{3mm}
\noindent{\tt pyxplot> {\bf set numerics real}}\newline
\noindent{\tt pyxplot> {\bf x=0.1}}\newline
\noindent{\tt pyxplot> {\bf minimise cos(x) via x}}\newline
\noindent{\tt pyxplot> {\bf print x/pi}}\newline
\noindent{\tt 1}
\vspace{3mm}

\noindent Note that this particular example doesn't work when complex
arithmetic is enabled, since $\cos(x)$ diverges to $-\infty$ at $x=\pi+\infty
i$.

Various caveats apply the {\tt minimise} command, as well as to the {\tt
maximise} and {\tt solve} commands.  All of these commands operate by searching
numerically for optimal sets of input parameters to meet the criteria set by
the user. As with all numerical algorithms, there is no guarantee that the {\it
locally} optimum solutions returned are the {\it globally} optimum solutions.
It is always advisable to double-check that the answers returned agree with
common sense.

These commands can often find solutions to equations when these solutions are
either very large or very small, but they usually work best when the solution
they are looking for is roughly of order unity.  PyXPlot does have mechanisms
which attempt to correct cases where the supplied initial guess turns out to be
many orders of magnitude different from the true solution, but it cannot be
guaranteed not to wildly overshoot and produce unexpected results in such
cases.  To reiterate, it is always advisable to double-check that the answers
returned agree with common sense.


\section{move}\indcmd{move}

\begin{verbatim}
move <item number> to <x>, <y>
\end{verbatim}

The \indcmdt{move} allows vector graphics objects to be moved around on the
current multiplot canvas. All vector graphics objects placed on multiplot
canvases receive unique identification numbers which count sequentially from
one, and which may be listed using the {\tt list} command. The item to be moved
should be specified using its identification number. The example

\begin{verbatim}
move 23 to 8,8
\end{verbatim}

\noindent would move multiplot item~23 to position $(8,8)$ centimetres. If this
item were a plot, the end result would be the same as if the command {\tt set
origin 8,8} had been executed before it had originally been plotted.


\section{plot}\indcmd{plot}

\begin{verbatim}
plot [3d] [item <id>] [{<range>}] ( '<filename>' | <function> )
     [axes <axes>] [every <expression> {:<expression>}]
     [index <value>] [select <expression>]
     [label <string expression>]
     [title <string>] [using <expression> {:<expression>}]
     [with {<option>}]
\end{verbatim}

The \indcmdt{plot} is used to produce graphs. The following simple example
would plot the sine function:

\begin{verbatim}
plot sin(x)
\end{verbatim}

Ranges for the axes of a graph can be specified by placing them in
square brackets before the name of the function to be plotted. An example of
this syntax would be:

\begin{verbatim}
plot [-pi:pi] sin(x)
\end{verbatim}

\noindent which would plot the function $\sin(x)$ between $-\pi$ and $\pi$.

\Datafile s may also be plotted as well as functions, in which case the
filename of the \datafile\ to be plotted should be enclosed in either single or
double quotation marks. An example of this syntax would be:

\begin{verbatim}
plot 'data.dat' with points
\end{verbatim}

\noindent which would plot data from the file {\tt data.dat}.
Section~\ref{sec:plot_datafiles} provides further details of the format that
input \datafile s should take and how PyXPlot may be directed to plot only
certain portions of \datafile s.

Multiple datasets can be plotted on a single graph by specifying them in a
comma-separated list, as in the example:

\begin{verbatim}
plot sin(x) with colour blue, cos(x) with linetype 2
\end{verbatim}

If the {\tt 3d} modifier is supplied to the {\tt plot} command, then a
three-dimensional plot is produced; many plot styles then take additional
columns of data to signify the positions of datapoints along the $z$-axis. This
is described further in Chapter~\ref{ch:plotting}. The angle from which
three-dimensional plots are viewed can be set using the {\tt set view} command.


\subsection{axes}\indcmd{plot axes}

The {\tt axes} modifier may be used to specify which axes data should be
plotted against when plots have multiple parallel axes -- for example, if a
plot has an {\tt x}-axis along its lower edge and an {\tt x2}-axis along its
upper edge. The following example would plot data using the {\tt x2}-axis as
the ordinate axis and the {\tt y}-axis as the abscissa axis:

\begin{verbatim}
plot sin(x) axes x2y1
\end{verbatim}

\noindent It is also possible to use the {\tt axes} modifier to specify that a
vertical ordinate axis and a horizontal abscissa axis should be used; the
following example would plot data using the {\tt y}-axis as the ordinate axis
and the {\tt x}-axis as the abscissa axis:

\begin{verbatim}
plot sin(x) axes yx
\end{verbatim}


\subsection{label}\indcmd{plot label}

The {\tt label} modifier to the {\tt plot} command may be used to render text 
labels next to datapoints, as in the following examples:
\begin{verbatim}
set samples 8
plot [2:5] x**2 label "$x=%.2f$"%($1) with points

plot 'datafile' using 1:2 label "%s"%($3)
\end{verbatim}

\noindent Note that if a particular column of a \datafile\ contains strings
which are to be used as labels, as in the second example above, syntax such as 
{\tt "\%s"\%(\$3)} must be used to explicitly read the data as strings rather
than as numerical quantities.  As PyXPlot treats any whitespace as separating
columns of data, such labels cannot contain spaces, though \LaTeX's {\tt
$\sim$} character can be used to achieve a space.

Datapoints can be labelled when plotted in any of the following plot styles:
{\tt arrows} (and similar styles), {\tt dots}, {\tt errorbars} (and similar
styles), {\tt errorrange} (and similar styles), {\tt impulses}, {\tt
linespoints}, {\tt lowerlimits}, {\tt points}, {\tt stars} and {\tt 
upperlimits}. It is not possible to label datapoints plotted in other styles.
Labels are rendered in the same colour as the datapoints with which they are
associated.


\subsection{title}\indcmd{plot title}

By default, PyXPlot generates its own entry in the legend of a plot for each
dataset plotted.  This default behaviour can be overridden using the {\tt
title} modifier. The following example labels a dataset as `Dataset~1':

\begin{verbatim}
plot sin(x) title 'Dataset 1'
\end{verbatim}

\noindent If a blank string, i.e.\ {\tt ""}, is supplied, then no entry is made
in the plot's legend for that dataset. The same effect can be achieved using
the {\tt notitle} modifier.


\subsection{with}\indcmd{plot with}

The {\tt with} modifier controls the style in which data should be plotted. For
example, the statement
\begin{verbatim}
plot "data.dat" index 1 using 4:5 with lines
\end{verbatim}
specifies that data should be plotted using lines connecting each \datapoint to
its neighbours. More generally, the {\tt with} modifier can be followed by a
range of settings which fine-tune the manner in which the data are displayed;
for example, the statement
\begin{verbatim}
plot "data.dat" with lines linewidth 2.0
\end{verbatim}
would use twice the default width of line.

The following is a complete list of all of PyXPlot's plot styles -- i.e.\ all
of the words which may be used in place of {\tt lines}: {\tt arrows\_\-head},
{\tt arrows\_\-no\-head}, {\tt arrows\_\-two\-head}, {\tt boxes}, {\tt
Colour\-Map}, {\tt Contour\-Map}, {\tt dots}, {\tt Filled\-Region}, {\tt
fsteps}, {\tt histeps}, {\tt impulses}, {\tt lines}, {\tt Lines\-Points}, {\tt
Lower\-Limits}, {\tt points}, {\tt stars}, {\tt steps}, {\tt surface}, {\tt
Upper\-Limits}, {\tt wbox\-es}, {\tt X\-Error\-Bars}, {\tt X\-Error\-Range},
{\tt XY\-Error\-Bars}, {\tt XY\-Error\-Range}, {\tt XYZ\-Error\-Bars}, {\tt
XYZ\-Error\-Range}, {\tt XZ\-Error\-Bars}, {\tt XZ\-Error\-Range}, {\tt
Y\-Error\-Bars}, {\tt Y\-Error\-Range}, {\tt Y\-Error\-Shaded}, {\tt
YZ\-Error\-Bars}, {\tt YZ\-Error\-Range}, {\tt Z\-Error\-Bars}, {\tt
Z\-Error\-Range}. In addition, {\tt lp} and {\tt pl} are recognised as
abbreviations for {\tt lines\-points}; {\tt error\-bars} is recognised as an
abbreviation for {\tt y\-error\-bars}; {\tt error\-range} is recognised as an
abbreviation for {\tt y\-error\-range}; and {\tt arrows\_\-two\-way} is
recognised as an alternative for {\tt arrows\_\-two\-head}.

As well as the names of these plot styles, the {\tt with} modifier can also be
followed by style modifiers such as {\tt line\-width} which alter the exact
appearance of various plot styles. A complete list of these is as follows:
\begin{itemize}
\item \indmodt{colour} -- used to select the colour in which each dataset is to be plotted. It should be followed either by an integer, to set a colour from the present palette (see Section~\ref{sec:palette}), or by a recognised colour name, a complete list of which can be found in Section~\ref{sec:colour_names}. Alternatively, arbitrary colours may be specified by using one of the forms {\tt rgb0.1:\-0.2:\-0.3}, {\tt hsb0.1:\-0.2:\-0.3} or {\tt cmyk0.4:\-0.3:\-0.2:\-0.1}, where the colon-separated values indicate the RGB, HSB or CMYK components of the desired colour in the range~0 to~1. This modifier may also be spelt {\tt color}.\index{colours!setting for datasets}
\item \indmodt{fillcolour} -- used to select the colour in which each dataset is filled. The colour may be specified using any of the styles listed for {\tt colour}. May also be spelt {\tt fillcolour}.
\item \indmodt{linetype} -- used to numerically select the type of line -- for example, solid, dotted, dashed, etc.\ -- which should be used in line-based plot styles. A complete list of PyXPlot's numbered line types can be found in Chapter~\ref{ch:linetypes_table}. May be abbreviated {\tt lt}.
\item \indmodt{linewidth} -- used to select the width of line which should be used in line-based plot styles, where unity represents the default width. May be abbreviated {\tt lw}.
\item \indmodt{pointlinewidth} -- used to select the width of line which should be used to stroke points in point-based plot styles, where unity represents the default width. May be abbreviated {\tt plw}.
\item \indmodt{pointsize} -- used to select the size of drawn points, where unity represents the default size. May be abbreviated {\tt ps}.
\item \indmodt{pointtype} -- used to numerically select the type of point -- for example, crosses, circles, etc.\ -- used by point-based plot styles. A complete list of PyXPlot's numbered point types can be found in Chapter~\ref{ch:linetypes_table}. May be abbreviated {\tt pt}.
\end{itemize}

Any number of these modifiers may be placed sequentially after the keyword {\tt
with}, as in the following examples:

\begin{verbatim}
plot 'datafile' using 1:2 with points pointsize 2
plot 'datafile' using 1:2 with lines colour red linewidth 2
plot 'datafile' using 1:2 with lp col 1 lw 2 ps 3
\end{verbatim}

\noindent Where modifiers take numerical values, expressions of the form {\tt
\$2+1}, similar to those supplied to the {\tt using} modifier, may be used to
indicate that each datapoint should be displayed in a different style or in a
different colour. The following example would plot a \datafile\ with {\tt
points}, drawing the position of each point from the first two columns of the
supplied \datafile\ and the size of each point from the third column:
\begin{verbatim}
plot 'datafile' using 1:2 with points pointsize $3
\end{verbatim}


\section{print}\indcmd{print}

\begin{verbatim}
print <expression> {, <expression>}
\end{verbatim}

The \indcmdt{print} displays a string or the value of a mathematical expression to the
terminal. It is most often used to find the value of a variable, though it can
also be used to produce formatted textual output from a PyXPlot script. For example,

\begin{verbatim}
print a
\end{verbatim}

\noindent would print the value of the variable {\tt a}, and

\begin{verbatim}
print "a = %s"%(a)
\end{verbatim}

\noindent would produce the same result in the midst of formatted text.


\section{pwd}\indcmd{pwd}

\begin{verbatim}
pwd
\end{verbatim}

The \indcmdt{pwd} prints the location of the current working directory.


\section{quit}\indcmd{quit}

\begin{verbatim}
quit
\end{verbatim}

The \indcmdt{quit} can be used to exit PyXPlot. See {\tt exit} for more
details.


\section{rectangle}\indcmd{rectangle}

\begin{verbatim}
rectangle [ item <id> ] at <x>, <y> width <w> height <h>
               [ rotate <r> ] [ with {<option>} ]

rectangle [ item <id> ] from <x1>, <y1> to <x2>, <y2>
               [ rotate <r> ] [ with {<option>} ]
\end{verbatim}

See {\tt box}.


\section{refresh}\indcmd{refresh}

\begin{verbatim}
refresh
\end{verbatim}

The \indcmdt{refresh} produces an exact copy of the latest display. It can be
useful, for example, after changing the terminal type, to produce a second copy
of a plot in a different graphic format. It differs from the {\tt replot}
command in that it doesn't replot anything; use of the {\tt set} command since
the previous {\tt plot} command has no effect on the output.


\section{point}\indcmd{point}

\begin{verbatim}
point [ item <id> ] [at] <x>, <y> [ label <string> ]
                    [ with {<option>} ]
\end{verbatim}

The \indcmdt{point} allows a single point to be plotted on the current
multiplot canvas independently of any graph.  It is equivalent to plotting a
\datafile\ containing a single datum and with invisible axes.  If an optional
{\tt label} is specified then the text string provided is rendered next to the
point.  The {\tt with} modifier allows the style of the point to be specified
using similar options to those accepted by the {\tt plot} command.

All vector graphics objects placed on multiplot canvases receive unique
identification numbers which count sequentially from one, and which may be
listed using the {\tt list} command.  By reference to these numbers, they can
be deleted and subsequently restored with the {\tt delete} and {\tt undelete}
commands respectively.


\section{replot}\indcmd{replot}

\begin{verbatim}
replot [ item <id> ] ...
\end{verbatim}

The \indcmdt{replot} has the same syntax as the {\tt plot} command and is
used to add more datasets to an existing plot, or to change its axis ranges.
For example, having plotted one \datafile\ using the command

\begin{verbatim}
plot 'datafile1.dat'
\end{verbatim}

\noindent another can be plotted on the same axes using the command

\begin{verbatim}
replot 'datafile2.dat' using 1:3
\end{verbatim}

\noindent or the ranges of the axes on the original plot can be changed using
the command

\begin{verbatim}
replot [0:1][0:1]
\end{verbatim}

\noindent The plot is also updated to reflect any changes to settings made
using the {\tt set} command.  In multiplot mode, the \indcmdt{replot} can
likewise be used to modify the last plot added to the page. For example, the
following would change the title of the latest plot to `foo', and add a plot of
the function $g(x)$ to it:

\begin{verbatim}
set title 'foo'
replot cos(x)
\end{verbatim}

Additionally, in multiplot mode it is possible to modify any plot on the
current multiplot canvas by adding an {\tt item} modifier to the {\tt replot}
statement to specify which plot should be replotted.  The following example
would produce two plots, and then add an additional function to the first plot:

\begin{verbatim}
set multiplot
plot f(x)
set origin 10,0
plot g(x)
replot item 1 h(x)
\end{verbatim}

If no {\tt item} number is specified, then the \indcmdt{replot} acts by default
upon the most recent plot to have been added to the multiplot canvas.


\section{reset}\indcmd{reset}

\begin{verbatim}
reset
\end{verbatim}

The \indcmdt{reset} reverts the values of all settings that have been changed
with the {\tt set} command back to their default values. It also clears the
current multiplot canvas.


\section{save}\indcmd{save}

\begin{verbatim}
save '<filename>'
\end{verbatim}

The \indcmdt{save} saves a list of all of the commands which have been executed
in the current interactive PyXPlot session into a file. The filename to be used
for the output should be placed in quotes, as in the example:

\begin{verbatim}
save 'foo'
\end{verbatim}

\noindent would save a command history into the file named {\tt foo}.


\section{set}\indcmd{set}

\begin{verbatim}
set <option> <value>
\end{verbatim}

The \indcmdt{set} is used to configure the values of a wide range of parameters
within PyXPlot which control its behaviour and the style of the output which it
produces.  For example:

\begin{verbatim}
set pointsize 2
\end{verbatim}

\noindent would set the size of points plotted by PyXPlot to be twice the
default. In the majority of cases, the syntax follows that above: the {\tt set}
command should be followed by a keyword which specifies which parameter
should be set, followed by the value to which that parameter should be
set. Those parameters which work in an on/off fashion take a different syntax
along the lines of:

\vspace{3mm}
\begin{tabular}{ll}
{\tt set key} & {\tt\it \# Set option ON} \\
{\tt set nokey} & {\tt\it \# Set option OFF}
\end{tabular}
\vspace{3mm}

\noindent
More details of the effects of each individual parameter can be found in the
subsections below, which forms a complete list of the recognised setting
keywords.

The reader should also see the {\tt show} command, which can be used to display
the current values of settings, and the {\tt unset} command, which returns
settings to their default values. Chapter~\ref{ch:configuration} describes how
commonly used settings can be saved into a configuration file.


\subsection{arrow}\indcmd{set arrow}

\begin{verbatim}
set arrow <arrow number>
      from [<system>] <x>, [<system>] <y>
      to   [<system>] <x>, [<system>] <y>
      [ with {<option>} ]
\end{verbatim}

\noindent where {\tt <system>} may take any of the values
\newline\noindent
{\tt ( first | second | screen | graph | axis<number> )}
\vspace{5mm}

The \indcmdt{set arrow} is used to add arrows to graphs. The example

\begin{verbatim}
set arrow 1 from 0,0 to 1,1
\end{verbatim}

\noindent would draw an arrow between the points $(0,0)$ and $(1,1)$, as
measured along the {\tt x} and {\tt y}-axes.  The tag {\tt 1} immediately
following the keyword {\tt arrow} is an identification number and allows arrows
to be subsequently removed using the {\tt unset arrow} command.  By default,
the coordinates are specified relative to the first horizontal and vertical
axes, but they can alternatively be specified any one of several of coordinate
systems. The coordinate system to be used is specified as in the example:

\begin{verbatim}
set arrow 1 from first 0, second 0 to axis3 1, axis4 1
\end{verbatim}

\noindent The name of the coordinate system to be used precedes the position
value in that system. The coordinate system {\tt first}, the default, measures
the graph using the {\tt x}- and {\tt y}-axes. {\tt second} uses the {\tt x2}-
and {\tt y2}-axes.  {\tt screen} and {\tt graph} both measure in centimetres
from the origin of the graph.  The syntax {\tt axis<n>} may also be used, to
use the $n$th horizontal or vertical axis; for example, {\tt axis3} above.

The {\tt set arrow} command can be followed by the keyword {\tt with} to
specify the style of the arrow. For example, the specifiers {\tt nohead}, {\tt
head} and {\tt twohead}, when placed after the keyword {\tt with}, can be used
to make arrows with no arrow heads, normal arrow heads, or two arrow heads.
{\tt twoway} is an alias for {\tt twohead}. All of the line type modifiers
accepted by the {\tt plot} command can also be used here, as in the example:

\begin{verbatim}
set arrow 2 from first 0, second 2.5 to axis3 0,
             axis4 2.5 with colour blue nohead
\end{verbatim}


\subsection{autoscale}\indcmd{set autoscale}

\begin{verbatim}
set autoscale {<axis>}
\end{verbatim}

The \indcmdt{set autoscale} causes PyXPlot to choose the scaling for an axis
automatically based on the data and/or functions to be plotted against it. The
example

\begin{verbatim}
set autoscale x1
\end{verbatim}

\noindent would cause the range of the first horizontal axis to be scaled to
fit the data.  Multiple axes can be specified, as in the example

\begin{verbatim}
set autoscale x1y3
\end{verbatim}

\noindent Note that ranges explicitly specified in a \indcmdt{plot} will
override the {\tt set autoscale} command.


\subsection{axescolour}\indcmd{set axescolour}

\begin{verbatim}
set axescolour <colour>
\end{verbatim}

The setting {\tt axescolour} changes the colour used to draw graph axes.  The example

\begin{verbatim}
set axescolour blue
\end{verbatim}

\noindent would specify that graph axes should be drawn in blue. Any of the
recognised colour names listed in Section~\ref{sec:colour_names} can be used.


\subsection{axis}\indcmd{set axis}

\begin{verbatim}
set axis <axis> [ ( visible | invisible ) ]
  [ ( top | bottom | left | right | front | back ) ]
  [ ( atzero | notatzero ) ]
  [ ( automirrored | mirrored | fullmirrored ) ]
  [ ( noarrow | arrow | reversearrow | twowayarrow ) ]
  [ linked [item <number>] <axis> [using <expression>] ]
\end{verbatim}

The \indcmdt{set axis} is used to add additional axes to plots and to
configure their appearance. Where an axis is stated on its own, as in the
example
\begin{verbatim}
set axis x2
\end{verbatim}
additional horizontal or vertical axes are added with default
settings. The converse statements
\begin{verbatim}
set noaxis x2
unset axis x2
\end{verbatim}
are used, respectively, to remove axes from plots and to return them
to their default configurations, which often has the same effect of removing
them from the graph, unless they are configured otherwise in a configuration
file.

The position of any axis can be explicitly set using syntax of the form:
\begin{verbatim}
set axis x top
set axis y right
set axis z back
\end{verbatim}
Horizontal axes can be set to appear either at the {\tt top} or {\tt bottom};
vertical axes can be set to appear either at the {\tt left} or {\tt right}; and
$z$-axes can be set to appear either at the {\tt front} or {\tt back}.  By
default, the {\tt x1}-axis is placed along the bottom of graphs and the {\tt
y1}-axis is placed up the left-hand side of graphs. On three-dimensional plots,
the {\tt z1}-axis is placed at the front of the graph. The second set of axes
are placed opposite to the first: the {\tt x2}-, {\tt y2}- and {\tt z2}-axes
are placed respectively along the top, right and back sides of graphs.
Higher-numbered axes are placed alongside the {\tt x1}-, {\tt y1}- and {\tt
z1}-axes.

The following keywords may also be placed alongside the positional keywords
listed above to specify how the axis should appear:
\begin{itemize}
\item {\tt arrow} -- Specifies that an arrowhead should be drawn on the right/top end of the axis. [{\bf Not default}].
\item {\tt atzero} -- Specifies that rather than being placed along an edge of the plot, the axis should mark the lines where the perpendicular axes {\tt x1}, {\tt y1} and/or {\tt z1} are zero. [{\bf Not default}].
\item {\tt automirrored} -- Specifies that an automatic decision should be made between the behaviour of {\tt mirrored} and {\tt nomirrored}. If there are no axes on the opposite side of the graph, a mirror axis is produced. If there are already axes on the opposite side of the graph, no mirror axis is produced. [{\bf Default}].
\item {\tt fullmirrored} -- Similar to {\tt mirrored}. Specifies that this axis should have a corresponding twin placed on the opposite side of the graph with mirroring ticks and labelling. [{\bf Not default}; see {\tt automirrored}].
\item {\tt invisible} -- Specifies that the axis should not be drawn; data can still be plotted against it, but the axis is unseen. See Example~\ref{ex:australia} for a plot where all of the axes are invisible.
\item {\tt linked} -- Specifies that the axis should be linked to another axis; see Section~\ref{sec:linked_axes}.
\item {\tt mirrored} -- Specifies that this axis should have a corresponding twin placed on the opposite side of the graph with mirroring ticks but with no labels on the ticks. [{\bf Not default}; see {\tt automirrored}].
\item {\tt noarrow} -- Specifies that no arrowheads should be drawn on the ends of the axis. [{\bf Default}].
\item {\tt nomirrored} -- Specifies that this axis should not have any corresponding twins. [{\bf Not default}; see {\tt automirrored}].
\item {\tt notatzero} -- Opposite of {\tt atzero}; the axis should be placed along an edge of the plot. [{\bf Default}].
\item {\tt notlinked} -- Specifies that the axis should no longer be linked to any other; see Section~\ref{sec:linked_axes}. [{\bf Default}].
\item {\tt reversearrow} -- Specifies that an arrowhead should be drawn on the left/bottom end of the axis. [{\bf Not default}].
\item {\tt twowayarrow} -- Specifies that arrowheads should be drawn on both ends of the axis. [{\bf Not default}].
\item {\tt visible} -- Specifies that the axis should be displayed; opposite of {\tt invisible}. [{\bf Default}].
\end{itemize}


\subsection{axisunitstyle}\indcmd{set axisunitstyle}

\begin{verbatim}
set axisunitstyle ( bracketed | squarebracketed | ratio )
\end{verbatim}

The setting {\tt axisunitstyle} controls the style with which the units of
plotted quantities are indicated on the axes of plots. The {\tt bracketed}
option causes the units to be placed in parentheses following the axis labels,
whilst the {\tt square\-bracketed} option using square brackets instead.  The
{\tt ratio} option causes the units to follow the label as a divisor so as to
leave the quantity dimensionless.


\subsection{backup}\indcmd{set backup}

\begin{verbatim}
set backup
\end{verbatim}

The setting {\tt backup} changes PyXPlot's behaviour when it detects that a
file which it is about to write is going to overwrite an existing file. Whereas
by default the existing file would be overwritten by the new one, when the
setting {\tt backup} is turned on, it is renamed, placing a tilde at the end of
its filename. For example, suppose that a plot were to be written with filename
{\tt out.ps}, but such a file already existed.  With the backup setting turned
on the existing file would be renamed {\tt out.ps$\sim$} to save it from being
overwritten.

The setting is turned off using the {\tt set nobackup} command.


\subsection{bar}\indcmd{set bar}

\begin{verbatim}
set bar ( large | small | <value> )
\end{verbatim}

The setting {\tt bar} changes the size of the bar drawn on the end of the error
bars, relative to the current point size.  For example:

\begin{verbatim}
set bar 2
\end{verbatim}

\noindent sets the bars to be twice the size of the points.  The options {\tt large} and
{\tt small} are equivalent to~1 (the default) and~0 (no bar) respectively.


\subsection{binorigin}\indcmd{set binorigin}

\begin{verbatim}
set binorigin <value>
\end{verbatim}

The setting {\tt binorigin} affects the behaviour of the \indcmdt{histogram} by
adjusting where it places the boundaries between the bins into which it places
data. The \indcmdt{histogram} selects a system of bins which, if extended to
infinity in both directions, would put a bin boundary at the value specified in
the \indcmdt{set binorigin}. Thus, if a value of $0.1$ were specified to the
\indcmdt{set binorigin}, and a bin width of~20 were chosen by the
\indcmdt{histogram}, bin boundaries might lie at~$20.1$, $40.1$, $60.1$, and so
on. The specified value may have any physical units, but must be real and
finite.


\subsection{binwidth}\indcmd{set binwidth}

\begin{verbatim}
set binwidth <value>
\end{verbatim}

The setting {\tt binwidth} changes the width of the bins used by the {\tt
histogram} command. The specified width may have any physical units, but must
be real and finite.


\subsection{boxfrom}\indcmd{set boxfrom}

\begin{verbatim}
set boxfrom <value>
\end{verbatim}

The setting {\tt boxfrom} alters the vertical line from which bars are drawn
when PyXPlot draws bar charts.  By default, bars all originate from the line
$y=0$, but the example

\begin{verbatim}
set boxfrom 2
\end{verbatim}

\noindent would make the bars originate from the line $y=2$. The specified
vertical abscissa value may have any physical units, but must be real and
finite.



\subsection{boxwidth}\indcmd{set boxwidth}

\begin{verbatim}
set boxwidth <width>
\end{verbatim}

The setting {\tt boxwidth} alters PyXPlot's behaviour when plotting bar charts.
It sets the default width of the boxes used, in ordinate axis units.  If the
specified width is negative then, as happens by default, the boxes have
automatically selected widths, such that the interfaces between them occur at
the horizontal midpoints between their specified positions.  For example:

\begin{verbatim}
set boxwidth 2
\end{verbatim}

\noindent would set all boxes to be two units wide, and

\begin{verbatim}
set boxwidth -2
\end{verbatim}

\noindent would set all of the bars to have differing widths, centred upon
their specified horizontal positions, such that their interfaces occur at the
horizontal midpoints between them. The specified width may have any physical
units, but must be real and finite.


\subsection{c1format}\indcmd{set x1format}

\begin{verbatim}
set c1format ( auto | '<format>' )
      ( horizontal | vertical | rotate <angle> )
\end{verbatim}

The {\tt c1format} setting is used to manually specify an explicit format for
the axis labels to take along the colour scale bars drawn alongside plots which
make use of the \indpst{colourmap} plot style. It has similar syntax to the
{\tt set xformat} command.

\subsection{c1label}\indcmd{set c1label}

\begin{verbatim}
set c1label '<text>' [ rotate <angle> ]
\end{verbatim}

The setting {\tt c1label} sets the label which should be written alongside the
colour scale bars drawn next to plots when the \indpst{colourmap} plot style
is used. An optional rotation angle may be specified to rotate axis labels
clockwise by arbitrary angles. The angle should be specified either as a
dimensionless number of degrees, or as a quantity with physical dimensions of
angle.


\subsection{calendar}\indcmd{set calendar}

\begin{verbatim}
set calendar [ ( input | output ) ] <calendar>
\end{verbatim}

The \indcmdt{set calendar} sets the calendar that PyXPlot uses to convert dates
between calendar dates and Julian Day numbers. PyXPlot uses two separate
calendars which may be different: an input calendar for processing dates that
the user inputs as calendar dates, and an output calendar that controls how
dates are displayed or written on plots.  The available calendars are {\tt
British}, {\tt French}, {\tt Greek}, {\tt Gregorian}, {\tt Hebrew}, {\tt
Islamic}, {\tt Jewish}, {\tt Julian}, {\tt Muslim}, {\tt Papal} and {\tt
Russian}, where {\tt Jewish} is an alias for {\tt Hebrew} and {\tt Muslim} is
an alias for {\tt Islamic}.


\subsection{clip}\indcmd{set clip}

\begin{verbatim}
set clip
\end{verbatim}

The \indcmdt{set clip} causes PyXPlot to clip points which extend over the edge
of plots. The opposite effect is achieved using the {\tt set noclip}
command.


\subsection{colourkey}\indcmd{set colourkey}

\begin{verbatim}
set colourkey [<position>]
\end{verbatim}

The setting {\tt colourkey} determines whether colour scales are drawn along
the edges of plots drawn using the \indpst{colourmap} plot style, indicating
the mapping between represented values and colours. Note that such scales are
only ever drawn when the \indpst{colourmap} plot style is supplied with only
three columns of data, since the colour mappings are themselves
multi-dimensional when more columns are supplied. Issuing the command

\begin{verbatim}
set colourkey
\end{verbatim}

\noindent by itself causes such a scale to be drawn on graphs in the default
position, usually along the right-hand edge of the graphs. The converse action
is achieved by:

\begin{verbatim}
set nocolourkey
\end{verbatim}

\noindent The command

\begin{verbatim}
unset colourkey
\end{verbatim}

\noindent causes PyXPlot to revert to its default behaviour, as specified in a
configuration file, if present. A position for the key may optionally be
specified after the {\tt set colourkey} command, as in the example:

\begin{verbatim}
set colourkey bottom
\end{verbatim}

Recognised positions are {\tt top}, {\tt bottom}, {\tt left} and {\tt right}.
{\tt above} is an alias for {\tt top}; {\tt below} is an alias for {\tt bottom}
and {\tt outside} is an alias for {\tt right}.


\subsection{colourmap}\indcmd{set colourmap}

\begin{verbatim}
set colourmap ( rgb<r>:<g>:<b> |
                hsb<h>:<s>:<b> |
                cmyk<c>:<m>:<y>:<k> )
              [ mask <expr> ]
\end{verbatim}

The setting {\tt colourmap} is used to specify the mapping between ordinate
values and colours used by the \indpst{colourmap} plot style. As elsewhere in
PyXPlot, the colour components should be numerical expressions which evaluate
to a value in the range zero to one. Within these expressions, the variables
{\tt c1}, {\tt c2}, {\tt c3} and {\tt c4} refer quantities calculated from the
third through sixth columns of data supplied to the \indpst{colourmap} plot
style in a way determined by the {\tt c<n>range} setting.  Thus, the following
colour mapping, which is the default, produces a greyscale colour mapping of
the third column of data supplied to the \indpst{colourmap} plot style; further
columns of data, if supplied, are not used:

\begin{verbatim}
set c1range [*:*] renormalise
set colourmap rgb(c1):(c1):(c1)
\end{verbatim}

If a mask expression is supplied, then any areas in a colour map where this
expression evaluates to zero (i.e.\ false) are made transparent.


\subsection{contours}\indcmd{set contours}

\begin{verbatim}
set contours [ ( <number> |
               \( <value> {, <value>} \) ) ]
             [ (label | nolabel) ]
\end{verbatim}

The setting {\tt contours} is used to define the set of ordinate values for
which contours are drawn when using the \indpst{contourmap} plot style. If {\tt
<number>} is specified, the contours are evenly spaced -- either linearly or
logarithmically, depending upon the state of the {\tt logscale c1} setting --
between the values specified in the {\tt c1range} setting. Otherwise, the list
of ordinate values may be specified as a ()-bracketed comma-separated list.

If the option {\tt label} is specified, then each contour is labelled with the
ordinate value that it follows. If the option {\tt nolabel} is specified, then
the contours are not labelled.


\subsection{c$<$n$>$range}\indcmd{set crange}

\begin{verbatim}
set c<n>range [ <range> ]
              [ reversed | noreversed ]
              [ renormalise | norenormalise ]
\end{verbatim}

The {\tt set c<n>range} command changes the range of ordinate values
represented by different colours in the \indpst{colourmap} plot style, and in
the case of the {\tt set c1range} command, also by contours in the
\indpst{contourmap} plot style. The value {\tt <n>} should be an integer in the
range 1--4.

\subsubsection{Contour Maps}

The effect of the {\tt set c1range} command upon the set of ordinate values for
which contours are drawn in the {\tt contourmap} plot style is dependent upon
whether the {\tt set contours} command has been supplied with a number of
contours to draw, or a list of explicit ordinate values for which they should
be drawn. In the latter case, the {\tt set c1range} command has no effect. In
the former case, the contours are evenly spaced, either linearly or
logarithmically depending upon the state of the {\tt logscale c1} setting,
between the minimum and maximum ordinate values supplied to the {\tt set
c1range} command.  If an asterisk ({\tt *}) is supplied in place of either the
minimum and/or the maximum, then the range of values used is automatically
scaled to fit the range of the data supplied.

\subsubsection{Colour Maps}

The colour of each pixel in a colour map is determined by the {\tt colourmap}
setting, which should contain an expression of the form {\tt
rgb(c1):(c2):(c3)}, specifying the components of a colour in either RGB, HSB or
CMYK space, as a function of the variables {\tt c1}, {\tt c2}, {\tt c3} and
{\tt c4}. The {\tt colourmap} plot style should be supplied with between three
and six columns of data, the first two of which contain the $x$- and
$y$-positions of points, and the remainder of which are used to set the values
of the variables {\tt c1}, {\tt c2}, {\tt c3} and {\tt c4} when calculating the
colour with which that point should be represented. If fewer than six columns
of data are supplied, then not all of these variables will be set.

The {\tt set c<n>range} command is used to determine how the raw data values
are mapped to the values of the variables {\tt c1}--{\tt c4}. If the {\tt
no\-renor\-malise} option is specified, then the raw values are passed directly
to the expression. Otherwise, they are first scaled into the range zero to one.
If an explicit range is specified to the {\tt set c<n>range} command, then the
upper limit of this range maps to the value one, and the lower limit maps to
the value zero. This mapping is inverted if the {\tt reverse} option is
specified, such that the upper limit maps to zero, and the lower limit maps to
one. If an asterisk ({\tt *}) is supplied in place of either the upper and/or
lower limit, then the range automatically scales to fit the data supplied.
Intermediate values are scaled, either linearly or logarithmically, depending
upon the state of the {\tt logscale c<n>} setting. For more details of the
syntax of the range specifier, see the {\tt set xrange} command.


\subsection{data style}\indcmd{set data style}

See {\tt set style data}.


\subsection{display}\indcmd{set display}

\begin{verbatim}
set [no]display
\end{verbatim}

By default, whenever an item is added to a multiplot canvas, or an existing
item is moved or replotted, the whole multiplot is redrawn to reflect the
change.  This can be a time-consuming process when constructing large and
complex multiplot canvases, as fresh output is produced at each step. For this
reason, the {\tt set nodisplay} command is provided, which stops PyXPlot from
producing any graphical output. The {\tt set display} command can subsequently
be issued to return to normal behaviour. Scripts which produces large and
complex multiplot canvases are typically wrapped as follows:

\begin{verbatim}
set nodisplay
...
set display
refresh
\end{verbatim}


\subsection{filter}\indcmd{set filter}

\begin{verbatim}
set filter '<filename wildcard>' '<filter command>'
\end{verbatim}

The \indcmdt{set filter} allows input filter programs to be specified to allow
PyXPlot to deal with file types that are not in the plaintext format which it
ordinarily expects.  Firstly the pattern used to recognise the filenames of the
\datafile s to which the filter should apply to must be specified; the standard
wildcard characters {\tt *} and {\tt ?} may be used.  Then a filter program
should be specified, along with any necessary commandline options which should
be passed to it.  This program should take the name of the file to be filtered
as the final option on its command line, immediately following any commandline
options specified above.  It should output a suitable PyXPlot \datafile on its
standard output stream for PyXPlot to read.  For example, to filter all files
that end in {\tt .foo} through the a program called {\tt defoo} using the {\tt
--text} option one would type:

\begin{verbatim}
set filter "*.foo" "/usr/local/bin/defoo --text"
\end{verbatim}

\subsection{fontsize}\indcmd{set fontsize}

\begin{verbatim}
set fontsize <value>
\end{verbatim}

The setting {\tt fontsize} changes the size of the font used to render all text
labels which appear on graphs and multiplot canvases, including keys, axis
labels, text labels produced using the {\tt text} command, and so forth. The
value specified should be a multiplicative factor greater than zero; a value
of~{\tt 2} would cause text to be rendered at twice its normal size, and a
value of~{\tt 0.5} would cause text to be rendered at half its normal size.
The default value is one.

As an alternative, font sizes can be specified with coarser granulation
directly in the \LaTeX\ text of labels, as in the example:

\begin{verbatim}
set xlabel '\Large This is a BIG label'
\end{verbatim}


\subsection{function style}\indcmd{set function style}

See {\tt set style function}.


\subsection{grid}\indcmd{set grid}

\begin{verbatim}
set [no]grid {<axis>}
\end{verbatim}

The setting {\tt grid} controls whether a grid is placed behind graphs or not.
Issuing the command

\begin{verbatim}
set grid
\end{verbatim}

\noindent would cause a grid to be drawn with its lines connecting to the ticks
of the default axes (usually the first horizontal and vertical axes).
Conversely, issuing the command

\begin{verbatim}
set nogrid
\end{verbatim}

\noindent would remove from the plot all gridlines associated with the ticks of
any axes.  One or more axes can be specified for the {\tt set grid} command to
draw gridlines from; in such cases, gridlines are then drawn only to connect
with the ticks of the specified axes, as in the example

\begin{verbatim}
set grid x1 y3
\end{verbatim}

It is possible, though not always aesthetically pleasing, to draw gridlines
from multiple parallel axes, as in example:

\begin{verbatim}
set grid x1x2x3
\end{verbatim}


\subsection{gridmajcolour}\indcmd{set gridmajcolour}

\begin{verbatim}
set gridmajcolour <colour>
\end{verbatim}

The setting {\tt gridmajcolour} changes the colour that is used to draw the
gridlines (see the {\tt set grid} command) which are associated with the major
ticks of axes (i.e.\ major gridlines). For example:

\begin{verbatim}
set gridmajcolour purple
\end{verbatim}

\noindent would cause the major gridlines to be drawn in purple. Any of the
recognised colour names listed in Section~\ref{sec:colour_names} can be used.

See also the {\tt set gridmincolour} command.


\subsection{gridmincolour}\indcmd{set gridmincolour}

\begin{verbatim}
set gridmincolour <colour>
\end{verbatim}

The setting {\tt gridmincolour} changes the colour that is used to draw the
gridlines (see the {\tt set grid} command) which are associated with the minor
ticks of axes (i.e.\ minor gridlines). For example:

\begin{verbatim}
set gridmincolour purple
\end{verbatim}

\noindent would cause the minor gridlines to be drawn in purple. Any of the
recognised colour names listed in Section~\ref{sec:colour_names} can be used.

See also the {\tt set gridmajcolour} command.


\subsection{key}\indcmd{set key}

\begin{verbatim}
set key <position> [<xoffset>, <yoffset>]
\end{verbatim}

The setting {\tt key} determines whether legends are drawn on graphs, and if
so, where they should be located on the plots. Issuing the command

\begin{verbatim}
set key
\end{verbatim}

\noindent by itself causes legends to be drawn on graphs in the default
position, usually in the upper-right corner of the graphs. The converse action
is achieved by:

\begin{verbatim}
set nokey
\end{verbatim}

\noindent The command

\begin{verbatim}
unset key
\end{verbatim}

\noindent causes PyXPlot to revert to its default behaviour, as specified in a
configuration file, if present. A position for the key may optionally be
specified after the {\tt set key} command, as in the example:

\begin{verbatim}
set key bottom left
\end{verbatim}

Recognised positions are {\tt top}, {\tt bottom}, {\tt left}, {\tt right}, {\tt
below}, {\tt above}, {\tt outside}, {\tt xcentre} and {\tt ycentre}. In
addition, if none of these options quite achieve the desired position, a
horizontal and vertical offset may be specified as a comma-separated pair after
any of the positional keywords above.  The first value is assumed to be the
horizontal offset, and the second the vertical offset, both measured in
centimetres.  The example

\begin{verbatim}
set key bottom left 0.0, -0.5
\end{verbatim}

\noindent would display a key below the bottom left corner of the graph.


\subsection{keycolumns}\indcmd{set keycolumns}

\begin{verbatim}
set keycolumns ( <value> | auto )
\end{verbatim}

The setting {\tt keycolumns} sets how many columns the legend of a plot should
be arranged into. By default, the legends of plots are arranged into an
automatically-selected number of columns, equivalent to the behaviour achieved
by issuing the command {\tt set key\-columns auto}. However, if a different
arrangement is desired, the {\tt set keycolumns} command can be followed by any
positive integer to specify that the legend should be split into that number of
columns, as in the example:

\begin{verbatim}
set keycolumns 3
\end{verbatim}


\subsection{label}\indcmd{set label}

\begin{verbatim}
set label <label number> '<text>'
      [<system>] <x>, [<system>] <y>
      [ rotate <angle> ]
      [ with colour <colour> ]
\end{verbatim}

\noindent where {\tt <system>} may take any of the values
\newline\noindent
{\tt ( first | second | screen | graph | axis<number> )}
\vspace{5mm}

The \indcmdt{set label} is used to place text labels on graphs. The example

\begin{verbatim}
set label 1 'Hello' 0, 0
\end{verbatim}

\noindent would place a label reading `Hello' at the point $(0,0)$ on a graph,
as measured along the {\tt x}- and {\tt y}-axes.  The tag {\tt 1} immediately
following the keyword {\tt label} is an identification number and allows the
label to be subsequently removed using the {\tt unset label} command. By
default, the positional coordinates of the label are specified relative to the
first horizontal and vertical axes, but they can alternatively be specified in
any one of several coordinate systems. The coordinate system to be used is
specified as in the example:

\begin{verbatim}
set label 1 'Hello' first 0, second 0
\end{verbatim}

\noindent The name of the coordinate system to be used precedes the position
value in that system. The coordinate system {\tt first}, the default, measures
the graph using the {\tt x}- and {\tt y}-axes. {\tt second} uses the {\tt x2}-
and {\tt y2}-axes.  {\tt screen} and {\tt graph} both measure in centimetres
from the origin of the graph.  The syntax {\tt axis<n>} may also be used, to
use the $n\,$th horizontal or vertical axis; for example, {\tt axis3}:

\begin{verbatim}
set label 1 'Hello' axis3 1, axis4 1
\end{verbatim}

A rotation angle may optionally be specified after the keyword {\tt rotate}
to produce text rotated to any arbitrary angle, measured in degrees
counter-clockwise. The following example would produce upward-running text:

\begin{verbatim}
set label 1 'Hello' 1.2, 2.5 rotate 90
\end{verbatim}

By default the labels are black; however, an arbitrary colour may be specified
using the {\tt with colour} modifier.  For example,

\begin{verbatim}
set label 3 'A purple label' 0, 0 with colour purple
\end{verbatim}

\noindent would place a purple label at the origin.


\subsection{linewidth}\indcmd{set linewidth}

\begin{verbatim}
set linewidth <value>
\end{verbatim}

The \indcmdt{set linewidth} sets the default line width of the lines used to
plot datasets onto graphs using plot styles such as {\tt lines}, {\tt
errorbars}, etc. The value supplied should be a multiplicative factor relative
to the default line width; a value of~1.0 would result in lines being drawn
with their default thickness. For example, in the following statement, lines of
three times the default thickness are drawn:

\begin{verbatim}
set linewidth 3
plot sin(x) with lines
\end{verbatim}

\noindent The {\tt set linewidth} command only affects plot statements where no
line width is manually specified.


\subsection{logscale}\indcmd{set logscale}

\begin{verbatim}
set logscale {<axis>} [<base>]
\end{verbatim}

The setting {\tt logscale} causes an axis to be laid out with logarithmically,
rather than linearly, spaced intervals.  For example, issuing the command:

\begin{verbatim}
set log
\end{verbatim}

\noindent would cause all of the axes of a plot to be scaled logarithmically.
Alternatively, only one, or a selection of axes, can be set to scale
logarithmically as follows:

\begin{verbatim}
set log x1 y2
\end{verbatim}

\noindent This would cause the first horizontal and second vertical axes to be
scaled logarithmically.  Linear scaling can be restored to all axes using the
command

\begin{verbatim}
set nolog
\end{verbatim}

\noindent meanwhile the command

\begin{verbatim}
unset log
\end{verbatim}

\noindent restores axes to their default scaling, as specified in any
configuration file which may be present. Both of these commands can also be
applied to only one or a selection of axes, as in the examples

\begin{verbatim}
set nolog x1 y2
\end{verbatim}

\noindent and

\begin{verbatim}
unset log x1y2
\end{verbatim}

Optionally, a base may be specified at the end of the {\tt set logscale}
command, to produce axes labelled in logarithms of arbitrary bases.  The
default base is~10.

In addition to acting upon any combination of $x$-, $y$- and $z$-axes, the {\tt
set logscale} command may also be requested to set the {\tt c1}, {\tt c2}, {\tt
c3}, {\tt c4} and/or {\tt t} axes to scale logarithmically. The first four
of these options affect whether the colours on colour maps scale linearly or
logarithmically with input ordinate values; see the {\tt set c<n>range} command
for more details. The last of these options specifies whether parametric
functions are sampled linearly or logarithmically in the variable {\tt t}; see
the {\tt set trange} command for more details.


\subsection{multiplot}\indcmd{set multiplot}

\begin{verbatim}
set multiplot
\end{verbatim}

Issuing the command

\begin{verbatim}
set multiplot
\end{verbatim}

\noindent causes PyXPlot to enter multiplot mode, which allows many graphs to
be plotted together and displayed side-by-side. See Section~\ref{sec:multiplot}
for a full discussion of multiplot mode.


\subsection{mxtics}\indcmd{set mxtics}

See {\tt set xtics}.


\subsection{mytics}\indcmd{set mytics}

See {\tt set xtics}.


\subsection{mztics}\indcmd{set mztics}

See {\tt set ztics}.


\subsection{noarrow}\indcmd{set noarrow}

\begin{verbatim}
set noarrow [<arrow number>]
\end{verbatim}

Issuing the command

\begin{verbatim}
set noarrow
\end{verbatim}

\noindent removes all arrows configured with the {\tt set arrow} command.
Alternatively, individual arrows can be removed using commands of the form

\begin{verbatim}
set noarrow 2
\end{verbatim}

\noindent where the tag {\tt 2} is the identification number given to the arrow
to be removed when it was initially specified with the {\tt set arrow} command.


\subsection{noaxis}\indcmd{set noaxis}

\begin{verbatim}
set noaxis [ <axis> {, <axis> } ]
\end{verbatim}

The {\tt set noaxis} command is used to remove axes from graphs; it achieves
the opposite effect from the {\tt set axis} command. It should be followed by a
comma-separated lists of the axes which are to be removed from the current axis
configuration.


\subsection{nobackup}\indcmd{set nobackup}

See {\tt backup}.


\subsection{noclip}\indcmd{set noclip}

See {\tt clip}.


\subsection{nocolourkey}\indcmd{set nocolourkey}

\begin{verbatim}
set nocolourkey
\end{verbatim}

Issuing the command {\tt set nocolourkey} causes plots to be generated with no
colour scale when the \indpst{colourmap} plot style is used. See the {\tt set
colourkey} command for more details.


\subsection{nodisplay}\indcmd{set nodisplay}

See {\tt display}.


\subsection{nogrid}\indcmd{set nogrid}

\begin{verbatim}
set nogrid {<axis>}
\end{verbatim}

Issuing the command {\tt set nogrid} removes gridlines from the current plot. On
its own, the command removes all gridlines from the plot, but alternatively,
those gridlines connected to the ticks of certain axes can be selectively
removed.  The following example would remove gridlines associated with the
first horizontal axis and the second vertical axis:

\begin{verbatim}
set nogrid x1 y2
\end{verbatim}


\subsection{nokey}\indcmd{set nokey}

\begin{verbatim}
set nokey
\end{verbatim}

Issuing the command {\tt set nokey} causes plots to be generated with no legend.
See the {\tt set key} command for more details.


\subsection{nolabel}\indcmd{set nolabel}

\begin{verbatim}
set nolabel {<label number>}
\end{verbatim}

Issuing the command

\begin{verbatim}
set nolabel
\end{verbatim}

\noindent removes all text labels configured using the {\tt set label} command.
Alternatively, individual labels can be removed using the syntax:

\begin{verbatim}
set nolabel 2
\end{verbatim}

\noindent where the tag {\tt 2} is the identification number given to the label
to be removed when it was initially set using the {\tt set label} command.


\subsection{nologscale}\indcmd{set nologscale}

\begin{verbatim}
set nologscale {<axis>}
\end{verbatim}

The setting {\tt nologscale} causes an axis to be laid out with linearly,
rather than logarithmically, spaced intervals; it is equivalent to the setting
{\tt linearscale}. It is the converse of the setting {\tt logscale}.  For
example, issuing the command

\begin{verbatim}
set nolog
\end{verbatim}

\noindent would cause all of the axes of a plot to be scaled linearly.
Alternatively only one, or a selection of axes, can be set to scale linearly as
follows:

\begin{verbatim}
set nologscale x1 y2
\end{verbatim}

\noindent This would cause the first horizontal and second vertical axes to be
scaled linearly.


\subsection{nomultiplot}\indcmd{set nomultiplot}

\begin{verbatim}
set nomultiplot
\end{verbatim}

The \indcmdt{set nomultiplot} causes PyXPlot to leave multiplot mode; outside
of multiplot mode, only single graphs and vector graphics objects are displayed
at any one time, whereas in multiplot mode, galleries of plots and vector
graphics can be placed alongside one another.  See Section~\ref{sec:multiplot}
for a full discussion of multiplot mode.


\subsection{nostyle}\indcmd{set nostyle}

\begin{verbatim}
set nostyle <style number>
\end{verbatim}

The setting {\tt nostyle} deletes a numbered plot style set using the {\tt set
style} command. For example, the command

\begin{verbatim}
set nostyle 3
\end{verbatim}

\noindent would delete the third numbered plot style, if defined. See the
command {\tt set style} for more details.


\subsection{notitle}\indcmd{set notitle}

\begin{verbatim}
set notitle
\end{verbatim}

Issuing the command {\tt set notitle} will cause graphs to be produced with no
title at the top.


\subsection{noxtics}\indcmd{set noxtics}

\begin{verbatim}
set no<axis>tics
\end{verbatim}

This command causes graphs to be produced with no major tick marks along the
specified axis. For example, the {\tt set noxtics} command removes all major
tick marks from the {\tt x}-axis.


\subsection{noytics}\indcmd{set noytics}

Similar to the {\tt set noxtics} command, but acts on vertical axes.


\subsection{noztics}\indcmd{set noztics}

Similar to the {\tt set noxtics} command, but acts on $z$-axes.


\subsection{numerics}\indcmd{set numerics}

\begin{verbatim}
set numerics [ ( complex | real ) ] [ errors ( explicit | quiet) ]
    [ display ( latex | natural | typeable) ]
    [ sigfig <precision> ]
\end{verbatim}

The \indcmdt{set numerics} is used to adjust the way in which calculations are
carried out and numerical quantities are displayed:

\begin{itemize}

\item The option {\tt complex} causes PyXPlot to switch from performing real
arithmetic (default) to performing complex arithmetic. The option {\tt real}
causes any calculations which return results with finite imaginary components
to generate errors.

\item The option {\tt errors} controls how numerical errors such as divisions
by zero, numerical overflows, and the querying functions outside of the domains
in which they are defined, are communicated to the user.  The option {\tt
explicit} (default) causes an error message to be displayed on the terminal
whenever a calculation causes an error.  The option {\tt quiet} causes such
calculations to silently generate a {\tt nan} (not a number) result. The latter
is especially useful when, for example, plotting an expression with the
ordinate axis range set to extend outside the domain in which that expression
returns a well-defined real result; it suppresses the error messages which
might otherwise result from PyXPlot's attempts to evaluate the expression in
those domains where its result is undefined. The option {\tt nan} is a synonym
for {\tt quiet}.

\item The setting {\tt display} changes the format in which numbers are
displayed on the terminal.  Setting the option to {\tt typeable} causes the
numbers to be printed in a form suitable for pasting back into PyXPlot
commands.  The setting {\tt latex} causes \LaTeX-compatible output to be
generated.  The setting {\tt natural} generates concise, human-readable output
which has neither of the above properties.

\item The setting {\tt sigfig} changes the number of significant figures to
which numbers are displayed on the PyXPlot terminal.  Regardless of the value
set, all calculations are internally carried out and stored at double
precision, accurate to around~16 significant figures.

\end{itemize}


\subsection{origin}\indcmd{set origin}

\begin{verbatim}
set origin <x>, <y>
\end{verbatim}

The \indcmdt{set origin} is used to set the location of the bottom-left corner
of the next graph to be placed on a multiplot canvas.  For example, the
command

\begin{verbatim}
set origin 3,5
\end{verbatim}

\noindent would cause the next graph to be plotted with its bottom-left corner
at position $(3,5)$ centimetres on the multiplot canvas. Alternatively, either
of the coordinates may be specified as quantities with physical units of
length, such as {\tt unit(35*mm)}.  The {\tt set origin} command is of little
use outside of multiplot mode.


\subsection{output}\indcmd{set output}

\begin{verbatim}
set output '<filename>'
\end{verbatim}

The setting {\tt output} controls the name of the file that is produced for
non-interactive terminals ({\tt postscript}, {\tt eps}, {\tt jpeg}, {\tt gif}
and {\tt png}).  For example,

\begin{verbatim}
set output 'myplot.eps'
\end{verbatim}

\noindent causes the output to be written to the file {\tt myplot.eps}.


\subsection{palette}\indcmd{set palette}

\begin{verbatim}
set palette <colour> {, <colour>}
\end{verbatim}

PyXPlot has a palette of colours which it assigns sequentially to datasets when
colours are not manually assigned. This is also the palette to which reference
is made if the user issues a command such as

\begin{verbatim}
plot sin(x) with colour 5
\end{verbatim}

\noindent requesting the fifth colour from the palette. By default, this palette
contains a range of distinctive colours. However, the user can choose to
substitute his own list of colours using the {\tt set palette} command. It
should be followed by a comma-separated list of colour names, for example:

\begin{verbatim}
set palette red,green,blue
\end{verbatim}

\noindent If, after issuing this command, the following plot statement were to
be executed:

\begin{verbatim}
plot sin(x), cos(x), tan(x), exp(x)
\end{verbatim}

\noindent the first function would be plotted in red, the second in green, and
the third in blue. Upon reaching the fourth, the palette would cycle back to
red.

Any of the recognised colour names listed in Section~\ref{sec:colour_names} can
be used.


\subsection{papersize}\indcmd{set papersize}

\begin{verbatim}
set papersize ( <named size> | <height>,<width> )
\end{verbatim}

The setting {\tt papersize} changes the size of output produced by the {\tt
postscript} terminal, and whenever the {\tt enlarge} terminal option is set
(see the {\tt set terminal} command). This can take the form of either a
recognised paper size name -- a list of these is given in
Appendix~\ref{ch:paper_sizes} -- or as a (height, width) pair of values, both
measured in millimetres. The following examples demonstrate these
possibilities:

\begin{verbatim}
set papersize a4
set papersize letter
set papersize 200,100
\end{verbatim}


\subsection{pointlinewidth}\indcmd{set pointlinewidth}

\begin{verbatim}
set pointlinewidth <value>
\end{verbatim}

The setting {\tt pointlinewidth} changes the width of the lines that are used
to plot \datapoint s.  For example,

\begin{verbatim}
set pointlinewidth 20
\end{verbatim}

\noindent would cause points to be plotted with lines~20 times the default
thickness.  The setting {\tt pointlinewidth} can be abbreviated as {\tt plw}.


\subsection{pointsize}\indcmd{set pointsize}

\begin{verbatim}
set pointsize <value>
\end{verbatim}

The setting {\tt pointsize} changes the size at which points are drawn,
relative to their default size. It should be followed by a single value which
can be any positive multiplicative factor. For example,

\begin{verbatim}
set pointsize 1.5
\end{verbatim}

\noindent would cause points to be drawn at~1.5 times their default size.


\subsection{preamble}\indcmd{set preamble}

\begin{verbatim}
set preamble <text>
\end{verbatim}

The setting {\tt preamble} changes the text of the preamble that is passed to
\LaTeX\ prior to the rendering of each text item on the current multiplot
canvas.  This allows, for example, different packages to be loaded by default
and user-defined macros to be set up, as in the examples:

\begin{verbatim}
set preamble \usepackage{marvosym}
set preamble \def\degrees{$^\circ$}
\end{verbatim}


%\subsection{projection}\indcmd{set projection}
%
%\begin{verbatim}
%set projection flat
%\end{verbatim}
%
%The {\tt set projection} command will change between different projections of
%plots in a future version of PyXPlot.  Currently only the {\tt flat} option is
%supported.


\subsection{samples}\indcmd{set samples}

\begin{verbatim}
set samples [<value>]
            [grid <x_samples> [x] <y_samples>]
            [interpolate ( InverseSquare |
                           MonaghanLattanzio |
                           NearestNeighbour ) ]
\end{verbatim}

The setting {\tt samples} determines the number of values along the ordinate
axis at which functions are evaluated when they are plotted. For example, the
command

\begin{verbatim}
set samples 100
\end{verbatim}

\noindent would cause functions to be evaluated at 100~points along the
ordinate axis.  Increasing this value will cause functions to be plotted more
smoothly, but also more slowly, and the PostScript files generated will also be
larger. When functions are plotted with the {\tt points} plot style, this
setting controls the number of points plotted.

After the keyword {\tt grid} may be specified the dimensions of the
two-dimensional grid of samples used in the \indpst{colourmap} and
\indpst{surface} plot styles, and internally when calculating the contours to
be plotted in the \indpst{contourmap} plot style. If a {\tt *} is given in
place of either of the dimensions, then the same number of samples as are
specified in {\tt <value>} are taken.

After the keyword {\tt interpolate}, the method used for interpolating
non-gridded two-dimensional data onto the above-mentioned grid may be
specified. The available options are {\tt Inverse\-Square}, {\tt
Monag\-han\-Lat\-tan\-zio} and {\tt Nearest\-Neigh\-bour}.


\subsection{seed}\indcmd{set seed}

\begin{verbatim}
set seed <value>
\end{verbatim}

The \indcmdt{set seed} sets the seed used by all of those mathematical
functions which generate random samples from probability distributions.  This
allows repeatable sequences of pseudo-random numbers to be generated.  Upon
initialisation, PyXPlot returns the sequence of random numbers obtained after
issuing the command {\tt set seed~0}.


\subsection{size}\indcmd{set size}

\begin{verbatim}
set size [<width>]
         [( ratio <ratio> | noratio  | square)]
         [(zratio <ratio> | nozratio )]
\end{verbatim}

The setting {\tt size} is used to set the width or aspect ratio of the next
graph to be generated. If a width is specified, then it may either take the
form of a dimensionless number implicitly measured in centimetres, or a
quantity with physical dimensions of length such as {\tt unit(50*mm)}.

When the keyword {\tt ratio} is specified, it should be followed by the ratio
of the graph's height to its width, i.e.\ of the length of its $y$-axes to that
of its $x$-axes. The keyword {\tt noratio} returns the aspect ratio to its
default value of the golden ratio, and the keyword {\tt square} sets the aspect
ratio to one.

When the keyword {\tt zratio} is specified, it should be followed by the ratio
of the length of three-dimensional graphs' $z$-axes to that of their $x$-axes.
The keyword {\tt nozratio} returns this aspect ratio to its default value of
the golden ratio.


\subsubsection{noratio}\index{set size command!noratio modifier@{\tt noratio} modifier}

\begin{verbatim}
set size noratio
\end{verbatim}

Executing the command

\begin{verbatim}
set size noratio
\end{verbatim}

\noindent resets PyXPlot to produce plots with its default aspect ratio, which
is the golden ratio. Other aspect ratios can be set with the {\tt set size
ratio} command.


\subsubsection{ratio}\index{set size command!ratio modifier@{\tt ratio} modifier}

\begin{verbatim}
set size ratio <ratio>
\end{verbatim}

This command sets the aspect ratio of plots produced by PyXPlot.  The height of
resulting plots will equal the plot width, as set by the {\tt set width}
command, multiplied by this aspect ratio.  For example,

\begin{verbatim}
set size ratio 2.0
\end{verbatim}

\noindent would cause PyXPlot to produce plots that are twice as high as they
are wide.  The default aspect ratio which PyXPlot uses is a golden ratio of
$2/(1+\sqrt{5})$.


\subsubsection{square}\index{set size command!square modifier@{\tt square} modifier}

\begin{verbatim}
set size square
\end{verbatim}

This command sets PyXPlot to produce square plots, i.e.\ with unit aspect
ratio. Other aspect ratios can be set with the {\tt set size ratio} command.


\subsection{style}\indcmd{set style}

\begin{verbatim}
set style <style number> {<style option>}
\end{verbatim}

At times, the string of style keywords following the {\tt with} modifier in
plot commands can grow rather unwieldy in its length. For clarity, frequently
used plot styles can be stored as numbered plot {\tt styles}. The syntax for
setting a numbered plot style is:

\begin{verbatim}
set style 2 points pointtype 3
\end{verbatim}

\noindent where the {\tt 2} is the identification number of the plot style.
In a subsequent plot statement, this line style can be recalled as follows:

\begin{verbatim}
plot sin(x) with style 2
\end{verbatim}


\subsection{style data | style function}\indcmd{set style data}\indcmd{set style function}

\begin{verbatim}
set style { data | function } {<style option>}
\end{verbatim}

The {\tt set style data} command affects the default style with which data from
files is plotted.  Likewise the {\tt set style function} command changes the
default style with which functions are plotted.  Any valid style modifier which
can follow the keyword {\tt with} in the {\tt plot} command can be used.  For
example, the commands

\begin{verbatim}
set style data points
set style function lines linestyle 1
\end{verbatim}

\noindent would cause \datafile s to be plotted, by default, using points and
functions using lines with the first defined line style.


\subsection{terminal}\indcmd{set terminal}

\begin{verbatim}
set terminal ( X11_SingleWindow | X11_MultiWindow | X11_Persist |
               bmp | eps | gif | jpeg | pdf | png | postscript |
               svg | tiff )
             ( colour | color | monochrome )
             ( dpi <value> )
             ( portrait | landscape )
             ( invert | noinvert )
             ( transparent | solid )
             ( antialias | noantialias )
             ( enlarge | noenlarge )
\end{verbatim}

The \indcmdt{set terminal} controls the graphical format in which PyXPlot
renders plots and multiplot canvases, for example configuring whether it should
output plots to files or display them in a window on the screen. Various
options can also be set within many of the graphical formats which PyXPlot
supports using this command.

The following graphical formats are supported:  {\tt X11\_\-Single\-Window},
{\tt X11\_\-Multi\-Window}, {\tt X11\_\-Persist}, {\tt bmp}, {\tt eps}, {\tt
gif}, {\tt jpeg}, {\tt pdf}, {\tt png}, {\tt postscript}, {\tt
svg}\footnote{The {\tt svg} output terminal is experimental and may be
unstable. It relies upon the use of the {\tt svg} output device in Ghostscript,
which may not be present on all systems.}, {\tt tiff}.  To select one of these
formats, simply type the name of the desired format after the {\tt set
terminal} command. To obtain more details on each, see the subtopics below.
The following settings, which can also be typed following the {\tt set
terminal} command, are used to change the options within some of these graphic
formats: {\tt colour}, {\tt monochrome}, {\tt dpi}, {\tt portrait}, {\tt
landscape}, {\tt invert}, {\tt noinvert}, {\tt transparent}, {\tt solid}, {\tt
enlarge}, {\tt noenlarge}. Details of each of these can be found below.


\subsubsection{antialias}\index{set terminal command!antialias modifier@{\tt antialias} modifier}

The {\tt antialias} terminal option causes plots produced with the bitmap
terminals (i.e.\ {\tt bmp}, {\tt gif}, {\tt jpeg}, {\tt png} and {\tt tiff}) to be
antialiased; this is the default behaviour.


\subsubsection{bmp}\index{set terminal command!bmp modifier@{\tt gif} modifier}

The {\tt bmp} terminal renders output as Windows bitmap images. The filename to
which output is to be sent should be set using the {\tt set output} command;
the default is {\tt pyxplot.bmp}. The number of dots per inch used can be
changed using the {\tt dpi} option. The {\tt invert} option may be used to produce an
image with inverted colours.


\subsubsection{colour}\index{set terminal command!colour modifier@{\tt colour} modifier}

The {\tt colour} terminal option causes plots to be produced in colour; this is
the default behaviour.


\subsubsection{color}\index{set terminal command!color modifier@{\tt color} modifier}

The {\tt color} terminal option is the US-English equivalent of {\tt colour}.


\subsubsection{dpi}\index{set terminal command!dpi modifier@{\tt dpi} modifier}

When PyXPlot is set to produce bitmap graphics output, using the {\tt bmp},
{\tt gif}, {\tt jpg} or {\tt png} terminals, the setting {\tt dpi} changes the
number of dots per inch with which these graphical images are produced. That is
to say, it changes the image resolution of the output images. For example,

\begin{verbatim}
set terminal dpi 100
\end{verbatim}

\noindent sets the output to a resolution of~100 dots per inch. Higher DPI
values yield better quality images, but larger file sizes.


\subsubsection{enlarge}\index{set terminal command!enlarge modifier@{\tt enlarge} modifier}

The {\tt enlarge} terminal option causes plots and multiplot canvases to be
enlarged or shrunk to fit within the margins of the currently selected paper
size. It is especially useful when using the {\tt postscript} terminal, as it
allows for the production of immediately-printable output.


\subsubsection{eps}\index{set terminal command!eps modifier@{\tt eps} modifier}

Sends output to Encapsulated PostScript ({\tt eps}) files.  The filename to
which output should be sent can be set using the {\tt set output} command; the
default is {\tt pyxplot.eps}.  This terminal produces images suitable for
including in, for example, \LaTeX\ documents.


\subsubsection{gif}\index{set terminal command!gif modifier@{\tt gif} modifier}

The {\tt gif} terminal renders output as gif images. The filename to which
output should be sent can be set using the {\tt set output} command; the
default is {\tt pyxplot.gif}. The number of dots per inch used can be changed
using the {\tt dpi} option. Transparent gifs can be produced with the {\tt
transparent} option. The {\tt invert} option may be used to produce an image
with inverted colours.


\subsubsection{invert}\index{set terminal command!invert modifier@{\tt invert} modifier}

The {\tt invert} terminal option causes the bitmap terminals (i.e.\ {\tt bmp},
{\tt gif}, {\tt jpeg}, {\tt png} and {\tt tiff}) to produce output with
inverted colours.


\subsubsection{jpeg}\index{set terminal command!jpeg modifier@{\tt jpeg} modifier}

The {\tt jpeg} terminal renders output as jpeg images. The filename to which
output should be sent can be set using the {\tt set output} command; the
default is {\tt pyxplot.jpg}.  The number of dots per inch used can be changed
using the {\tt dpi} option. The {\tt invert} option may be used to produce an
image with inverted colours.


\subsubsection{landscape}\index{set terminal command!landscape modifier@{\tt landscape} modifier}

The {\tt landscape} terminal option causes PyXPlot's output to be displayed in
rotated orientation.  This can be useful for fitting graphs onto sheets of
paper, but is generally less useful for plotting things on screen.


\subsubsection{monochrome}\index{set terminal command!monochrome modifier@{\tt monochrome} modifier}

The {\tt monochrome} terminal option causes plots to be rendered in black and
white. This changes the default behaviour of the {\tt plot} command to be
optimised for monochrome display, and so, for example, different dash styles
are used to differentiate between lines on plots with several datasets.


\subsubsection{noantialias}\index{set terminal command!noantialias modifier@{\tt noantialias} modifier}

The {\tt noantialias} terminal option causes plots produced with the bitmap
terminals (i.e.\ {\tt bmp}, {\tt gif}, {\tt jpeg}, {\tt png} and {\tt tiff})
not to be antialiased. This can be useful when making plots which will
subsequently have regions cut out and made transparent.


\subsubsection{noenlarge}\index{set terminal command!noenlarge modifier@{\tt noenlarge} modifier}

The {\tt noenlarge} terminal option causes the output not to be scaled to fit
within the margins of the currently-selected papersize. This is the opposite of
{\tt enlarge} option.


\subsubsection{noinvert}\index{set terminal command!noinvert modifier@{\tt noinvert} modifier}

The {\tt noinvert} terminal option causes the bitmap terminals (i.e.\ {\tt
gif}, {\tt jpeg}, {\tt png}) to produce normal output without inverted colours.
This is the opposite of the {\tt inverse} option.


\subsubsection{pdf}\index{set terminal command!pdf modifier@{\tt pdf}
modifier}

The {\tt pdf} terminal renders output in Adobe's Portable Document Format
(PDF).


\subsubsection{png}\index{set terminal command!png modifier@{\tt png} modifier}

The {\tt png} terminal renders output as png images. The filename to which
output should be sent can be set using the {\tt set output} command; the
default is {\tt pyxplot.png}. The number of dots per inch used can be changed
using the {\tt dpi} option. Transparent pngs can be produced with the {\tt
transparent} option. The {\tt invert} option may be used to produce an image
with inverted colours.


\subsubsection{portrait}\index{set terminal command!portrait modifier@{\tt portrait} modifier}

The {\tt portrait} terminal option causes PyXPlot's output to be displayed in
upright (normal) orientation; it is the converse of the {\tt landscape} option.


\subsubsection{postscript}\index{set terminal command!postscript modifier@{\tt postscript} modifier}

The {\tt postscript} terminal renders output as PostScript files. The filename
to which output should be sent can be set using the {\tt set output} command;
the default is {\tt pyxplot.ps}.  This terminal produces non-encapsulated
PostScript suitable for sending directly to a printer; it should not be used
for producing images to be embedded in documents, for which the {\tt eps}
terminal should be used.


\subsubsection{solid}\index{set terminal command!solid modifier@{\tt solid} modifier}

The {\tt solid} option causes the {\tt gif} and {\tt png} terminals to produce
output with a non-transparent background, the converse of {\tt transparent}.


\subsubsection{transparent}\index{set terminal command!transparent modifier@{\tt transparent} modifier}

The {\tt transparent} terminal option causes the {\tt gif} and {\tt png}
terminals to produce output with a transparent background.


\subsubsection{X11\_multiwindow}\index{set terminal command!X11\_multiwindow modifier@{\tt X11\_multiwindow} modifier}

The {\tt X11\_multiwindow} terminal displays plots on the screen in X11
windows. Each time a new plot is generated it appears in a new window, and the
old plots remain visible.  As many plots as may be desired can be left on the
desktop simultaneously. When PyXPlot exits, however, all of the windows are
closed.

\subsubsection{X11\_persist}\index{set terminal command!X11\_persist
modifier@{\tt X11\_persist} modifier}

The {\tt X11\_persist} terminal displays plots on the screen in X11 windows.
Each time a new plot is generated it appears in a new window, and the old plots
remain visible.  When PyXPlot is exited the windows remain in place until they
are closed manually.

\subsubsection{X11\_singlewindow}\index{set terminal command!X11\_singlewindow modifier@{\tt X11\_singlewindow} modifier}

The {\tt X11\_singlewindow} terminal displays plots on the screen in X11
windows. Each time a new plot is generated it replaces the old one, preventing
the desktop from becoming flooded with old plots. This terminal is the default
when running interactively.


\subsection{textcolour}\indcmd{set textcolour}

\begin{verbatim}
set textcolour <colour>
\end{verbatim}

The setting {\tt textcolour} changes the default colour of all text displayed
on plots or multiplot canvases.  For example,

\begin{verbatim}
set textcolour red
\end{verbatim}

\noindent causes all text labels, including the labels on graph axes and
legends, etc.\ to be rendered in red. Any of the recognised colour names listed
in Section~\ref{sec:colour_names} can be used; colours can also be referred to
numerically with reference to the current palette.


\subsection{texthalign}\indcmd{set texthalign}

\begin{verbatim}
set texthalign ( left | centre | right )
\end{verbatim}

The setting {\tt texthalign} controls how text labels are justified
horizontally with respect to their specified positions, acting both upon labels
placed on plots using the {\tt set label} command, and upon text items created
using the {\tt text} command. Three options are available:

\begin{verbatim}
set texthalign left
set texthalign centre
set texthalign right
\end{verbatim}


\subsection{textvalign}\indcmd{set textvalign}

\begin{verbatim}
set textvalign ( bottom | centre | top )
\end{verbatim}

The setting {\tt textvalign} controls how text labels are justified vertically
with respect to their specified positions, acting both upon labels placed on
plots using the {\tt set label} command, and upon text items created using the
{\tt text} command. Three options are available:

\begin{verbatim}
set textvalign bottom
set textvalign centre
set textvalign top
\end{verbatim}


\subsection{title}\indcmd{set title}

\begin{verbatim}
set title '<title>'
\end{verbatim}

The setting {\tt title} can be used to set a title for a plot, to be displayed
above it.  For example, the command:

\begin{verbatim}
set title 'foo'
\end{verbatim}

\noindent would cause a title `foo' to be displayed above a graph. The easiest
way to remove a title, having set one, is using the command:

\begin{verbatim}
set notitle
\end{verbatim}


\subsection{trange}\indcmd{set trange}

\begin{verbatim}
set trange <range>
\end{verbatim}

The {\tt set trange} command changes the range of the free parameter {\tt t}
used when generating parametric plots.  For more details of the syntax of the
range specifier, see the {\tt set xrange} command. Note that {\tt t} is not
allowed to autoscale, and so the {\tt *} character is not permitted in the
specified range.


\subsection{unit}\indcmd{set unit}

\begin{verbatim}
set unit [ angle ( dimensionless | nodimensionless ) ]
         [ of <dimension> <unit> ]
         [ scheme <unit scheme> ]
         [ preferred <unit> ]
         [ nopreferred <unit> ]
         [ display ( full | abbreviated | prefix | noprefix ) ]
\end{verbatim}

The \indcmdt{set unit} controls how quantities with physical units are
displayed by PyXPlot. The \indcmdt{set unit scheme} provides the most general
configuration option, allowing one of several {\it units
schemes}\index{units!unit schemes} to be selected, each of which comprises a
list of units which are deemed to be members of that particular scheme. For
example, in the CGS unit scheme\index{CGS units}\index{units!CGS}, all lengths
are displayed in centimetres, all masses are displayed in grammes, all energies
are displayed in ergs, and so forth.  In the imperial unit
scheme\index{imperial units}\index{units!imperial}, quantities are displayed in
British imperial units -- inches, pounds, pints, and so forth -- and in the US
unit scheme, US customary units are used. The available schemes are: {\tt
ancient}, {\tt cgs}, {\tt imperial}, {\tt planck}, {\tt si}, and {\tt us}.

To fine-tune the unit used to display quantities with a particular set of
physical dimensions, the {\tt set unit of} form of the command should be used.
For example, the following command would cause all lengths to be displayed in
inches:

\begin{verbatim}
set unit of length inch
\end{verbatim}

The \indcmdt{set unit preferred} offers a slightly more flexible way of
achieving the same result. Whereas the \indcmdt{set unit of} can only operate
on named quantities such as lengths and powers, and cannot act upon compound
units such as {\tt W/Hz}, the \indcmdt{set unit preferred} can act upon any
unit or combination of units, as in the examples:
\begin{verbatim}
set unit preferred parsec
set unit preferred W/Hz
set unit preferred N*m
\end{verbatim}
The latter two examples are particularly useful when working with spectral
densities (powers per unit frequency) or torques (forces multiplied by
distances). Unfortunately, both of these units are dimensionally equal to
energies, and so are displayed by PyXPlot in Joules by default. The above
statement overrides such behaviour. Having set a particular unit to be
preferred, this can be unset as in the following example:
\begin{verbatim}
set unit nopreferred parsec
\end{verbatim}

By default, units are displayed in their abbreviated forms, for example {\tt A}
instead of {\tt amperes} and {\tt W} instead of {\tt watts}. Furthermore, SI
prefixes such as milli- and kilo- are applied to SI units where they are
appropriate. Both of these behaviours can be turned on or off, in the former
case with the commands

\begin{verbatim}
set unit display abbreviated
set unit display full
\end{verbatim}

\noindent and in the latter case using the following pair of commands:

\begin{verbatim}
set unit display prefix
set unit display noprefix
\end{verbatim}


\subsection{view}\indcmd{set view}

\begin{verbatim}
set view <theta>, <phi>
\end{verbatim}

The \indcmdt{set view} is used to specify the angle from which
three-dimensional plots are viewed. It should be followed by two angles, which
can either be expressed in degrees, as dimensionless numbers, or as quantities
with physical units of angle:
\begin{verbatim}
set view 60,30

set unit angle nodimensionless
set view unit(0.1*rev),unit(2*rad)
\end{verbatim}
The orientation $(0,0)$ corresponds to having the $x$-axis horizontal, the
$z$-axis vertical, and the $y$-axis directed into the page. The first angle
supplied to the {\tt set view} command rotates the plot in the $(x,y)$ plane,
and the second angle tips the plot up in the plane containing the $z$-axis and
the normal to the user's two-dimensional display.


\subsection{viewer}\indcmd{set viewer}

\begin{verbatim}
set viewer ( auto | <command> )
\end{verbatim}

The \indcmdt{set viewer} is used to select which external PostScript viewing
application is used to display PyXPlot output on screen in the {\tt X11}
terminals. If the option {\tt auto} is selected, then either \ghostview\ or
{\tt ggv} is used, if installed. Alternatively, any other application such as
{\tt evince} or {\tt okular} can be selected by name, providing it is installed
in within your shell's search path or an absolute path is provided, as in the
examples:

\begin{verbatim}
set viewer evince
set viewer /usr/bin/okular
\end{verbatim}

\noindent Additional commandline switches may also be provided after the name
of the application to be used, as in the example

\begin{verbatim}
set viewer gv --grayscale
\end{verbatim}


\subsection{width}\indcmd{set width}

\begin{verbatim}
set width <value>
\end{verbatim}

The setting {\tt width} is used to set the width of the next graph to be
generated. The width is specified either as a dimensionless number
implicitly measured in centimetres, or as a quantity with physical dimensions
of length such as {\tt unit(50*mm)}.


\subsection{xformat}\indcmd{set xformat}

\begin{verbatim}
set <axis>format ( auto | '<format>' )
      ( horizontal | vertical | rotate <angle> )
\end{verbatim}

By default, the major tick marks along axes are labelled with representations
of the ordinate values at each point, each accurate to the number of
significant figures specified using the \indcmdt{set numerics sigfig}. These
labels may appear as decimals, such as $3.142$, in scientific notion, as in
$3\times10^8$, or, on logarithmic axes where a base has been specified for the
logarithms, using syntax such as\footnote{Note that the {\tt x} axis must be
referred to as {\tt x1} here to distinguish this statement from {\tt set log
x2}.}
\begin{verbatim}
set log x1 2
\end{verbatim}
in a format such as $1.5\times2^8$.

The \indcmdt{set xformat} -- together with its companions such as {\tt set
yformat} -- is used to manually specify an explicit format for the axis labels
to take, as demonstrated by the following pair of examples:
\begin{verbatim}
set xformat "%.2f"%(x)
set yformat "%s$^\prime$"%(y/unit(feet))
\end{verbatim}
The first example specifies that values should be displayed to two
decimal places along the {\tt x}-axis; the second specifies that distances should
be displayed in feet along the {\tt y}-axis. Note that the dummy variable used to
represent the represented value is {\tt x}, {\tt y} or {\tt z} depending upon the
direction of the axis, but that the dummy variable used in the {\tt set
x2format} command is still {\tt x}. The following pair of examples both have
the equivalent effect of returning the {\tt x2}-axis to its default system of
tick labels:
\begin{verbatim}
set x2format auto
set x2format "%s"%(x)
\end{verbatim}

The following example specifies that ordinate values should be displayed as
multiples of $\pi$:
\begin{verbatim}
set xformat "%s$\pi$"%(x/pi)
plot [-pi:2*pi] sin(x)
\end{verbatim}

Note that where possible, PyXPlot intelligently changes the positions along
axes where it places the ticks to reflect significant points in the chosen
labelling system.  The extent to which this is possible depends upon the format
string supplied. It is generally easier when continuous-varying numerical
values are substituted into strings, rather than discretely-varying values or
strings.

\subsection{xlabel}\indcmd{set xlabel}

\begin{verbatim}
set <axis>label '<text>' [ rotate <angle> ]
\end{verbatim}

The setting {\tt xlabel} sets the label which should be written along the {\tt
x}-axis.  For example,

\begin{verbatim}
set xlabel '$x$'
\end{verbatim}

\noindent sets the label on the {\tt x}-axis to read `$x$'.  Labels can be
placed on higher numbered axes by inserting their number after the `{\tt x}';
for example,

\begin{verbatim}
set x10label 'foo'
\end{verbatim}

\noindent would label the tenth horizontal axis. Similarly, labels can be
placed on vertical axes as follows:

\begin{verbatim}
set ylabel '$y$'
set y2label 'foo'
\end{verbatim}

An optional rotation angle may be specified to rotate axis labels clockwise by
arbitrary angles. The angle should be specified either as a dimensionless
number of degrees, or as a quantity with physical dimensions of angle.

\subsection{xrange}\indcmd{set xrange}

\begin{verbatim}
set <axis>range <range> [reverse]
\end{verbatim}

The setting {\tt xrange} controls the range of values spanned by the {\tt
x}-axes of plots.  For function plots, this is also the domain across which the
function will be evaluated.  For example,

\begin{verbatim}
set xrange [0:10]
\end{verbatim}

\noindent sets the first horizontal axis to run from~0 to~10.  Higher numbered
axes may be referred to be inserting their number after the {\tt x}; the ranges
of vertical axes may similarly be set by replacing the {\tt x} with a {\tt y}.
Hence,

\begin{verbatim}
set y23range [-5:5]
\end{verbatim}

\noindent sets the range of the 23rd vertical axis to run from~$-5$ to~5.  To
request a range to be automatically scaled an asterisk can be used.  The
following command would set the {\tt x}-axis to have an upper limit of 10, but
does not affect the lower limit; its range remains at its previous setting:

\begin{verbatim}
set xrange [:10][*:*]
\end{verbatim}

The keyword {\tt reverse} is used to indicate that the two limits of an axis
should be swapped. This is useful for setting auto-scaling axes to be displayed
running from right to left, or from top to bottom.


\subsection{xtics}\indcmd{set xtics}

\begin{verbatim}
set [m]<axis>tics
    [ ( axis | border | inward | outward | both ) ]
    [ ( autofreq
          | [<minimum>,] <increment> [, <maximum>]
          | \( { '<label>' <position> } \)
         ] )
\end{verbatim}

By default, PyXPlot places a series of tick marks at significant points along
each axis, with the most significant points being labelled.  Labelled tick
marks are termed {\it major} ticks, and unlabelled tick marks are termed {\it
minor} ticks.  The position and appearance of the major ticks along the {\tt
x}-axis can be configured using the \indcmdt{set xtics}; the corresponding
{\tt set mxtics} command configures the appearance of the minor ticks along the
{\tt x}-axis. Analogous commands such as {\tt set ytics} and {\tt set mx2tics}
configure the major and minor ticks along other axes.

The keywords \indkeyt{inward}, \indkeyt{outward} and \indkeyt{both} are used to
configure how the ticks appear -- whether they point inward, towards the plot,
as is default, or outwards towards the axis labels, or in both directions.  The
keyword \indkeyt{axis} is an alias for \indkeyt{inward}, and \indkeyt{border}
is an alias for \indkeyt{outward}.

The remaining options are used to configure where along the axis ticks are
placed. If a series of comma-separated values {\tt <minimum>, <increment>,
<maximum>} are specified, then ticks are placed at evenly spaced intervals
between the specified limits. The {\tt <minimum>} and {\tt <maximum>} values
are optional; if only one value is specified then it is taken to be the step
size between ticks. If two values are specified, then the first is taken to be
{\tt <minimum>}. In the case of logarithmic axes, {\tt <increment>} is applied
as a multiplicative step size.

Alternatively, if a bracketed list of quoted tick labels and tick positions are
provided, then ticks can be placed on an axis manually and each given its own
textual label. The quoted tick labels may be omitted, in which case they are
automatically generated:
\begin{verbatim}
set xtics ("a" 1, "b" 2, "c" 3)
set xtics (1,2,3)
\end{verbatim}
The keyword \indkeyt{autofreq} overrides any manual selection of ticks which
may have been placed on an axis and resumes the automatic placement of ticks
along it. The \indcmdt{show xtics}, together with its companions such as {\tt
show x2tics} and {\tt show ytics}, is used to query the current ticking
options. The \indcmdt{set noxtics} is used to stipulate that no ticks should
appear along a particular axis:

\begin{verbatim}
set noxtics
show xtics
\end{verbatim}


\subsection{yformat}\indcmd{set yformat}

See {\tt xformat}.


\subsection{ylabel}\indcmd{set ylabel}

See {\tt xlabel}.


\subsection{yrange}\indcmd{set yrange}

See {\tt xrange}.


\subsection{ytics}\indcmd{set ytics}

See {\tt xtics}.


\subsection{zformat}\indcmd{set zformat}

See {\tt xformat}.


\subsection{zlabel}\indcmd{set zlabel}

See {\tt xlabel}.


\subsection{zrange}\indcmd{set zrange}

See {\tt xrange}.


\subsection{ztics}\indcmd{set ztics}

See {\tt xtics}.


\section{show}\indcmd{show}

\begin{verbatim}
show { all | axes | functions | settings | units
       | userfunctions | variables | <parameter> }
\end{verbatim}

The \indcmdt{show} displays the present state of parameters which can be set
with the {\tt set} command. For example,

\begin{verbatim}
show pointsize
\end{verbatim}

\noindent displays the currently set point size.

Details of the various parameters which can be queried can be found under the
{\tt set} command; any keyword which can follow the {\tt set} command can also
follow the {\tt show} command.

In addition, {\tt show all} shows a complete list of the present values of all
of PyXPlot's configurable parameters.  The command {\tt show settings} shows
all of these parameters, but does not list the currently-configured variables,
functions and axes. {\tt show axes} shows the configuration states of all graph
axes. {\tt show variables} lists all of the currently defined variables. And
finally, {\tt show functions} lists all of the current user-defined functions.


\section{solve}\indcmd{solve}

\begin{verbatim}
solve <equation> {, <equation>}
    via <variable> {, <variable>}
\end{verbatim}

The \indcmdt{solve} can be used to solve simple systems of one or more
equations numerically. It takes as its arguments a comma-separated list of the
equations which are to be solved, and a comma-separated list of the variables
which are to be found. The latter should be prefixed by the word {\tt
via}, to separate it from the list of equations.

Note that the time taken by the solver dramatically increases with the number
of variables which are simultaneously found, whereas the accuracy achieved
simultaneously decreases. The following example solves a simple pair of
simultaneous equations of two variables:

\begin{verbatim}
pyxplot> solve x+y=10, x-y=3 via x,y
pyxplot> print x
6.5
pyxplot> print y
3.5
\end{verbatim}

\noindent No output is returned to the terminal if the numerical solver
succeeds, otherwise an error message is displayed. If any of the fitting
variables are already defined prior to the {\tt solve} command's being called,
their values are used as initial guesses, otherwise an initial guess of unity
for each fitting variable is assumed. Thus, the same \indcmdt{solve} returns
two different values in the following two cases:

\begin{verbatim}
pyxplot> x= # Undefine x
pyxplot> solve cos(x)=0 via x
pyxplot> print x/pi
0.5
pyxplot> x=10
pyxplot> solve cos(x)=0 via x
pyxplot> print x/pi
3.5
\end{verbatim}

\noindent In cases where any of the variables being solved for are not
dimensionless, it is essential that an initial guess with appropriate units be
supplied, otherwise the solver will try and fail to solve the system of
equations using dimensionless values:

\begin{verbatim}
x = unit(m)
y = 5*unit(km)
solve x=y via x
\end{verbatim}

The {\tt solve} command works by minimising the squares of the residuals of all
of the equations supplied, and so even when no exact solution can be found, the
best compromise is returned. The following example has no solution -- a system
of three equations with two variables is over-constrained -- but PyXPlot
nonetheless finds a compromise solution:

\begin{verbatim}
pyxplot> solve x+y=10, x-y=3, 2*x+y=16 via x,y
pyxplot> print x
6.3571429
pyxplot> print y
3.4285714
\end{verbatim}

When complex arithmetic is enabled, the {\tt solve} command allows each of the
variables being fitted to take any value in the complex plane, and thus the
number of dimensions of the fitting problem is effectively doubled -- the real
and imaginary components of each variable are solved for separately -- as in
the following example:

\begin{verbatim}
pyxplot> set numerics complex
pyxplot> solve exp(x)=e*i via x
pyxplot> print Re(x)
1
pyxplot> print Im(x)/pi
0.5
\end{verbatim}


\section{spline}\indcmd{spline}

\begin{verbatim}
spline [ <range> ] <function name>"()" '<filename>'
       [ every <expression> {:<expression>} ]
       [ index <value> ]
       [ select <expression> ]
       [ using <expression>  {:<expression>} ]
\end{verbatim}

The \indcmdt{spline} is an alias for the {\tt interpolate spline} command.
See the entry for the {\tt interpolate} command for more details.


\section{swap}\indcmd{swap}

\begin{verbatim}
swap <item1> <item2>
\end{verbatim}

Items on multiplot canvases are drawn in order of increasing identification
number, and thus items with low identification numbers are drawn first, at the
back of the multiplot, and items with higher identification numbers are later,
towards the front of the multiplot. When new items are added, they are given
higher identification numbers than previous items and appear at the front of
the multiplot.

If this is not the desired ordering, then the \indcmdt{swap} may be used to
rearrange items. It takes the identification numbers of two multiplot items and
swaps their identification numbers and hence their positions in the ordered
sequence.  Thus, if, for example, the corner of item~3 disappears behind the
corner of item~5, when the converse effect is actually desired, the following
command should be issued:
\begin{verbatim}
swap 3 5
\end{verbatim}


\section{tabulate}\indcmd{tabulate}

\begin{verbatim}
tabulate [ <range> ] ( <expression> | <filename> )
       [ every <expression> {:<expression>} ]
       [ index <value> ]
       [ select <expression> ]
       [ sortby <expression> ]
       [ using <expression>  {:<expression>} ]
       [ with <output format> ]
\end{verbatim}

PyXPlot's \indcmdt{tabulate} is similar to its {\tt plot} command, but instead
of plotting a series of \datapoint s onto a graph, it outputs them to \datafile
s. This can be used to produce text files containing samples of functions, to
rearrange/filter the columns in \datafile s, to change the units in which data
is expressed in \datafile s, and so forth.  The following example would produce
a \datafile\ called {\tt gamma.dat} containing a list of values of the gamma
function:

\begin{verbatim}
set output 'gamma.dat'
tabulate [1:5] gamma(x)
\end{verbatim}

\noindent Multiple functions may be tabulated into the same file, either by
using the \indmodt{using} modifier:

\begin{verbatim}
tabulate [0:2*pi] sin(x):cos(x):tan(x) u 1:2:3:4
\end{verbatim}

\noindent or by placing them in a comma-separated list, as in the {\tt plot}
command:

\begin{verbatim}
tabulate [0:2*pi] sin(x), cos(x), tan(x)
\end{verbatim}

In the former case, the functions are tabulated horizontally alongside one
another in a series of columns. In the latter case, the functions are tabulated
one after another in a series of index blocks separated by double linefeeds
(see Section~\ref{sec:plot_datafiles}).

The setting {\tt samples} can be used to control the number of points that are
produced when tabulating functions, in the same way that it controls the {\tt
plot} command:\indcmd{set samples}

\begin{verbatim}
set samples 200
\end{verbatim}

\noindent If the ordinate axis is set to be logarithmic then the points at which
functions are evaluated are spaced logarithmically, otherwise they are spaced
linearly.

The \indmodt{select}, \indmodt{using} and \indmodt{every} modifiers operate in
the same manner in the {\tt tabulate} command as in the {\tt plot} command.
Thus, the following example would write out the third, sixth and ninth columns
of the \datafile\ {\tt input.dat}, but only when the arcsine of the value in the
fourth column is positive:

\begin{verbatim}
set output 'filtered.dat'
tabulate 'input.dat' u 3:6:9 select (asin($4)>0)
\end{verbatim}

The numerical display format used in each column of the output file is chosen
automatically to preserve accuracy whilst simultaneously being as easily human
readable as possible.  Thus, columns which contain only integers are displayed
as such, and scientific notation is only used in columns which contain very
large or very small values.  If desired, however, a format statement may be
specified using the {\tt with format} specifier. The syntax for this is similar
to that expected by the string substitution operator ({\tt \%}; see
Section~\ref{sec:stringsubop}). As an example, to tabulate the values of $x^2$
to very many significant figures with some additional text, one could use:

\begin{verbatim}
tabulate x**2 with format "x = %f ; x**2 = %27.20e"
\end{verbatim}

\noindent This might produce the following output:

\begin{verbatim}
x = 0.000000 ; x**2 =  0.00000000000000000000e+00
x = 0.833333 ; x**2 =  6.94444444444442421371e-01
x = 1.666667 ; x**2 =  2.77777777777778167589e+00
\end{verbatim}

The data produced by the {\tt tabulate} command can be sorted in order of any
arbitrary metric by supplying an expression after the {\tt sortby} modifier;
where such expressions are supplied, the data is sorted in order from the
smallest value of the expression to the largest.


\section{text}\indcmd{text}

\begin{verbatim}
text [ item <id> ] '<text string>' [ at <x>, <y> ]
     [ rotate <angle> ] [ gap <gap> ]
     [ halign <alignment> ] [ valign <alignment> ]
     [ with colour <colour> ]
\end{verbatim}

The \indcmdt{text} allows strings of text to a added as labels on multiplot
canvases. The example

\begin{verbatim}
text 'Hello World!' at 0,2
\end{verbatim}

\noindent would render the text `Hello World!' at position $(0,2)$, measured in
centimetres. The alignment of the text item with respect to this position can
be set using the {\tt set texthalign} and {\tt set textvalign} commands, or
using the {\tt halign} and {\tt valign} modifiers supplied to the {\tt text}
command itself.

A gap may be specified, which should either have dimensions of length, or be
dimensionless, in which case it is interpreted as being measured in
centimetres. If a gap is specified, then the text string is slightly displaced
from the specified position, in the direction in which it is being aligned.

A rotation angle may optionally be specified after the keyword {\tt rotate}
to produce text rotated to any arbitrary angle, measured in degrees
counter-clockwise. The following example would produce upward-running text:

\begin{verbatim}
text 'Hello' at 1.5, 3.6 rotate 90
\end{verbatim}

By default the text is black; however, an arbitrary colour may be specified
using the {\tt with colour} modifier.  For example:

\begin{verbatim}
text 'A purple label' at 0, 0 with colour purple
\end{verbatim}

\noindent would add a purple label at the origin of the multiplot.

Outside of multiplot mode, the text command can be used to produce images
consisting simply of one single text item. This can be useful for importing
\LaTeX ed equations as gif images into slideshow programs such as Microsoft
Powerpoint which are incapable of producing such neat mathematical notation
by themselves.

All vector graphics objects placed on multiplot canvases receive unique
identification numbers which count sequentially from one, and which may be
listed using the {\tt list} command.  By reference to these numbers, they can
be deleted and subsequently restored with the {\tt delete} and {\tt undelete}
commands respectively.


\section{undelete}\indcmd{undelete}

\begin{verbatim}
undelete <item number> { , <item number> }
\end{verbatim}

The \indcmdt{undelete} allows vector graphics objects which have previously
been deleted from the current multiplot canvas to be restored.  The item(s)
which are to be restored should be identified using the reference number(s)
which were used to delete them, and can be queried using the {\tt list}
command. The example

\begin{verbatim}
undelete 1
\end{verbatim}

\noindent would cause the previously deleted item numbered {\tt 1} to reappear.


\section{unset}\indcmd{unset}

\begin{verbatim}
unset <setting>
\end{verbatim}

The \indcmdt{unset} causes a configuration option which has been changed using
the {\tt set} command to be returned to its default value.  For example:

\begin{verbatim}
unset linewidth
\end{verbatim}

\noindent returns the linewidth to its default value.

Any keyword which can follow the {\tt set} command to identify a configuration
parameter can also follow the {\tt unset} command; a complete list of these can
be found under the {\tt set} command.


\section{while}\indcmd{while}

\begin{verbatim}
while <condition> [loopname <loopname>]
 "{"
    ...
 "}"
\end{verbatim}

The \indcmdt{while} executes a block of commands repeatedly, checking the
provided condition at the start of each iteration. If the condition is true,
the loop executes again. This is similar to a {\tt do} loop, except that the
contents of a {\tt while} loop may not be executed at all if the iteration
criterion tests false upon the first iteration. For example, the following code
prints out the low-valued Fibonacci numbers:

\begin{verbatim}
i = 1
j = 1
while (j < 50)
 {
  print j
  i = i + j
  print i
  j = j + i
 }
\end{verbatim}

