% PLOTTING.TEX
%
% The documentation in this file is part of PyXPlot
% <http://www.pyxplot.org.uk>
%
% Copyright (C) 2006-2010 Dominic Ford <coders@pyxplot.org.uk>
%               2009-2010 Ross Church
%
% $Id$
%
% PyXPlot is free software; you can redistribute it and/or modify it under the
% terms of the GNU General Public License as published by the Free Software
% Foundation; either version 2 of the License, or (at your option) any later
% version.
%
% You should have received a copy of the GNU General Public License along with
% PyXPlot; if not, write to the Free Software Foundation, Inc., 51 Franklin
% Street, Fifth Floor, Boston, MA  02110-1301, USA

% ----------------------------------------------------------------------------

% LaTeX source for the PyXPlot Users' Guide

\chapter{Plotting: A Detailed Survey}
\label{ch:plotting}

Having described PyXPlot's mathematical environment and data-processing
facilities in detail, this part of the manual now returns to the subject of how
PyXPlot may be used to produce graphs and other vector graphics. In this
chapter we return to the \indcmdt{plot} to provide a more systematic survey of
how the appearance of plots can be configured, continuing from the brief
overview already given in Chapter~\ref{ch:first_steps}.  In the subsequent
chapters of this part, we will go on to describe how to produce graphical
output in a range of image formats (Chapter~\ref{ch:image_formats}) and how to
produce galleries of multiple plots side-by-side together with more
sophisticated vector graphics (Chapter~\ref{ch:vector_graphics}).

\section{The {\tt with} modifier}

Chapter~\ref{ch:first_steps} provided an overview of the syntax of the
\indcmdt{plot}, including the {\tt every}, {\tt index}, {\tt select} and {\tt
using} modifiers which can be used to control {\tt which} data, read from a
\datafile\ or sampled from a function, should be plotted. The {\tt with}
modifier controls {\tt how} data should be plotted. For example, the statement
\begin{verbatim}
plot "data.dat" index 1 using 4:5 with lines
\end{verbatim}
specifies that data should be plotted using lines connecting each \datapoint to
its neighbours. More generally, the {\tt with} modifier can be followed by a
range of settings which fine-tune the manner in which the data are displayed;
for example, the statement
\begin{verbatim}
plot "data.dat" with lines linewidth 2.0
\end{verbatim}
would use twice the default width of line.

In the following section, a complete list of all of PyXPlot's plot styles --
i.e.\ all of the words which may be used in place of {\tt lines} -- will be
given. In this section, we list all of the modifiers such as {\tt line\-width}
which may be used to alter the exact appearance of these plot styles, which are
as follows:
\begin{itemize}
\item \indmodt{colour} -- used to select the colour in which each dataset is to be plotted. It should be followed either by an integer, to set a colour from the present palette (see Section~\ref{sec:palette}), or by a recognised colour name, a complete list of which can be found in Section~\ref{sec:colour_names}. Alternatively, arbitrary colours may be specified by using one of the forms {\tt rgb0.1:\-0.2:\-0.3}, {\tt hsb0.1:\-0.2:\-0.3} or {\tt cmyk0.4:\-0.3:\-0.2:\-0.1}, where the colon-separated values indicate the RGB, HSB or CMYK components of the desired colour in the range~0 to~1. This modifier may also be spelt {\tt color}.\index{colours!setting for datasets}
\item \indmodt{fillcolour} -- used to select the colour in which each dataset is filled. This is not applicable to any of the plot styles listed above, but is included here for completeness. The colour may be specified using any of the styles listed for {\tt colour}. May also be spelt {\tt fillcolour}.
\item \indmodt{linetype} -- used to select the type of line -- for example, solid, dotted, dashed, etc.\ -- which should be used in line-based plot styles. A complete list of PyXPlot's numbered line types can be found in Chapter~\ref{ch:linetypes_table}. May be abbreviated {\tt lt}.
\item \indmodt{linewidth} -- used to select the width of line, where~1 represent the default width, which should be used in line-based plot styles. May be abbreviated {\tt lw}.
\item \indmodt{pointlinewidth} -- used to select the width of line, where~1 represent the default width, which should be used to stroke points in point-based plot styles. May be abbreviated {\tt plw}.
\item \indmodt{pointsize} -- used to select the size of drawn points, where~1 represent the default size. May be abbreviated {\tt ps}.
\item \indmodt{pointtype} -- used to select the type of point -- for example, crosses, circles, etc.\ -- used by point-based plot styles. A complete list of PyXPlot's numbered point types can be found in Chapter~\ref{ch:linetypes_table}. May be abbreviated {\tt pt}.
\end{itemize}

Any number of these modifiers may be placed sequentially after the {\tt with}
keyword, as in the following examples:

\begin{verbatim}
plot 'datafile' using 1:2 with points pointsize 2
plot 'datafile' using 1:2 with lines colour red linewidth 2
plot 'datafile' using 1:2 with lp col 1 lw 2 ps 3
\end{verbatim}

\noindent Where modifiers take numerical values, expressions of the form {\tt
\$2+1}, similar to those supplied to the {\tt using} modifier, may be used to
indicate that each datapoint should be displayed in a different style or in a
different colour. The following example would plot a \datafile\ with {\tt
points}, drawing the position of each point from the first two columns of the
supplied \datafile and the size of each point from the third column:
\begin{verbatim}
plot 'datafile' using 1:2 with points pointsize $3
\end{verbatim}

Not all of these modifiers are applicable to all of PyXPlot's plot styles. For
example, the {\tt line\-width} modifier has no effect on plot styles which do
not draw lines between datapoints. Where modifiers are applied to plot styles
for which they have no defined effect, the modifier has no effect, but no error
results.  Table~\ref{tab:style_modifiers} lists which modifiers act upon which
plot styles.

\begin{table}
\centerline{\includegraphics[width=\textwidth]{examples/eps/ex_plotstyletab.eps}}
\caption{A list of the plot styles affected by each style modifiers.}
\label{tab:style_modifiers}
\end{table}

\subsection{The Palette}
\label{sec:palette}

\index{palette}\index{colours!setting the palette} As indicated above, colours
may be referred to either specifically by name or RGB components, or by their
numbers in the current palette. By default, PyXPlot's palette contains a series
of visually distinctive colours which are insofar as possible are also
distinctive to users with most common forms of colour blindness. The current
palette may be queried using the \indcmdt{show palette}, and changed using the
\indcmdt{set palette}, which takes a comma-separated list of colours, as in the
example:

\begin{verbatim}
set palette BrickRed, LimeGreen, CadetBlue
\end{verbatim}

\noindent The palette is treated as a cyclic list, and so in the above example,
colour number~4 would map to {\tt BrickRed}, as would colour number~0. A list
of all of the named colours which PyXPlot recognises is given in
Section~\ref{sec:colour_names}. The default palette which PyXPlot uses upon
startup may be changed by setting up a configuration file, as described in
Chapter~\ref{ch:configuration}.

\subsection{Default Settings}

In addition to setting these parameters on a per-dataset basis, the {\tt
linewidth}, {\tt pointlinewidth} and {\tt pointsize} settings can also have
their default values changed for all datasets as in the following examples:
\begin{verbatim}
set linewidth 1
set pointlinewidth 2
set pointsize 3
plot "datafile"
\end{verbatim}
In each case, the normal default values of these settings are~1. The default
values of the {\tt colour}, {\tt linetype} and {\tt pointtype} settings depend
upon whether the current graphic output device is set to produce colour or
monochrome output (see Chapter~\ref{sec:set_terminal}). In the former case
(colour output), the colours of each of the comma-separated datasets plotted on
a graph are drawn sequentially from the currently-selected palette, all lines
are drawn as solid lines ({\tt line\-type~1}), and the symbols used to draw
each dataset are drawn sequentially from PyXPlot's available point types. In
the latter case (monochrome output), all datasets are plotted in black, and
both the line types and point types used to draw each dataset are drawn
sequentially from PyXPlot's available options. The following simple example
demonstrates this:
\begin{verbatim}
set terminal colour
plot [][6:0] 1 with lp, 2 with lp, 3 w lp, 4 w lp, 5 w lp
set terminal monochrome
replot
\end{verbatim}
\centerline{\includegraphics[width=\textwidth]{examples/eps/ex_col_vs_mono.eps}}

\section{PyXPlot's Plot Styles}

This section provides an exhaustive list of all of PyXPlot's {\it plot styles},
which we place into a series of groups for clarity.

\subsection{Lines and Points}

The following is a list of PyXPlot's simplest plot styles, all of which take as
input two columns of data, representing the $x$- and $y$-coordinates of the
positions of each point:
\begin{itemize}
\item \indpst{dots} -- places a small dot at each datum.
\item \indpst{lines} -- connects adjacent \datapoint s with straight lines.
\item \indpst{linespoints} -- a combination of both lines and points.
\item \indpst{lowerlimits} -- places a lower-limit sign (\includegraphics{examples/eps/ex_lowerlimit.eps}) at each datum.\index{lower-limit datapoints}
\item \indpst{points} -- places a marker symbol at each datum.
\item \indpst{stars} -- similar to {\tt points}, but uses a different set of marker symbols, based upon the stars drawn in Johann Bayer's highly ornate star atlas {\tt Uranometria} of 1603.
\item \indpst{upperlimits} -- places an upper-limit sign (\includegraphics{examples/eps/ex_upperlimit.eps}) at each datum.\index{upper-limit datapoints}
\end{itemize}

\example{ex:hrdiagram}{A Hertzsprung-Russell Diagram}{
Hertzsprung-Russell (HR) Diagrams are scatter-plots of the luminosities of
stars plotted against their colour which reveal that most normal stars lie
along a tight line called the main sequence, whilst unusual classes of stars --
giants and dwarfs -- can be readily identified on account of their not lying
along this main sequence. The principal difficulty in constructing accurate HR
diagrams is that the luminosities $L$ of stars can only be calculated from
their observed brightnesses $F$, using the relation $L=Fd^2$ if their distances
$d$ are known. In this example, we construct an HR diagram using observations
made by the European Space Agency's {\tt Hipparcos} spacecraft, which
accurately measured the distances of over a million stars between 1989 and
1993.
\nlnp
The Hipparcos catalogue can be downloaded for free from
\url{ftp://cdsarc.u-strasbg.fr/pub/cats/I/239/hip_main.dat.gz}; a description
of the catalogue can be found at
\url{http://cdsarc.u-strasbg.fr/viz-bin/Cat?I/239}. In summary, though the data
is arranged in a non-standard format which PyXPlot cannot read without a
special input filter, the following Python script generates a text file with
four columns containing the magnitudes $m$, $B-V$ colours and parallaxes $p$ of
the stars, together with the uncertainties in the parallaxes. From these
values, the absolute magnitudes $M$ of the stars -- a measure of their
luminosities -- can be calculated using the expression
$M=m+5\log_{10}\left(10^{2}p\right)$, where $p$ is measured in
milli-arcseconds:
\nlscf
\noindent{\tt for line in open("hip\_main.dat"):}\newline
\noindent{\tt \phantom{x}try:}\newline
\noindent{\tt \phantom{xx}Vmag  = float(line[41:46])}\newline
\noindent{\tt \phantom{xx}BVcol = float(line[245:251])}\newline
\noindent{\tt \phantom{xx}parr  = float(line[79:86])}\newline
\noindent{\tt \phantom{xx}parre = float(line[119:125])}\newline
\noindent{\tt \phantom{xx}print "\%s,\%s,\%s,\%s"\%(Vmag, BVcol, parr, parre)}\newline
\noindent{\tt \phantom{x}except ValueError: pass}
\nlscf
The resultant four columns of data can then be plotted in the {\tt dots} style
using the following PyXPlot script. Because the catalogue is very large, and
many of the parallax datapoints in it have large errorbars producing large
uncertainties in their vertical positions on the plot, we use the {\tt select}
statement to pick out those datapoints with parallax signal-to-noise ratios of
better than~20.
\nlscf
\noindent{\tt set nokey}\newline
\noindent{\tt set size square}\newline
\noindent{\tt set xlabel '\$B-V\$ colour'}\newline
\noindent{\tt set ylabel 'Absolute magnitude \$M\$'}\newline
\noindent{\tt plot [-0.4:2][14:-4] 'hr\_data.dat' $\backslash$}\newline
\noindent{\tt \phantom{xxxxx}using \$2:(\$1+5*log10(1e2*\$3)) $\backslash$}\newline
\noindent{\tt \phantom{xxxxx}select (\$4/\$3<0.05) $\backslash$}\newline
\noindent{\tt \phantom{xxxxx}with dots ps 3}
\nlscf
\centerline{\includegraphics[width=10cm]{examples/eps/ex_hrdiagram.eps}}
}

\subsection{Error Bars}
\index{errorbars}\label{sec:errorbars}

The following pair of plot styles allow datapoints to be plotted with errorbars
indicating the uncertainties in either their vertical or horizontal positions:
\begin{itemize}
\item \indpst{yerrorbars}
\item \indpst{xerrorbars}
\end{itemize}
Both of these plot styles take three columns of input data, the first two of
which represent the $x$- and $y$-coordinates of the positions of each point,
and the last of which represents the uncertainty in either the $x$- and
$y$-coordinate.  The plot style \indpst{errorbars} is an alias for
\indpst{yerrorbars}.  Additionally, the following plot style allows datapoints
to be plotted with both horizontal and vertical errorbars:
\begin{itemize}
\item \indpst{xyerrorbars}
\end{itemize}
This plot style takes four columns of data as input, the first two of which
represent the $x$- and $y$-coordinates of the positions of each point, the
third of which gives the uncertainty in the $x$-coordinate, and the last of
which gives the uncertainty in the $y$-coordinate.

Each of the plot styles listed above has a corresponding partner which takes
minimum and maximum limit in place of each uncertainty, equivalent to writing
$5^{+2}_{-3}$ instead of $5\pm2$, except that the limits of~2 and~7 should be
given in place of $5-3$ and $5+2$:
\begin{itemize}
\item \indpst{xerrorrange} -- takes four columns of data.
\item \indpst{yerrorrange} -- takes four columns of data.
\item \indpst{xyerrorrange} -- takes six columns of data.
\end{itemize}
The plot style \indpst{errorrange} is an alias of \indpst{yerrorrange}.

Corresponding plot styles also exist to plot data with errorbars along the
$z$-axes of three-dimensional plots\footnote{These plot styles are currently
present for future expansion purposes only, as PyXPlot~0.8.0 cannot produce
three-dimensional plots.}: {\tt zerrorbars}, {\tt zerrorrange}, {\tt
xzerrorbars}, {\tt xzerrorrange}, {\tt yzerrorbars}, {\tt yzerrorrange}, {\tt
xyzerrorbars}, {\tt xyzerrorrange}.

\subsection{Shaded Regions}

The following plot styles allow regions of graphs to be shaded with colour:
\begin{itemize}
\item \indpst{yerrorshaded}
\item \indpst{shadedregion}
\end{itemize}

Both of these plot styles fill specified regions of graphs with the selected
{\tt fillcolour} and draw a line around the boundary of the region with the
selected {\tt colour}, {\tt linetype} and {\tt linewidth}. They differ in the
format in which they expect the input data to be arranged. The
\indpst{yerrorshaded} plot style expects data to be arranged in the same format
as the \indpst{yerrorrange} plot style, specifying the $x$- and $y$-coordinates
of a series of \datapoint s in the first two columns, together with the minimum
and maximum extremes of the vertical errorbar on each \datapoint in the third
and fourth columns. The region contained between the upper and lower limits of
the error bars is filled with colour; the $y$-coordinate specified in the
second column is unused. This plot style provides easy conversion between plots
drawn with errorbars and with shaded error regions. Note that the \datapoint s
must be sorted in order of either increasing or decreasing $x$-coordinate for
sensible behaviour.

The \indpst{shadedregion} plot style takes only two columns of input data,
specifying the $x$- and $y$-coordinates of a series of \datapoint s which are
to be joined in a join-the-dots fashion. At the end of each dataset, the drawn
path is closed and filled.

\subsection{Barcharts and Histograms}
\label{sec:barcharts}
\index{bar charts}

The following plot styles allow barcharts to be produced:

\begin{itemize}
\item \indpst{boxes}
\item \indpst{impulses}
\item \indpst{wboxes}
\end{itemize}

\noindent These styles differ in where the horizontal interfaces between the
bars are placed along the $x$-axis and how wide the bars are.  In the
\indpst{boxes} plot style, the interfaces between the bars are at the midpoints
between the specified \datapoint s by default (see, for example,
Figure~\ref{fig:ex_barchart2}a).  Alternatively, the widths of the bars may be
set using the {\tt set boxwidth} command. In this case, all of the bars will be
centred upon their specified $x$-coordinates, and have total widths equalling
that specified in the \indcmdt{set boxwidth}. Consequently, there may be gaps
between them, or they may overlap, as seen in Figure~\ref{fig:ex_barchart2}(b).

\begin{figure}
\begin{center}
\includegraphics[width=\textwidth]{examples/eps/ex_barchart2.eps}
\end{center}
\caption[A gallery of the various bar chart styles which PyXPlot can produce]
{A gallery of the various bar chart styles which PyXPlot can produce.
See the text for more details.  The script and data file used to produce this
image are available on the PyXPlot website at
\protect\url{http://www.pyxplot.org.uk/examples/Manual/04barchart2/}.}
\label{fig:ex_barchart2}
\end{figure}

Having set a fixed box width, the default behaviour of scaling box widths
automatically may be restored either with the {\tt unset boxwidth} command,
or by setting the boxwidth to a negative width.

In the \indpst{wboxes} plot style, the width of each bar is specified manually
as an additional column of the input \datafile.  This plot style expects three
columns of data to be provided: the $x$- and $y$-coordinates of each bar in the
first two, and the width of the bars in the third.
Figure~\ref{fig:ex_barchart2}(c) shows an example of this plot style in use.

Finally, in the \indpst{impulses} plot style, the bars all have zero width; see
Figure~\ref{fig:ex_barchart1}(c) for an example.

In all of these plot styles, the bars originate from the line $y=0$ by default,
as is normal for a histogram. However, should it be desired for the bars to
start from a different vertical point, this may be achieved by using the
\indcmdt{set boxfrom}, for example:

\begin{verbatim}
set boxfrom 5
\end{verbatim}

\noindent In this case, all of the bars would now originate from the line
$y=5$. Figure~\ref{fig:ex_barchart1}(1) shows the kind of effect that is
achieved; for comparison, Figure~\ref{fig:ex_barchart1}(b) shows the same bar
chart with the boxes starting from their default position of $y=0$.

\begin{figure}
\begin{center}
\includegraphics[width=\textwidth]{examples/eps/ex_barchart1.eps}
\end{center}
\caption[A second gallery of the various bar chart styles which PyXPlot can
produce]
{A second gallery of the various bar chart styles which PyXPlot can
produce. See the text for more details.  The script and data file used to
produce this image are available on the PyXPlot website at
\protect\url{http://www.pyxplot.org.uk/examples/Manual/03barchart1/}.}
\label{fig:ex_barchart1}
\end{figure}

The bars may be filled using the {\tt with} \indmodt{fillcolour} modifier,
followed by the name of a colour:

\begin{verbatim}
plot 'data.dat' with boxes fillcolour blue
plot 'data.dat' with boxes fc 4
\end{verbatim}

\noindent Figures~\ref{fig:ex_barchart2}(b) and (d) demonstrate the use of
filled bars.

\subsubsection{Stacked Bar Charts}

If multiple \datapoint s are supplied to the \indpst{boxes} or \indpst{wboxes}
plot styles at a common $x$-coordinate, then the bars are stacked one above
another into a stacked barchart. Consider the following \datafile:

\begin{verbatim}
1 1
2 2
2 3
3 4
\end{verbatim}

\noindent The second bar at $x=2$ would be placed on top of the first, spanning
the range $2<y<5$, and having the same width as the first. If plot colours are
being automatically selected from the palette, then a different palette colour
is used to plot the upper bar.

\subsection{Steps}

The following plot styles allow data to be plotted with a series of horizontal
steps associated with each supplied \datapoint:
\begin{itemize}
\item \indpst{steps}
\item \indpst{fsteps}
\item \indpst{histeps}
\end{itemize}
Like the {\tt points} plot style, each of these styles take only two columns of
data as input, containing the $x$- and $y$-coordinates of each \datapoint.  An
example of their appearance  is shown in Figures~\ref{fig:ex_barchart1}(d), (e)
and (f); for clarity, the positions of each of the supplied \datapoint s are
marked by red crosses.  These plot styles differ in their placement of the
edges of each of the horizontal steps.  The \indpst{steps} plot style places
the right-most edge of each step on the \datapoint\ it represents.  The
\indpst{fsteps} plot style places the left-most edge of each step on the
\datapoint\ it represents.  The \indpst{histeps} plot style centres each step
upon the \datapoint\ it represents.

\subsection{Arrows}

The following plot styles allow arrows or lines to be drawn on graphs with
positions dictated by a series of \datapoint s:
\begin{itemize}
\item \indpst{arrows\_head}
\item \indpst{arrows\_nohead}
\item \indpst{arrows\_twohead}
\end{itemize}
The plot style of \indpst{arrows} is an alias for \indpst{arrows\_head}.  Each
of these plot styles take four columns of data -- $x_1$, $y_1$, $x_2$ and $y_2$
-- and each \datapoint\ results in an arrow being drawn from the point
$(x_1,y_1)$ to the point $(x_2,y_2)$. The three plot styles differ in the kinds
of arrows that they draw. \indpst{arrows\_head} draws an arrow head at the
point $(x_2,y_2)$; \indpst{arrows\_nohead} draws a simple line without arrow
heads on either end; \indpst{arrows\_twohead} draws arrow heads on both ends of
the arrow.

\example{ex:vortex}{A diagram of fluid flow around a vortex}{
In this example we produce a velocity map of fluid circulating in a vortex. For
simplicity, we assume that the fluid in the core of the vortex, at radii $r<1$,
is undergoing solid body rotation with velocity $v\propto r$, and that the
fluid outside this core is behaving as a free vortex with velocity $v\propto
1/r$. First of all, we use a simple python script to generate a \datafile\ with
the four columns:
\nlscf
\noindent{\tt from math import *}\newline
\noindent{\tt for i in range(-19,20,2):}\newline
\noindent{\tt \phantom{x}for j in range(-19,20,2):}\newline
\noindent{\tt \phantom{xx}x = float(i)/2}\newline
\noindent{\tt \phantom{xx}y = float(j)/2}\newline
\noindent{\tt \phantom{xx}r = sqrt(x**2 + y**2) / 4}\newline
\noindent{\tt \phantom{xx}theta = atan2(y,x)}\newline
\noindent{\tt \phantom{xx}if (r $<$ 1.0): v = 1.3*r}\newline
\noindent{\tt \phantom{xx}else        : v = 1.3/r}\newline
\noindent{\tt \phantom{xx}vy = v *  cos(theta)}\newline
\noindent{\tt \phantom{xx}vx = v * -sin(theta)}\newline
\noindent{\tt \phantom{xx}print "\%7.3f \%7.3f \%7.3f \%7.3f"\%(x,y,vx,vy)}
\nlscf
This data can then be plotted using the following PyXPlot script:
\nlscf
\noindent{\tt set size square}\newline
\noindent{\tt set nokey}\newline
\noindent{\tt set xlabel 'x'}\newline
\noindent{\tt set ylabel 'y'}\newline
\noindent{\tt set trange [0:2*pi]}\newline
\noindent{\tt plot $\backslash$}\newline
\noindent{\tt \phantom{x}'data' u 1:2:(\$1+\$3):(\$2+\$4) w arrows, $\backslash$}\newline
\noindent{\tt \phantom{x}parametric 4*sin(t):4*cos(t) w lt 2 col black}
\nlscf
\centerline{\includegraphics[width=10cm]{examples/eps/ex_vortex.eps}}
}

\section{The {\tt style} Keyword}

At times, the string of style keywords placed after the {\tt with} modifier in
{\tt plot} commands can grow rather unwieldy in its length. For clarity,
frequently used plot styles can be stored as numbered plot {\it styles}.  The
syntax for setting a numbered plot style is:

\begin{verbatim}
set style 2 points pointtype 3
\end{verbatim}

\noindent where the {\tt 2} is the identification number of the style. In a
subsequent {\tt plot} statement, this style can be recalled as follows:

\begin{verbatim}
plot sin(x) with style 2
\end{verbatim}

\section{Plotting Functions in Exotic Styles}

The use of plot styles which take more than two columns of input data to plot
functions requires more than one function to be supplied.  When functions are
plotted with syntax such as

\begin{verbatim}
plot sin(x) with lines
\end{verbatim}

\noindent two columns of data are generated: the first contains values of $x$
-- plotted against the horizontal axis -- and the second contains values of
$\sin(x)$ -- plotted against the vertical axis. Syntax such as

\begin{verbatim}
plot f(x):g(x) with yerrorbars
\end{verbatim}

\noindent generates three columns of data. As before, the first contains values
of $x$. The second and third contain samples from the colon-separated functions
$f(x)$ and $g(x)$. Specifically, in this example, $g(x)$ provides the
uncertainty in the value of $f(x)$.  The {\tt using} modifier may also be used
in combination with such syntax, as in

\begin{verbatim}
plot f(x):g(x) using 2:3
\end{verbatim}

\noindent though this example is not sensible. $g(x)$ would be plotted on the
$y$-axis, against $f(x)$ on the $x$-axis. However, this is unlikely to be
sensible because the range of values of $x$ substituting into these expressions
would correspond to the range of the plot's horizontal axis. The result might
be particularly unexpected if the above were attempted with an autoscaling
horizontal axis -- PyXPlot would find itself autoscaling the $x$-axis range to
the spread of values of $f(x)$, but find that this itself changed depending
upon the range of the $x$-axis. In this case, the user should have used the
{\tt parametric} plot option described in the next section.

\section{Plotting Parametric Functions}
\label{sec:parametric_plotting}

Parametric functions are functions expressed in forms such as
\begin{eqnarray*}
x & = & r \sin(t)  \\
y & = & r \cos(t) ,\\
\end{eqnarray*}
where separate expression are supplied for the horizontal and vertical
positions which correspond to the value of the function as a function of some
free parameter $t$. The above example is a parametric representation of a
circle of radius $r$. Before PyXPlot can usefully plot parametric functions, it
is generally necessary to stipulate the range of values of $t$ over which the
function should be sampled. This may be done using the \indcmdt{set trange}, as
in the example
\begin{verbatim}
set trange [unit(0*rad):unit(2*pi*rad)]
\end{verbatim}
or in the {\tt plot} command itself. By default, values in the range $0\leq
t\leq1$ are used. Note that the \indcmdt{set trange} differs from other
commands for setting axis ranges in that auto-scaling is not an allowed
behaviour; an explicit range {\it must} be specified for $t$.

Having set an appropriate range for $t$, parametric functions may be plotted by
placing the keyword {\tt parametric} before the list of functions to be
plotted, as in the following simple example which plots a circle:
\begin{verbatim}
set trange [unit(0*rev):unit(1*rev)]
plot parametric sin(t):cos(t)
\end{verbatim}
Optionally, a range for $t$ can be specified on a plot-by-plot basis
immediately after the keyword {\tt parametric}, and thus the effect above could
also be achieved using:
\begin{verbatim}
plot parametric [unit(0*rev):unit(1*rev)] sin(t):cos(t)
\end{verbatim}
The only difference between parametric function plotting and ordinary function
plotting -- other than the change of dummy variable from {\tt x} to {\tt t} --
is that one fewer column of data is generated. Thus, whilst
\begin{verbatim}
plot f(x)
\end{verbatim}
generates two columns of data, with values of $x$ in the first column,
\begin{verbatim}
plot parametric f(t)
\end{verbatim}
generates only one column of data.

\example{ex:spirograph}{Spirograph patterns}{
Spirograph patterns are produced when a pen is tethered to the end of a rod
which rotates at some angular speed $\omega_1$ about the end of another rod,
which is itself rotating at some angular speed $\omega_2$ about a fixed central
point. Spirographs are commonly implemented mechanically as wheels within
wheels -- epicycles within deferents, mathematically speaking -- but in this
example we implement them using the parametric functions
\begin{eqnarray*}
x & = & r_1 \sin(t) + r_2 \sin(t r_1 / r_2) \\
y & = & r_1 \cos(t) + r_2 \cos(t r_1 / r_2) \\
\end{eqnarray*}
which are simply the sum of two circular motions with angular velocities
inversely proportional to their radii. The complexity of the resulting
spirograph pattern depends upon how rapidly the rods return to their starting
configuration; if the two chosen angular speeds for the rods have a large
lowest common multiple, then a highly complicated pattern will result. In the
example below, we pick a ratio of $8:15$:
\nlscf
\noindent{\tt r1 = 1.5}\newline
\noindent{\tt r2 = 0.8}\newline
\noindent{\tt set size square}\newline
\noindent{\tt set trange[0:40*pi]}\newline
\noindent{\tt set samples 2500}\newline
\noindent{\tt plot parametric r1*sin(t) + r2*sin(t*(r1/r2)) : $\backslash$}\newline
\noindent{\tt \phantom{xxxxxxxxxxxxxxxx}r1*cos(t) + r2*cos(t*(r1/r2))}
\nlscf
\centerline{\includegraphics[width=8cm]{examples/eps/ex_spirograph.eps}}
\nlscf
Other ratios of {\tt r1}:{\tt r2} such as $7:19$ and $5:19$ also produce
intricate patterns.
}

\section{Graph Legends}
\index{keys}\index{legends}
\label{sec:legends}

By default, plots are displayed with legends in their top-right corners. The
textual description of each dataset is auto-generated from the command used
to plot it. Alternatively, the user may specify his own description for each
dataset by following the {\tt plot} command with the \indmodt{title} modifier,
as in the following examples:

\begin{verbatim}
plot sin(x) title 'A sine wave'
plot cos(x) title ''
\end{verbatim}

In the latter case a blank title is specified, which indicates to PyXPlot that
no entry should be made for the dataset in the legend. This allows for legends
which contain only a subset of the datasets on a plot. Alternatively, the
production of the legend can be completely turned off for all datasets using
the command \indcmdts{set nokey}. Having issued this command, the production of
keys can be resumed using the \indcmdt{set key}.

The \indcmdt{set key} can also be used to dictate how legends should be
positioned on graphs, using a syntax along the lines of:

\begin{verbatim}
set key top right
\end{verbatim}

The following recognised positioning keywords are self-explanatory:
\indkeyt{top}, \indkeyt{bottom}, \indkeyt{left}, \indkeyt{right},
\indkeyt{xcentre} and \indkeyt{ycentre}. Any single instance of the
\indcmdt{set key} can be followed by one horizontal alignment keyword and one
vertical alignment keyword; these keywords also affect the justification of the
legend -- for example, the keyword \indkeyt{left} aligns the legend with its
left edge against the left edge of the plot.

Alternatively, the position of the legend can be indicated using one of the
keywords \indkeyt{outside}, \indkeyt{below} or \indkeyt{above}. These cannot be
combined with the horizontal and vertical alignment keywords above, and are
used to indicate that the legend should be placed, respectively, outside the
plot on its right side, centred beneath the plot, and centred above the plot.

Two positional offset coordinates may be specified any of the named positions
listed above -- the first value is assumed to be a horizontal offset and the
second a vertical offset. Either may have units of length, or, if they are
dimensionless, are assumed to be measured in centimetres, as the following
examples demonstrate:

\begin{verbatim}
set key bottom left 0.0 -2
set key top xcentre 2*unit(mm),2*unit(mm)
\end{verbatim}

By default, entries in the legend are automatically sorted into an appropriate
number of columns. The number of columns to be used, can, instead, be
stipulated using the \indcmdt{set keycolumns}. This should be followed by
either the integer number of desired columns, or by the keyword {\tt auto} to
indicate that the default behaviour of automatic formatting should be resumed:

\begin{verbatim}
set keycolumns 2
set keycolumns auto
\end{verbatim}

\section{Configuring Axes}
\label{sec:multiple_axes}

By default, plots have only one horizontal $x$-axis and one vertical $y$-axis.
Additional axes may be added parallel to these, which are referred to as, for
example, the {\tt x2} axis, the {\tt x3} axis, and so forth up to a maximum of
{\tt x127}.  In keeping with this nomenclature, the first axis in each
direction can be referred to interchangably as, for example, {\tt x} or {\tt
x1}, or as {\tt y} or {\tt y1}.  Further axes are automatically generated when
statements such as the following are issued:

\begin{verbatim}
set x3label 'A second horizontal axis'
\end{verbatim}

\noindent Such axes may alternatively be created explicitly using the
\indcmdt{set axis}, as in the example

\begin{verbatim}
set axis x3
\end{verbatim}

\noindent or removed explicitly using the \indcmdt{unset axis}, as in the
example

\begin{verbatim}
unset axis x3
\end{verbatim}

\noindent The first axes {\tt x1} and {\tt y1} -- and {\tt z1} on
three-dimensional plots -- are unique in that they cannot be removed; all plots
must have at least one axis in each perpendicular direction.  Thus, the command
{\tt unset axis x1} does not remove this first axis, but merely returns it to
its default configuration. In either case, multiple axes can be created or
removed in a single statement, as in the examples

\begin{verbatim}
unset axis x3x5x6 y2
set axis x2y2
\end{verbatim}

It should be noted that if the following two commands are typed in succession,
the second may not entirely negate the first:

\begin{verbatim}
set x3label 'foo'
unset x3label 'foo'
\end{verbatim}

\noindent If an $x3$-axis did not previously exist, then the first will have
implicitly created one. This would need to be removed with the {\tt unset axis
x3} command if it was not desired.

\subsection{Selecting which Axes to Plot Against}

The axes against which data are plotted can be selected by passing the {\tt
axes} modifier to the {\tt plot} command. By default, data is plotted against
the first horizontal axis and the first vertical axis. In the following {\tt
plot} command the second horizontal axis and the third vertical axis would be
used:
\begin{verbatim}
plot f(x) axes x2y3
\end{verbatim}
It is also possible to plot data using a vertical axis as the abscissa axis
using syntax such as:
\begin{verbatim}
plot f(x) axes y3x2
\end{verbatim}
Similar syntax is used when plotting three-dimensional graphs, except that
three axes should be specified.

\subsection{Plotting Quantities with Physical Units}
\label{sec:set_axisunitstyle}

When data with non-dimensionless physical units are plotted against an axis,
for example using any of the statements
\begin{verbatim}
plot [0:10] x*unit(m)
plot [0:10] x using 1:$2*unit(m)
plot [0:1] asin(x)
\end{verbatim}
the axis is set to share the particular physical dimensions of that unit, and
thereafter no data with any other physical dimensions may be plotted against
that axis. When the axis comes to be drawn, PyXPlot makes a decision about
which physical unit should be used to label the axis. For example, in the
default SI system and with no preferred unit of length set, axes with units of
length might be displayed in millimetres, metres or kilometres depending upon
their scales.

The chosen unit is indicated in one of three styles in the axis label, selected
using the \indcmdt{set axisunitstyle}:
\begin{verbatim}
set axisunitstyle ratio
set axisunitstyle bracketed
set axisunitstyle squarebracketed
\end{verbatim}
The effect of these three options, respectively, is shown below for an axis
with units of momentum. In each case, the axis label was set simply using
\begin{verbatim}
set xlabel "Momentum"
\end{verbatim}
and the subsequent text was appended automatically by PyXPlot:

\vspace{3mm}
\centerline{\includegraphics[width=10cm]{examples/eps/ex_axisunits.eps}}
\vspace{3mm}

When the \indcmdt{set xformat} is used (see Section~\ref{sec:set_xformat}), no
indication of the units associated with axes are appended to axis labels, as
the \indcmdt{set xformat} can be used to hard-code this information. The user
must include this information in the axis label manually if it is needed.

\subsection{Specifying the Positioning of Axes}

By default, the {\tt x1} axis is placed along the bottom of graphs and the {\tt
y1} axis is placed up the left-hand side of graphs. On three-dimensional plots,
the {\tt z1} axis is placed at the front of the graph. The second set of axes
are placed opposite to the first: the {\tt x2}, {\tt y2} and {\tt z2} axes are
placed respectively along the top, right and back sides of graphs.
Higher-numbered axes are placed alongside the {\tt x1}, {\tt y1} and {\tt z1}
axes.

However, the position of any axis can be explicitly set using syntax of the
form:
\begin{verbatim}
set axis x top
set axis y right
set axis z back
\end{verbatim}
Horizontal axes can be set to appear either at the {\tt top} or {\tt bottom};
vertical axes can be set to appear either at the {\tt left} or {\tt right}; and
$z$-axes can be set to appear either at the {\tt front} or {\tt back}.

\subsection{Configuring the Appearance of Axes}

The \indcmdt{set axis} also accepts the following keywords alongside the
positioning keywords listed above, which specify how the axis should appear:
\begin{itemize}
\item {\tt arrow} -- Specifies that an arrowhead should be drawn on the right/top end of the axis. [{\bf Not default}].
\item {\tt atzero} -- Specifies that rather than being placed along an edge of the plot, the axis should mark the lines where the perpendicular axes {\tt x1}, {\tt y1} and/or {\tt z1} are zero. [{\bf Not default}].
\item {\tt automirrored} -- Specifies that an automatic decision should be made between the behaviour of {\tt mirrored} and {\tt nomirrored}. If there are no axes on the opposite side of the graph, a mirror axis is produced. If there are already axes on the opposite side of the graph, no mirror axis is produced. [{\bf Default}].
\item {\tt fullmirrored} -- Similar to {\tt mirrored}. Specifies that this axis should have a corresponding twin placed on the opposite side of the graph with mirroring ticks and labelling. [{\bf Not default}; see {\tt automirrored}].
\item {\tt invisible} -- Specifies that the axis should not be drawn; data can still be plotted against it, but the axis is unseen. See Example~\ref{ex:australia} for a plot where all of the axes are invisible.
\item {\tt linked} -- Specifies that the axis should be linked to another axis; see Section~\ref{sec:linked_axes}.
\item {\tt mirrored} -- Specifies that this axis should have a corresponding twin placed on the opposite side of the graph with mirroring ticks but with no labels on the ticks. [{\bf Not default}; see {\tt automirrored}].
\item {\tt noarrow} -- Specifies that no arrowheads should be drawn on the ends of the axis. [{\bf Default}].
\item {\tt nomirrored} -- Specifies that this axis should not have any corresponding twins. [{\bf Not default}; see {\tt automirrored}].
\item {\tt notatzero} -- Opposite of {\tt atzero}; the axis should be placed along an edge of the plot. [{\bf Default}].
\item {\tt notlinked} -- Specifies that the axis should no longer be linked to any other; see Section~\ref{sec:linked_axes}. [{\bf Default}].
\item {\tt reversearrow} -- Specifies that an arrowhead should be drawn on the left/bottom end of the axis. [{\bf Not default}].
\item {\tt twowayarrow} -- Specifies that arrowheadsshould be drawn on both ends of the axis. [{\bf Not default}].
\item {\tt visible} -- Specifies that the axis should be displayed; opposite of {\tt invisible}. [{\bf Default}].
\end{itemize}

The following simple examples demonstrate the use of some of these configuration options:
\begin{verbatim}
set axis x atzero twoway
set axis y atzero twoway
plot [-2:8][-10:10]
\end{verbatim}

\centerline{\includegraphics[width=8cm]{examples/eps/ex_axisatzero.eps}}

\begin{verbatim}
set axis x atzero arrow
set axis y atzero twoway
plot [0:10][-10:10]
\end{verbatim}

\centerline{\includegraphics[width=8cm]{examples/eps/ex_axisatzero2.eps}}

\begin{verbatim}
set axis x notatzero arrow nomirror
set axis y notatzero arrow nomirror
plot [0:10][0:20]
\end{verbatim}

\centerline{\includegraphics[width=8cm]{examples/eps/ex_axisatzero3.eps}}

\subsection{Specifying where Ticks should Appear along Axes}

By default, PyXPlot places a series of tick marks at significant points along
each axis, with the most significant points being labelled.  Labelled tick
marks are termed {\it major} ticks, and unlabelled tick marks are termed {\it
minor} ticks.  The position and appearance of the major ticks along the {\tt
x}-axis can be configured with the \indcmdt{set xtics}, which has the following
syntax:

\begin{verbatim}
set xtics { axis | border | inward | outward | both }
          {  autofreq
           | <increment>
           | { <minimum>, } <increment> { , <maximum> }
           | (     {"label"} <position>
               { , {"label"} <position> } .... )
          }
\end{verbatim}

The corresponding command {\tt set mxtics}, which has the same syntax as above,
configures the appearance of the minor ticks along the {\tt x}-axis. Analogous
commands such as {\tt set ytics} and {\tt set mx2tics} configure the major and
minor ticks along other axes.

The keywords \indkeyt{inward}, \indkeyt{outward} and \indkeyt{both} are used to
configure how the ticks appear -- whether they point inward, towards the plot,
as is default, or outwards towards the axis labels, or in both directions.  The
keyword \indkeyt{axis} is an alias for \indkeyt{inward}, and \indkeyt{border}
an alias for \indkeyt{outward}.

The remaining options are used to configure where along the axis ticks are
placed. If a series of comma-separated values {\tt <minimum>, <increment>,
<maximum>} are specified, then ticks are placed at evenly spaced intervals
between the specified limits. The {\tt <minimum>} and {\tt <maximum>} values
are optional; if only one value is specified then it is taken to be the step
size between ticks. If two values are specified, then the first is taken to be
{\tt <minimum>}. In the case of logarithmic axes, {\tt <increment>} is applied
as a multiplicative step size.

Alternatively, if a bracketed list of quoted tick labels and tick positions are
provided, then ticks can be placed on an axis individually and each given its
own textual label. The quoted tick labels may be omitted, in which case they
are automatically generated:
\begin{verbatim}
set xtics ("a" 1, "b" 2, "c" 3)
set xtics (1,2,3)
\end{verbatim}
The keyword \indkeyt{autofreq} overrides any manual selection of ticks which
may have been placed on an axis and resumes the automatic placement of ticks
along the axis. The \indcmdt{show xtics} command, together with its companions
such as {\tt show x2tics} and {\tt show ytics}, may be used to query the
current ticking options. The \indcmdt{set noxtics} may be used to stipulate
that no ticks should appear along a particular axis:

\begin{verbatim}
set noxtics
show xtics
\end{verbatim}

\example{ex:axistics}{A plot of the function $\exp(x)\sin(1/x)$}{
In this example we produce a plot illustrating some of the crossing points of
the function $\exp(x)\sin(1/x)$.  We set the $x$-axis to have tick marks at
$x=0.05$, $0.1$, $0.2$ and $0.4$.  The $x2$-axis has custom labelled ticks at
$x=1/\pi, 2/\pi$, etc., pointing outwards from the plot.  The left-hand
$y$-axis has tick marks placed automatically whereas the $y2$-axis has no tics
at all.
\nlscf
\noindent{\tt set log x1x2}\newline
\noindent{\tt set xrange [0.05:0.5]}\newline
\noindent{\tt set axis x2 top linked x}\newline
\noindent{\tt set xtics 0.05, 2, 0.4}\newline
\noindent{\tt set x2tics border $\backslash$}\newline
\noindent{\tt \phantom{xxxxx}("\$$\backslash$frac\{1\}\{$\backslash$pi\}\$" 1/pi,      "\$$\backslash$frac\{1\}\{2$\backslash$pi\}\$" 1/(2*pi), $\backslash$}\newline
\noindent{\tt \phantom{xxxxxx}"\$$\backslash$frac\{1\}\{3$\backslash$pi\}\$" 1/(3*pi), "\$$\backslash$frac\{1\}\{4$\backslash$pi\}\$" 1/(4*pi), $\backslash$}\newline
\noindent{\tt \phantom{xxxxxx}"\$$\backslash$frac\{1\}\{5$\backslash$pi\}\$" 1/(5*pi), "\$$\backslash$frac\{1\}\{6$\backslash$pi\}\$" 1/(6*pi))}\newline
\noindent{\tt set grid x2}\newline
\noindent{\tt set nokey}\newline
\noindent{\tt set xlabel '\$x\$'}\newline
\noindent{\tt set ylabel '\$$\backslash$exp(x)$\backslash$sin(1/x)\$'}\newline
\noindent{\tt plot exp(x)*sin(1/x), 0}
\nlscf
\centerline{\includegraphics[width=9cm]{examples/eps/ex_axistics.eps}}
}

\subsection{Configuring how Tick Marks are Labelled}
\label{sec:set_xformat}

By default, major ticks are labelled with representations of the ordinate
values at each point, each accurate to the number of significant figures
specified using the \indcmdt{set numerics sigfig}. These labels may appear as
decimals, such as $3.142$, in scientific notion, as in $3\times10^8$, or, on
logarithmic axes where a base has been specified for the logarithms using
syntax such as \footnote{Note that the {\tt x} axis must be referred to as {\tt
x1} here to distinguish this statement from {\tt set log x2}.}
\begin{verbatim}
set log x1 2
\end{verbatim}
in a format such as $1.5\times2^8$.

The \indcmdt{set xformat} -- together with its companions such as {\tt set
yformat}\footnote{There is no {\tt set mxformat} command since minor axis ticks
are never labelled unless labels are explicitly provided for them using the
syntax {\tt set mxtics (...)}.} -- may be used to specify a format for the axis
labels to take, as demonstrated by the following pair of examples:
\begin{verbatim}
set xformat "%.2f"%(x)
set yformat "%s$^\prime$"%(y/unit(feet))
\end{verbatim}
The first example specifies that ordinate values should be displayed to two
decimal places along the $x$-axis; the second specifies that distances should
be displayed in feet along the $y$-axis. Note that the dummy variable used to
represent the ordinate value is {\tt x}, {\tt y} or {\tt z} depending upon the
direction of the axis, but that the dummy variable used in the {\tt set
x2format} command is still {\tt x}. The following pair of examples both have
the equivalent effect of returning the {\tt x2}-axis to its default system of
tick labels:
\begin{verbatim}
set x2format auto
set x2format "%s"%(x)
\end{verbatim}

The following example specifies that ordinate values should be displayed as
multiples of $\pi$:
\begin{verbatim}
set xformat "%s$\pi$"%(x/pi)
plot [-pi:2*pi] sin(x)
\end{verbatim}

\noindent\centerline{\includegraphics[width=8cm]{examples/eps/ex_axistics2.eps}}

Note that where possible, PyXPlot intelligently changes the positions where it
places the ticks along the axis to reflect significant points in the chosen
labelling system.  The extent to which this is possible depends upon the format
string supplied. It is generally easier when continuous-varying numerical
values are substituted into strings, rather than discretely-varying values or
strings. Thus, rather than
\begin{dontdo}
set xformat "%d"%(floor(x))
\end{dontdo}
the following is preferred
\begin{dontdo}
set xformat "%d"%(x)
\end{dontdo}
and rather than
\begin{dontdo}
set xformat "%s"%time_string(x)
\end{dontdo}
the following is preferred
\begin{dontdo}
set xformat "%d/%02d/%d"%(time_day(x),time_monthnum(x),time_daymonth(x))
\end{dontdo}

\subsubsection{Changing the Slant of Axis Labels}

The \indcmdt{set xformat} and its companions may also be followed by keywords
which control the angle at which tick labels are drawn. By default, all tick
labels are written horizontally, a behaviour which may be reproduced by issuing
the command:
\begin{verbatim}
set xformat horizontal
\end{verbatim}
Alternatively, tick labels may be set to be written vertically, by issuing the command
\begin{verbatim}
set xformat vertical
\end{verbatim}
or to be written at any clockwise rotation angle from the horizontal using commands of the form
\begin{verbatim}
set xformat rotate unit(10*deg)
\end{verbatim}

Axis labels may also be made to appear at arbitrary rotations using commands such as
\begin{verbatim}
set xlabel "I'm upside down" rotate unit(0.5*revolution)
\end{verbatim}

\subsection{Linked Axes}
\label{sec:linked_axes}

Often it may be desired that multiple axes on a graph share a common range, or
be related to one another by some algebraic expression. For example, a plot
with wavelength $\lambda$ of light as one axis may usefully also have parallel
axes showing frequency of light $\nu=c/\lambda$ or photon energy
$E=hc/\lambda$. The following example sets the {\tt x2} axis to share a common
range with the {\tt x} axis:
\begin{verbatim}
set axis x2 linked x
\end{verbatim}
An algebraic relationship between two axes may be set by stating the algebraic
relationship after the keyword {\tt using}, as in the following example which
implement the formulae shown above for the frequency and energy of photons of
light as a function of their wavelength:
\begin{verbatim}
set xrange [200*unit(nm):unit(800*nm)]
set axis x2 linked x1 using phy\_c/x
set axis x3 linked x2 using phy\_h*x
\end{verbatim}
As in the {\tt set xformat} command, a dummy variable of {\tt x}, {\tt y} or
{\tt z} is used in the linkage expression depending upon the direction of the
axis being linked to, but a dummy variable of {\tt x} is still used when
linking to, for example, the {\tt x2} axis.

As these examples demonstrate, the functions used to link axes need not be
linear. In fact, axes with any arbitrary mapping between distance and ordinate
value can be produced by linked in a non-linear fashion to another linear axis,
which, if desired, can then be hidden using the {\tt set axis invisible}
command. Multi-valued mappings are also permitted. Any data plotted against the
following {\tt x2}-axis for a suitable range of {\tt x}-axis
\begin{verbatim}
set axis x2 linked x1 using x**2
\end{verbatim}
would appear twice, symmetrically on either side of $x=0$.

Insofar as is possible, linked axes autoscale intelligently when no range is
set.  Thus, if the {\tt x2}-axis is linked to the {\tt x}-axis, and no range to
set for the {\tt x}-axis, the {\tt x}-axis will autoscale to include all of the
data plotted against both itself and the {\tt x2}-axis. Similarly, if the {\tt
x2}-axis is linked to the {\tt x}-axis by means of some algebraic expression,
the {\tt x}-axis will attempt to autoscale to include the data plotted against
the {\tt x2}-axis, though in some cases -- especially with non-monotonic
linking functions -- this may prove too difficult. Where PyXPlot detects that
it has failed, a warning is issued recommending that a hard range be set for --
in this example -- the {\tt x}-axis.

\example{ex:multiaxes}{A plot of many blackbodies demonstrating the use of linked axes}{
In this example we produce a plot of blackbody spectra for five different
temperatures $T$, using the Planck formula
\begin{displaymath}
B_\nu(\nu,T)=\left(\frac{2h^3}{c^2}\right)\frac{\nu^3}{\exp(h\nu/kT)-1}
\end{displaymath}
which is evaluated in PyXPlot by the system-defined mathematical function {\tt
Bv(nu,T)}. We use the axis linkage commands listed as an example in the text of
Section~\ref{sec:linked_axes} to produce three parallel horizontal axes showing
wavelength of light, frequency of light and photon energy.
\nlscf
\noindent{\tt set numeric display latex}\newline
\noindent{\tt set log x y}\newline
\noindent{\tt set key bottom right}\newline
\noindent{\tt set ylabel "Flux density / W/Hz/m\$\^{}2\$/sterad"}\newline
\noindent{\tt set x1label "Wavelength"}\newline
\noindent{\tt set x2label "Frequency"~~~~~; set unit of frequency Hz}\newline
\noindent{\tt set x3label "Photon Energy" ; set unit of energy eV}\newline
\noindent{\tt set axis x2 linked x1 using phy\_c/x}\newline
\noindent{\tt set axis x3 linked x2 using phy\_h*x}\newline
\\
\noindent{\tt bb(wlen,T) = Bv(phy\_c/wlen,T)/unit(W/Hz/m\^{}2/sterad)}\newline
\\
\noindent{\tt plot [80*unit(nm):unit(mm)][1e-20:] $\backslash$}\newline
\noindent{\tt \phantom{xx}bb(x,~~30) title "\$T=~~30\$$\backslash$,K", $\backslash$}\newline
\noindent{\tt \phantom{xx}bb(x, 100) title "\$T= 100\$$\backslash$,K", $\backslash$}\newline
\noindent{\tt \phantom{xx}bb(x, 300) title "\$T= 300\$$\backslash$,K", $\backslash$}\newline
\noindent{\tt \phantom{xx}bb(x,1000) title "\$T=1000\$$\backslash$,K", $\backslash$}\newline
\noindent{\tt \phantom{xx}bb(x,3000) title "\$T=3000\$$\backslash$,K"}
\nlscf
\centerline{\includegraphics[width=10cm]{examples/eps/ex_multiaxes.eps}}
}

\example{ex:cmbrtemp}{A plot of the temperature of the CMB as a function of redshift demonstrating non-linear axis linkage}{
In this example we produce a plot of the temperature of the cosmic microwave
background radiation (CMBR) as a function of time $t$ since the Big Bang, on
the {\tt x}-axis, and equivalently as a function of redshift $z$, on the {\tt
x2}-axis.  The specialist cosmology function
\indfunt{ast\_\-Lcdm\_\-z($t$,\-$H_0$,\-$\Omega_\mathrm{M}$,\-$\Omega_\Uplambda$)}
is used to make the highly non-linear conversion between time $t$ and redshift
$z$, adopting some standard values for the cosmological parameters $H_0$,
$\Omega_\mathrm{M}$ and $\Omega_\Uplambda$. Because the temperature of the CMBR
is easiest expressed as a function of redshift as $T=2.73\,\mathrm{K}/(1+z)$,
we plot is using this function against axis {\tt x2}.
\nlscf
\noindent{\tt h0 = 70}\newline
\noindent{\tt omega\_m = 0.27}\newline
\noindent{\tt omega\_l = 0.73}\newline
\noindent{\tt age = ast\_Lcdm\_age(h0,omega\_m,omega\_l)}\newline
\noindent{\tt set xrange [0.01*age:0.99*age]}\newline
\noindent{\tt set unit of time Gyr}\newline
\noindent{\tt set axis x2 linked x using ast\_Lcdm\_z(age-x,h0,omega\_m,omega\_l)}\newline
\noindent{\tt set xlabel "Time since Big Bang \$t\$"}\newline
\noindent{\tt set ylabel "CMB Temperature \$T\$"}\newline
\noindent{\tt set x2label "Redshift \$z\$"}\newline
\noindent{\tt plot unit(2.73*K)*(1+x) axes x2y1}
\nlscf
\centerline{\includegraphics[width=8cm]{examples/eps/ex_cmbrtemp.eps}}
}

\section{Gridlines}

Gridlines may be placed on a plot and subsequently removed via the statements:

\begin{verbatim}
set grid
set nogrid
\end{verbatim}

\noindent respectively. The following commands are also valid:

\begin{verbatim}
unset grid
unset nogrid
\end{verbatim}

\noindent By default, gridlines are drawn from the major and minor ticks of the
default $x$- and $y$-axes (which are the first $x$- and $y$-axes unless set
otherwise in the configuration file; see Chapter~\ref{ch:configuration}).
However, the axes which should be used may be specified after the \indcmdt{set
grid}\index{grid}:

\begin{verbatim}
set grid x2y2
set grid x x2y2
\end{verbatim}

\noindent The top example would connect the gridlines to the ticks of the $x2$-
and $y2$-axes, whilst the lower would draw gridlines from both the $x$- and the
$x2$-axes.

If one of the specified axes does not exist, then no gridlines will be drawn in
that direction.  Gridlines can subsequently be removed selectively from some
axes via:

\begin{verbatim}
set nogrid x2x3
\end{verbatim}

The colours of gridlines\index{grid!colour}\index{colours!grid} can be
controlled via the \indcmdts{set gridmajcolour} and \indcmdts{set
gridmincolour} commands, which control the gridlines emanating from major and
minor axis ticks respectively. An example would be:

\begin{verbatim}
set gridmincolour blue
\end{verbatim}

\noindent Any of the colour names listed in Section~\ref{sec:colour_names} can
be used.

A related command\index{axes!colour}\index{colours!axes} is \indcmdts{set
axescolour}, which has a syntax similar to that above, and sets the colour of
the graph's axes.\label{sec:set_colours}

\section{Clipping Behaviour}

The treatment of datapoints close to the edges of plots may be specified using
the \indcmdt{set clip}, which provides two options. Either datapoints close to
the axes can be clipped and not allowed to overrun the axes -- specified by
{\tt set clip} -- or such datapoints may be allowed to extend over the lines of
the axes -- specified by {\tt set noclip} and the default behaviour.

\section{Labelling Graphs}

The \indcmdt{set arrow} and \indcmdt{set label}s allow arrows and text labels
to be added to graphs to label significant points or to add simple vector
graphics to them.

\subsection{Arrows}

\label{sec:set_arrow}\index{arrows} The \indcmdt{set arrow} may be used to add
draw arrows on top of graphs; its syntax is illustrated by the following simple
example:

\begin{verbatim}
set arrow 1 from 0,0 to 1,1
\end{verbatim}

\noindent The number {\tt 1} immediately following \indcmdts{set arrow}
specifies an identification number for the arrow, allowing it to be
subsequently removed via the command

\begin{verbatim}
unset arrow 1
\end{verbatim}

\noindent or equivalently, via\indcmd{set noarrow}

\begin{verbatim}
set noarrow 1
\end{verbatim}

\noindent or to be replaced with a different arrow by issuing a new command of
the form {\tt set arrow 1~...}.  The {\tt set arrow} command may be followed by
the keyword {\tt with} to specify the style of the arrow. The keywords
\indkeyt{nohead}, \indkeyt{head} and \indkeyt{twohead}, placed after the
keyword {\tt with}, can be used to generate arrows with no arrow heads, normal
arrow heads, or with two arrow heads.  \indkeyt{twoway} is an alias for
\indkeyt{twohead}, as in the following example:

\begin{verbatim}
set arrow 1 from 0,0 to 1,1 with twoway
\end{verbatim}

\noindent Line types and colours can also be specified after the keyword {\tt
with}, as in the example:

\begin{verbatim}
set arrow 1 from 0,0 to 1,1 with nohead \
linetype 1 c blue
\end{verbatim}

The coordinates for the start and end points of the arrow can be specified in
a range of coordinate systems. The coordinate system to be used should be
specified immediately before the coordinate value. The default system,
\indcot{first} measures the graph using the $x$- and $y$-axes. The
\indcot{second} system uses the $x2$- and $y2$-axes. The \indcot{screen} and
\indcot{graph} systems both measure in centimetres from the origin of the
graph. In the following example, we use these specifiers, and specify
coordinates using variables rather than doing so explicitly:

\begin{verbatim}
x0 = 0.0
y0 = 0.0
x1 = 1.0
y1 = 1.0
set arrow 1 from first  x0, first  y0 \
            to   screen x1, screen y1 \
            with nohead
\end{verbatim}

In addition to these four options, \indcot{axis{\it n}} specifies that the
position is to be measured along the $n\,$th $x$- or $y$-axis -- for example,
`{\tt axis3}'.\indcmd{set arrow} This allows the graph to be measured with
reference to any arbitrary axis on plots which make use of large numbers of
parallel axes (see Section~\ref{sec:multiple_axes}).

\subsection{Text Labels}

Text labels may be placed on plots using the \indcmdt{set label}. As with all
textual labels in PyXPlot, these are rendered in \LaTeX:

\begin{verbatim}
set label 1 'Hello World' at 0,0
\end{verbatim}

As in the previous section, the number {\tt 1} is a reference number, which
allows the label to be removed by either of the following two commands:

\begin{verbatim}
set nolabel 1
unset label 1
\end{verbatim}

\noindent The positional coordinates for the text label, placed after the {\tt
at} keyword, can be specified in any of the coordinate systems described for
arrows above. A rotation angle may optionally be specified after the keyword
\indkeyt{rotate}, to rotate text counter-clockwise by a given angle, measured
in degrees. For example, the following would produce upward-running text:

\begin{verbatim}
set label 1 'Hello World' at axis3 3.0, axis4 2.7 rotate 90
\end{verbatim}

A colour can also be specified, if desired, using the {\tt with colour}
modifier.  For example, the following would produce a green label at the origin:

\begin{verbatim}
set label 2 'This label is green' at 0, 0 with colour green
\end{verbatim}

\index{fontsize}\index{text!size} The size of the text in such labels can be set
globally using the \indcmdt{set fontsize}. This applies not only to the {\tt
set label} command, but also to plot titles, axis labels, keys, etc. The value
supplied should be a multiplicative factor greater than zero; a 
value of~{\tt 2} would cause text to be rendered at twice its normal size, and
a value of~{\tt 0.5} would cause text to be rendered at half its normal size.
The default value is one.

\index{text!colour}\index{colours!text} The \indcmdt{set textcolour} can be
used to globally set the colour of all text output, and applies to all of the
text that the {\tt set fontsize} command does. It is especially useful when
producing plots to be embedded in presentation slideshows, where bright text on
a dark background may be desired. It should be followed either by an integer,
to set a colour from the present palette, or by a colour name. A list of the
recognised colour names can be found in Section~\ref{sec:colour_names}.  For
example:

\begin{verbatim}
set textcolour 2
set textcolour blue
\end{verbatim}

\index{text!alignment}\index{alignment!text}By default, each label's specified
position corresponds to its bottom left corner. This alignment may be changed
with the \indcmdts{set texthalign} and \indcmdts{set textvalign} commands. The
former takes the options \indkeyt{left}, \indkeyt{centre} or \indkeyt{right},
and the latter takes the options \indkeyt{bottom}, \indkeyt{centre} or
\indkeyt{top}, for example:

\begin{verbatim}
set texthalign right
set textvalign top
\end{verbatim}

\example{ex:hlines}{A diagram of the atomic lines of hydrogen}{
The wavelengths of the spectral lines of atomic hydrogen are given by the Rydberg formula,
\begin{displaymath}
\frac{1}{\lambda} = R_\mathrm{H}\left(\frac{1}{n^2}-\frac{1}{m^2}\right),
\end{displaymath}
where $\lambda$ is wavelength, $R_\mathrm{H}$ is the Rydberg constant,
predefined in PyXPlot as the variable {\tt phy\_Ry}, and {\tt n} and {\tt m}
are positive non-zero integers such that {\tt m>n}. The first few series are
called the Lyman series ({\tt n}$=1$), the Balmer series ({\tt n}$=2$), the
Paschen series ({\tt n}$=3$) and the Brackett series ({\tt n}$=4$). Within each
series, the lines are given Greek letter designations -- $\alpha$ for {\tt
m}$=${\tt n}$+1$, $\beta$ for {\tt m}$=${\tt n}$+2$, and so forth.
\nlnp
In the following example, we produce a diagram of the lines in the first four
series, drawing the first~20 lines within each. At the bottom of the diagram,
we overlay indications of the wavelengths of ten colour filters commonly used
by astronomers (data taken from Binney \& Merrifield, {\it Galactic Astronomy},
Princeton, 1998).
\nlscf
{\footnotesize
\noindent{\tt set numeric display latex}\newline
\noindent{\tt set width 20}\newline
\noindent{\tt set size ratio 0.4}\newline
\noindent{\tt set numerics sf 4}\newline
\noindent{\tt set log x}\newline
\noindent{\tt set x1label "Wavelength"}\newline
\noindent{\tt set x2label "Frequency"     ; set unit of frequency Hz}\newline
\noindent{\tt set x3label "Photon Energy" ; set unit of energy eV}\newline
\noindent{\tt set axis x2 linked x1 using phy\_c/x}\newline
\noindent{\tt set axis x3 linked x2 using phy\_h*x}\newline
\noindent{\tt set noytics ; set nomytics}\newline
}\\{\footnotesize
\noindent{\tt \# Draw lines of first four series of hydrogen lines}\newline
\noindent{\tt an=2}\newline
\noindent{\tt n=1}\newline
\noindent{\tt foreach SeriesName in ("Ly","Ba","Pa","Br")}\newline
\noindent{\tt \phantom{x}\{}\newline
\noindent{\tt \phantom{xx}for m=n+1 to n+21}\newline
\noindent{\tt \phantom{xxx}\{}\newline
\noindent{\tt \phantom{xxxx}wl = 1/(phy\_Ry*(1/n**2-1/m**2))}\newline
\noindent{\tt \phantom{xxxx}set arrow an from wl,0.3 to wl,0.6 with nohead col n}\newline
\noindent{\tt \phantom{xxxx}if (m-n==1) \{ ; GreekLetter = "$\backslash$$\backslash$alpha" ; \}}\newline
\noindent{\tt \phantom{xxxx}if (m-n==2) \{ ; GreekLetter = "$\backslash$$\backslash$beta"  ; \}}\newline
\noindent{\tt \phantom{xxxx}if (m-n==3) \{ ; GreekLetter = "$\backslash$$\backslash$gamma" ; \}}\newline
\noindent{\tt \phantom{xxxx}if (m-n$<$4)}\newline
\noindent{\tt \phantom{xxxxx}\{}\newline
\noindent{\tt \phantom{xxxxxx}set label an "$\backslash$parbox\{5cm\}\{$\backslash$footnotesize$\backslash$center\{$\backslash$}\newline
\noindent{\tt \phantom{xxxxxxxx}\%s-\$\%s\$$\backslash$$\backslash$newline \$\%d$\backslash$to\%d\$$\backslash$$\backslash$newline \%s$\backslash$$\backslash$newline\}\}" $\backslash$}\newline
\noindent{\tt \phantom{xxxxxxxx}\%(SeriesName,GreekLetter,n,m,wl) at wl,0.55+0.2*(m-n) $\backslash$}\newline
\noindent{\tt \phantom{xxxxxxxx}hal centre val centre}\newline
\noindent{\tt \phantom{xxxxx}\}}\newline
\noindent{\tt \phantom{xxxx}an = an+1}\newline
\noindent{\tt \phantom{xxx}\}}\newline
\noindent{\tt \phantom{xx}n=n+1}\newline
\noindent{\tt \phantom{x}\}}\newline
}\\{\footnotesize
\noindent{\tt \# Label astronomical photometric colours}\newline
\noindent{\tt foreach datum i,name,wl\_c,wl\_w in "--" using $\backslash$}\newline
\noindent{\tt \phantom{xxxx}1:"\%s"\%(\$2):(\$3*unit(nm)):(\$4*unit(nm))}\newline
\noindent{\tt \phantom{x}\{}\newline
\noindent{\tt \phantom{xx}arry = 0.12+0.1*(i\%2) \# Vertical positions for arrows}\newline
\noindent{\tt \phantom{xx}laby = 0.07+0.1*(i\%2) \# Vertical positions for labels}\newline
\noindent{\tt \phantom{xx}x0 = (wl\_c-wl\_w/2) \# Shortward end of passband}\newline
\noindent{\tt \phantom{xx}x1 =  wl\_c         \# Centre of passband}\newline
\noindent{\tt \phantom{xx}x2 = (wl\_c+wl\_w/2) \# Longward end of passband}\newline
\noindent{\tt \phantom{xx}set arrow an from x0,arry to x2,arry with nohead}\newline
\noindent{\tt \phantom{xx}set label an name at x1,laby hal centre val centre}\newline
\noindent{\tt \phantom{xx}an = an+1}\newline
\noindent{\tt \phantom{x}\}}\newline
\noindent{\tt \phantom{x}1 U  365   66}\newline
\noindent{\tt \phantom{x}2 B  445   94}\newline
\noindent{\tt \phantom{x}3 V  551   88}\newline
\noindent{\tt \phantom{x}4 R  658  138}\newline
\noindent{\tt \phantom{x}5 I  806  149}\newline
\noindent{\tt \phantom{x}6 J 1220  213}\newline
\noindent{\tt \phantom{x}7 H 1630  307}\newline
\noindent{\tt \phantom{x}8 K 2190  390}\newline
\noindent{\tt \phantom{x}9 L 3450  472}\newline
\noindent{\tt 10 M 4750  460}\newline
\noindent{\tt END}\newline
}\\{\footnotesize
\noindent{\tt \# Draw a marker for the Lyman limit}\newline
\noindent{\tt ll = 91.1267*unit(nm)}\newline
\noindent{\tt set arrow 1 from ll,0.12 to ll,0.22}\newline
\noindent{\tt set label 1 "Lyman Limit: \%s"\%(ll) at 95*unit(nm),0.17 $\backslash$}\newline
\noindent{\tt \phantom{xxxxx}hal left val centre}\newline
}\\{\footnotesize
\noindent{\tt \# Finally produce plot}\newline
\noindent{\tt plot [80*unit(nm):5500*unit(nm)][0:1.25]}\newline
}
\nlscf
\begin{center}
\includegraphics[width=11cm]{examples/eps/ex_hlines.eps}
\end{center}
}

\newpage
\example{ex:australia}{A map of Australia}{
In this example, we use PyXPlot to plot a map of Australia, using a coastal
outline obtained from \protect\url{http://www.maproom.psu.edu/dcw/}. We use the
{\tt set label} command to label the states and major cities. The files {\tt
ex\_\-map\_\-1.dat.gz} and {\tt ex\_\-map\_\-2.dat} can be found in the PyXPlot
installation tarball in the directory {\tt doc/\-examples/}.
\nlscf
{\footnotesize
\noindent{\tt set width 20}\newline
}\\{\footnotesize
\noindent{\tt \# We want a plot without axes or key}\newline
\noindent{\tt set nokey}\newline
\noindent{\tt set axis x invisible}\newline
\noindent{\tt set axis y invisible}\newline
}\\{\footnotesize
\noindent{\tt \# Labels for the states}\newline
\noindent{\tt set label 1 '\{$\backslash$large $\backslash$sf $\backslash$slshape Western Australia\}' 117, -28}\newline
\noindent{\tt set label 2 '\{$\backslash$large $\backslash$sf $\backslash$slshape South Australia\}' 130, -29.5}\newline
\noindent{\tt set label 3 '\{$\backslash$large $\backslash$sf $\backslash$slshape Northern Territory\}' 129.5, -20.5}\newline
\noindent{\tt set label 4 '\{$\backslash$large $\backslash$sf $\backslash$slshape Queensland\}' 141,-24}\newline
\noindent{\tt set label 5 '\{$\backslash$large $\backslash$sf $\backslash$slshape New South Wales\}' 142,-32.5}\newline
\noindent{\tt set label 6 '\{$\backslash$large $\backslash$sf $\backslash$slshape Victoria\}' 139,-41}\newline
\noindent{\tt set arrow 6 from 141,-40 to 142, -37 with nohead}\newline
\noindent{\tt set label 7 '\{$\backslash$large $\backslash$sf $\backslash$slshape Tasmania\}' 149,-42}\newline
\noindent{\tt set arrow 7 from 149, -41.5 to 146.5, -41.75 with nohead}\newline
\noindent{\tt set label 8 '\{$\backslash$large $\backslash$sf $\backslash$slshape Capital Territory\}' 151,-37}\newline
\noindent{\tt set arrow 8 from 151, -36.25 to 149, -36 with nohead}\newline
}\\{\footnotesize
\noindent{\tt \# Labels for the cities}\newline
\noindent{\tt set label 10 '\{$\backslash$sf Perth\}' 116.5, -32.4}\newline
\noindent{\tt set label 11 '\{$\backslash$sf Adelaide\}' 136, -38}\newline
\noindent{\tt set arrow 11 from 137.5,-37.2 to 138.601, -34.929}\newline
\noindent{\tt set label 12 '\{$\backslash$sf Darwin\}' 131, -13.5}\newline
\noindent{\tt set label 13 '\{$\backslash$sf Brisbane\}' 149, -27.5}\newline
\noindent{\tt set label 14 '\{$\backslash$sf Sydney\}' 151.5, -34.5}\newline
\noindent{\tt set label 15 '\{$\backslash$sf Melbourne\}' 143, -37.3}\newline
\noindent{\tt set label 16 '\{$\backslash$sf Hobart\}' 147.5, -44.25}\newline
\noindent{\tt set label 17 '\{$\backslash$sf Canberra\}' 145, -35.25}\newline
}\\{\footnotesize
\noindent{\tt \# A big label saying "Australia"}\newline
\noindent{\tt set label 20 '\{$\backslash$Huge $\backslash$sf $\backslash$slshape Australia\}' 117,-42}\newline
}\\{\footnotesize
\noindent{\tt \# Plot the coastline and cities}\newline
\noindent{\tt plot [][-45:] 'ex\_map\_1.dat.gz' every ::1 with lines, $\backslash$}\newline
\noindent{\tt \phantom{xxxxx}'ex\_map\_2.dat' with points pointtype 10 pointsize 2}
}
\nlscf
\begin{center}
\includegraphics[width=\textwidth]{examples/eps/ex_map.eps}
\end{center}
}

% \section{Non-Flat Projections}

\section{Three-Dimensional Plotting}

PyXPlot 0.8.0 has no facilities for three-dimensional plotting. The ability to
configure axes such as {\tt z} and {\tt z2}, together with various references
to three-dimensional plotting in this manual, are provided for future
expansion only.

