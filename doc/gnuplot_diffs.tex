% GNUPLOT_DIFFS.TEX
%
% The documentation in this file is part of PyXPlot
% <http://www.pyxplot.org.uk>
%
% Copyright (C) 2006-8 Dominic Ford <coders@pyxplot.org.uk>
%               2008   Ross Church
%
% $Id$
%
% PyXPlot is free software; you can redistribute it and/or modify it under the
% terms of the GNU General Public License as published by the Free Software
% Foundation; either version 2 of the License, or (at your option) any later
% version.
%
% You should have received a copy of the GNU General Public License along with
% PyXPlot; if not, write to the Free Software Foundation, Inc., 51 Franklin
% Street, Fifth Floor, Boston, MA  02110-1301, USA

% ----------------------------------------------------------------------------

% LaTeX source for the PyXPlot Users' Guide

\chapter{Summary of Differences Between PyXPlot and \gnuplot}
\label{ch:gnuplot_diffs}

Rewrite Me.

Because PyXPlot is still work in progress, it does not implement all of the
features of \gnuplot. It currently does not implement any three-dimensional or
surface plotting -- i.e.\ the \indcmdt{splot} of \gnuplot. It also does not
support the plotting of parametric functions.

Some of \gnuplot's features have been significantly re-worked to improve upon
their operation. The prime example is \gnuplot's multiplot mode, which allows
multiple graphs to be placed side-by-side. While we retain a similar syntax, we
have made it significantly more flexible. The use of dual axes is another
example: PyXPlot now places no limit on the number of parallel horizontal and
vertical axes which may be drawn on a graph.

These extensions to \gnuplot's interface are described in detail in the
following chapters.

