% GNUPLOT_DIFFS.TEX
%
% The documentation in this file is part of PyXPlot
% <http://www.pyxplot.org.uk>
%
% Copyright (C) 2006-2010 Dominic Ford <coders@pyxplot.org.uk>
%               2009-2010 Ross Church
%
% $Id$
%
% PyXPlot is free software; you can redistribute it and/or modify it under the
% terms of the GNU General Public License as published by the Free Software
% Foundation; either version 2 of the License, or (at your option) any later
% version.
%
% You should have received a copy of the GNU General Public License along with
% PyXPlot; if not, write to the Free Software Foundation, Inc., 51 Franklin
% Street, Fifth Floor, Boston, MA  02110-1301, USA

% ----------------------------------------------------------------------------

% LaTeX source for the PyXPlot Users' Guide

\chapter{Summary of Differences Between PyXPlot and \gnuplot}
\label{ch:gnuplot_diffs}

Because PyXPlot is still work in progress, it does not implement all of the
features of \gnuplot. It currently does not implement any three-dimensional or
surface plotting -- i.e.\ the \indcmdt{splot} of \gnuplot. It also does not
support the plotting of parametric functions.

Some of \gnuplot's features have been significantly re-worked to improve upon
their operation. The prime example is \gnuplot's multiplot mode, which allows
multiple graphs to be placed side-by-side. While we retain a similar syntax, we
have made it significantly more flexible. The use of dual axes is another
example: PyXPlot now places no limit on the number of parallel horizontal and
vertical axes which may be drawn on a graph.

These extensions to \gnuplot's interface are described in detail in the
following chapters.




The principal difference between the two programs which \gnuplot\ users will
need to be aware of is that PyXPlot renders all of its text labels
automatically in the \LaTeX\ typesetting environment. This produces much neater
labels than the default typesetting engine used by \gnuplot, and makes it
straightforward to label graphs with mathematical expressions. Moreover, if you
plan to import your graphs into a \LaTeX\ document, it makes it straightforward
to match the fonts used in your figures with those used in the main text of
your document.

This does, however, necessarily introduce some incompatibility with \gnuplot,
since some strings which were valid before are no longer valid (see
Section~\ref{sec:latex_incompatibility} for more details). For
example:\index{latex}

\begin{dontdo}
set xlabel 'x\^{}2'
\end{dontdo}

\noindent would have been valid in \gnuplot, but is not valid input for \LaTeX:
it needs to be written in \LaTeX\ mathmode as:

\begin{dodo}
set xlabel '\$x\^{}2\$'
\end{dodo}

\noindent It is our view, though, that the nuisance of this incompatibility is
far outweighed by the power that \LaTeX\ brings, however. For users with no
prior knowledge of \LaTeX\ we recommend Tobias Oetiker's\index{Tobias Oetiker}
excellent introduction, {\it The Not So Short Guide to \LaTeX\
$2\epsilon$}\index{Not So Short Guide to \LaTeX\ $2\epsilon$,
The}\footnote{Download from:\\
\url{http://www.ctan.org/tex-archive/info/lshort/english/lshort.pdf}}. Later in
this manual, we will also show that PyXPlot's built-in  {\tt texify()} function
can also automatically convert strings of text into valid \LaTeX.

