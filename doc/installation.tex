% INSTALLATION.TEX
%
% The documentation in this file is part of PyXPlot
% <http://www.pyxplot.org.uk>
%
% Copyright (C) 2006-2010 Dominic Ford <coders@pyxplot.org.uk>
%               2008-2010 Ross Church
%
% $Id$
%
% PyXPlot is free software; you can redistribute it and/or modify it under the
% terms of the GNU General Public License as published by the Free Software
% Foundation; either version 2 of the License, or (at your option) any later
% version.
%
% You should have received a copy of the GNU General Public License along with
% PyXPlot; if not, write to the Free Software Foundation, Inc., 51 Franklin
% Street, Fifth Floor, Boston, MA  02110-1301, USA

% ----------------------------------------------------------------------------

% LaTeX source for the PyXPlot Users' Guide

\chapter{Installation}

\label{ch:installation}

In this chapter we outline how to install PyXPlot on a range of UNIX-like
operating systems.

\section{System Requirements}

PyXPlot works on most UNIX-like operating systems. We have tested it under
Linux, Solaris\index{Solaris} and MacOS~X\index{MacOS X}, and believe that it
should work on other similar POSIX systems. We regret that it is not available
for Microsoft Windows, and have no plans for porting it at this time. It
requires that the following software packages (not included) be
installed:\index{system requirements}

\vspace{0.5cm}
\begin{itemize}
\item fftw (version 2 or, preferably, 3+) \index{fftw}
\item Ghostscript \index{Ghostscript}
\item The Gnu Scientific Library (version 1.10+) \index{gsl}
\item ImageMagick \index{ImageMagick}
\item latex (version $2\epsilon$; a full installation is likely to be required in distributions which offer a choice) \index{latex}
\item libkpathsea \index{libkpathsea}
\item libpng (version 1.2+) \index{libpng}
\item libxml2 (version 2.6+) \index{libxml}
\item zlib \index{zlib}
\end{itemize}
\vspace{0.5cm}

\noindent It is also very strongly recommended that the following software
packages be installed. In each case, PyXPlot tests for the availability of
these packages upon installation, and issues a warning if the package is not
found. Installation can proceed, but some of PyXPlot's features will be
disabled. Once they have been added to the system, PyXPlot should be
reinstalled to take advantage of their presence.

\vspace{0.5cm}
\begin{itemize}
\item cfitsio -- required in order to plot \datafile s in FITS format.
\item Ghostview \index{Ghostview} -- required in order for PyXPlot to display plots live on the screen; PyXPlot remains able to generate image files on disk without it. Alternatively, any of the programs {\tt ggv}, {\tt evince} or {\tt okular} may be substituted for Ghostview, and the \indcmdt{set viewer} within PyXPlot allows a selection to be made between them, but the use of Ghostview is recommended on grounds of stability.
\item gunzip \index{gunzip} -- required in order to plot compressed \datafile s in {\tt .gz} format.
\item The Gnu Readline Library (version 5+) \index{readline} -- required in order to provide tab completion and command histories in PyXPlot's interactive command-line interface.
\item wget \index{wget} -- required in order to plot \datafile s directly from the internet.
\end{itemize}
\vspace{0.5cm}

Debian and Ubuntu users can find the above software in the following packages:
\index{Debian Linux}\index{Ubuntu Linux}\index{installation!under Debian}\index{installation!under Ubuntu}

\vspace{2mm}
\noindent {\tt fftw3-dev}, {\tt gs}, {\tt gv}, {\tt imagemagick}, {\tt libcfitsio3-dev}, {\tt libgsl0-dev},\newline
\noindent {\tt libkpathsea-dev}, {\tt libpng12-dev}, {\tt libreadline5-dev}, {\tt libxml2-dev},\newline
\noindent {\tt tetex-extra}, {\tt wget}, {\tt zlib1g-dev}.
\vspace{2mm}

\noindent These packages may be installed from a command prompt by typing, all on one line:

\begin{verbatim}
sudo apt-get install fftw3-dev gs gv imagemagick libcfitsio3-dev
       libgsl0-dev libkpathsea-dev libpng12-dev libreadline5-dev
       libxml2-dev tetex-extra wget zlib1g-dev
\end{verbatim}

\section{Installation}
\index{installation}

\subsection{Installation within Linux Distributions}

PyXPlot is available as a user-installable package within some Linux
distributions. At the time of writing, Gentoo\index{Gentoo
Linux}\index{installation!under Gentoo}\footnote{See
\url{http://gentoo-portage.com/sci-visualization/pyxplot}}, Ubuntu\index{Ubuntu
Linux}\index{installation!under Ubuntu} and Debian provide packages containing
PyXPlot~0.7.x. It should be noted that this manual describes PyXPlot~0.8.0,
which is a very substantial upgrade to version~0.7.x. Many of the features
described in this manual are not available in previous versions.

Under Debian and Ubuntu Linux, version~0.8.0 may be manually installed by
downloading the appropriate {\tt .deb} file for your distribution from the
PyXPlot website and typing:

\begin{verbatim}
dpkg -i <filename>.deb
\end{verbatim}

Users of other distributions should use the {\tt .tar.gz} archives available
from the PyXPlot website as described below.

\subsection{Installation as User}

The following steps describe the installation of PyXPlot from a {\tt .tar.gz}
archive by a user without superuser (i.e.\ root) access to his machine. The
required archive can be downloaded from the PyXPlot website. It is assumed that
the packages listed above have already been installed; if they are not, you
will need to contact your system administrator.\index{installation!user-level}

\begin{itemize}
\item Unpack the distributed .tar.gz:

\begin{verbatim}
tar xvfz pyxplot_0.8.0.tar.gz
cd pyxplot
\end{verbatim}

\item Run the installation script:

\begin{verbatim}
./configure
make
\end{verbatim}

\item Finally, start PyXPlot:

\begin{verbatim}
./bin/pyxplot
\end{verbatim}

\end{itemize}

\subsection{System-wide Installation}

Having completed the steps described above, PyXPlot may be installed
system-wide by a superuser with the following additional
step:\index{installation!system-wide}

\begin{verbatim}
make install
\end{verbatim}

By default, the PyXPlot executable installs to {\tt /usr/local/bin/pyxplot}.
If desired, this installation path may be modified in the file {\tt
Makefile.skel}, by changing the variable {\tt USRDIR} in the first line to an
alternative desired installation location.

PyXPlot may now be started by any system user, simply by typing:

\begin{verbatim}
pyxplot
\end{verbatim}

